% Options for packages loaded elsewhere
\PassOptionsToPackage{unicode}{hyperref}
\PassOptionsToPackage{hyphens}{url}
\documentclass[
]{article}
\usepackage{xcolor}
\usepackage{amsmath,amssymb}

\usepackage{hyperref}
\usepackage{ulem}  % For underlining links
\normalem

\hypersetup{
    colorlinks=false,         % Disable colored links (boxes will appear)
    citebordercolor=red,      % Border color for citation links
    linkbordercolor=red,      % Border color for internal links
    urlbordercolor=blue,      % Border color for external links (URLs)
    pdfborder={0 0 2}         % Specifies the border thickness: {horizontal vertical thickness}
}


% Only underline external links (URLs)
\let\oldhref\href
\renewcommand{\href}[2]{\ifx#1\urlprefix\oldhref{#1}{#2}\else\uline{\oldhref{#1}{#2}}\fi}


\renewcommand{\[}{\begin{equation}}
\renewcommand{\]}{\end{equation}}


\setcounter{secnumdepth}{5}
\usepackage{iftex}
\ifPDFTeX
  \usepackage[T1]{fontenc}
  \usepackage[utf8]{inputenc}
  \usepackage{textcomp} % provide euro and other symbols
\else % if luatex or xetex
  \usepackage{unicode-math} % this also loads fontspec
  \defaultfontfeatures{Scale=MatchLowercase}
  \defaultfontfeatures[\rmfamily]{Ligatures=TeX,Scale=1}
\fi
\usepackage{lmodern}
\ifPDFTeX\else
  % xetex/luatex font selection
\fi
% Use upquote if available, for straight quotes in verbatim environments
\IfFileExists{upquote.sty}{\usepackage{upquote}}{}
\IfFileExists{microtype.sty}{% use microtype if available
  \usepackage[]{microtype}
  \UseMicrotypeSet[protrusion]{basicmath} % disable protrusion for tt fonts
}{}
\makeatletter
\@ifundefined{KOMAClassName}{% if non-KOMA class
  \IfFileExists{parskip.sty}{%
    \usepackage{parskip}
  }{% else
    \setlength{\parindent}{0pt}
    \setlength{\parskip}{6pt plus 2pt minus 1pt}}
}{% if KOMA class
  \KOMAoptions{parskip=half}}
\makeatother
\setlength{\emergencystretch}{3em} % prevent overfull lines
\providecommand{\tightlist}{%
  \setlength{\itemsep}{0pt}\setlength{\parskip}{0pt}}
\usepackage[]{biblatex}
\addbibresource{refs.bib}
\usepackage{bookmark}
\IfFileExists{xurl.sty}{\usepackage{xurl}}{} % add URL line breaks if available
\urlstyle{same}

\title{Nucleonics}
  \author{Matt Lilley}
  \date{\today}  % Default to today if no date is provided

\begin{document}
\maketitle

\section{What is nucleonics?}\label{what-is-nucleonics}

\begin{quote}
\textbf{Nucleonics is a scientific and engineering discipline that
studies and applies the principles of physics to design, create, and
operate devices that manipulate nucleons.}
\end{quote}

We take inspiration from the emergence of electronics that marked a
fundamental shift in humanity's relationship to information. Underlying
that shift was a revolution in the way we understood and interacted with
electrons: rather than continuing to blow large aggregates of electrons
through vacuum tubes, the transistor represented deliberate, precise
control of electronic states.

What if, similar to our control of electronic states, we gained precise
control of nuclear states? Before we can answer that question, it's
essential to first ask the ``Is it possible?'' question.

Some of the essential physics of nucleonics (e.g.~nuclear superradiance)
was already proposed in
\href{https://doi.org/10.1103/physrevlett.14.589}{1965 by Terhune and
Baldwin}, but it's only in recent years that it's been possible to
verify those ideas - see
\href{https://www.nature.com/articles/s41567-017-0001-z}{Chumakov et.al
2017}. We're now seeing increased interest and experimental progress in
the controlling of nuclear states
e.g.~\href{https://www.science.org/doi/10.1126/sciadv.abc3991}{Bocklage
et.al 2021} ,
\href{https://www.nature.com/articles/s41586-021-03276-x}{Heeg et.al
2021}, and
\href{https://www.nature.com/articles/s41567-024-02773-w}{Chai et.al
2025}.

The answer seems to be yes - nucleonics is possible. Now to the question
of ``What if?''.

\section{A vision for nucleonics}\label{a-vision-for-nucleonics}

What if, similar to our control of electronic states, we gained precise
control of nuclear states? We might imagine a world where
\href{https://nucleonics.substack.com/p/going-beyond-radiation}{nuclear
radiation is a thing of the past} or a world where we can increase
nuclear fusion rates in solid states to technologically relevant levels.

There are other possible visions for nucleonics, e.g.~one might take an
information technology angle relating to nuclear spintronics. However,
our vision takes an energy angle. We can recast the vision into a
``mission statement''

\begin{quote}
\textbf{We want to enable rational engineering of small-scale devices
fuelled by clean nuclear energy}
\end{quote}

Now that we have a ``destination'', how are we going to get there? Are
there any potential roadblocks along the way? We need a roadmap.

\section{Nucleonics Roadmap}\label{nucleonics-roadmap}

To meet the vision, we need nucleonics to be more than possible, we need
it to be practical and so the essential question that a nucleonics
roadmap is answering is:

\begin{quote}
\textbf{Is nucleonics practical?}
\end{quote}

It may be that manipulating nuclear states for the purposes of energy
technology is sufficiently difficult to make it more of an academic
curiosity than an enabler of new technology.

From the energy perspective that our mission statement focusses on,
practicality means that we need to be able to:

\begin{itemize}
\tightlist
\item
  Accelerate nuclear transitions by many orders of magnitude
\item
  Transfer nuclear energy non-radiatively between nuclei
\item
  Control nuclear transitions with low frequency stimulation
\item
  Extract nuclear energy in benign forms
\end{itemize}

From a theoretical perspective, nucleonics does seem practical. In 2024,
\href{https://arxiv.org/pdf/2501.08338}{Hagelstein et.al} showed for the
first time that known quantum physics principles could be applied to
satisfy the above criteria and moreover explain long standing
experimental nuclear anomalies found in LENR experiments.

Hagelstein's theoretical framework relies on combining several pieces of
well established quantum physics in a novel way. The central question is
whether Mother Nature respects this novel combination. The framework
needs to be stress tested in several carefully designed experiments in
order to pass the ultimate litmus test.

\subsection{Accelerating nuclear transitions by many orders of
magnitude}\label{accelerating-nuclear-transitions-by-many-orders-of-magnitude}

One of the key bits of physics responsible for accelerating nuclear
transitions is Dicke Superradiance. First proposed by
\href{https://doi.org/10.1103/PhysRev.93.99}{Dicke in the 1950s} for
atoms and then extended by
\href{https://doi.org/10.1103/physrevlett.14.589}{Terhune and Baldwin in
the 1960s} for nuclei, Dicke's model showed that \(N\) particles can
radiate/decay collectively up to \(N^2\) faster than an individual
particle. In a solid lattice, the number density is
\(\sim 10^{28} \ \rm m^{-3}\) and so Dicke enhancements can in principle
be extremely large.

The challenge for nuclear superradaince is how close the nuclei need to
be for collective emission to occur. Dicke's model is based on all \(N\)
nuclei being coupled together through an interaction with a common
oscillating field. In order to get the full \(N^2\) enhancement, all the
nuclei need to be located within a single wavelength of one other. The
wavelength is determined by the frequency (and hence energy) of the
field oscillations. Typically the frequency is chosen to be matched to
the nuclear transitions which mostly have energies of
\(E > 10 \ \rm keV\) and wavelengths of \(\lambda < 0.1 \ \rm nm\). In a
typical solid lattice you can fit at most one particle in cube whose
sides are given by such a small wavelength.
\href{https://www.nature.com/articles/s41567-017-0001-z}{Chumakov et.al
2017} achieved only small levels of nuclear superradiance
(\(\sim 10 \times\) rate enhancements) in \(\rm ^{57}Fe\) experiments
because of this.

Much larger Dicke enhancement rates are required for practical
applications. Bridging the rate enhancement gap can in principle be
achieved by choosing a low frequency (and hence long wavelength) that's
mismatched with respect to the nuclear transitions. In such a set-up, a
nucleus radiates a number of smaller energy oscillator quanta instead of
a single large one. The challenge with this approach is that the more
quanta that are involved, the slower the process becomes - this can
dominate over Dicke enhancements depending on the specific details.

There is a novelty in demonstrating Dicke enhancements with oscillator
and nuclei that are mismatched. It would therefore be preferable to
separate that novelty from the specific demonstration of extremely large
nuclear Dicke enhancements. One way to do this is to repeat the Chumakov
experiment but instead of using \(\rm ^{57}Fe\) we use
\href{https://en.wikipedia.org/wiki/Isotopes_of_thorium\#Thorium-229m}{\(\rm ^{229m}Th\)}.

\(\rm ^{229}Th\) has a nuclear isomer \(\rm ^{229m}Th\) whose existence
was hypothesised for many years but
\href{https://link.springer.com/article/10.1140/epjs/s11734-024-01098-2}{was
only nailed down in 2024}. \(\rm ^{229m}Th\) is unique in that it has an
extremely low energy nuclear transition energy of \(8.36 \ \rm eV\)
which corresponds to a wavelength of \(148 \ \rm nm\). A cube of this
size would contain about \(3\times 10^7\) nuclei at typical solid
density - more than enough to observe some very large Dicke
enhancements.

The specifics of \(\rm ^{229m}Th\) require us to arrange the system to:

\begin{itemize}
\tightlist
\item
  Maximise radiative decay over internal conversion. This can be
  achieved by excite half of the \(\rm ^{229m}Th\) nuclei instead of all
  of them will give us \(N^2\) enhancement of radiation and only \(N/2\)
  enhancement of internal conversion.
\item
  Minimise the reflection of the UV laser light from the surface. This
  can be achived by embedding the \(\rm ^{229m}Th\) into an UV
  transparent material.
\end{itemize}

\subsection{Transferring nuclear energy non-radiatively between
nuclei}\label{transferring-nuclear-energy-non-radiatively-between-nuclei}

Excitation transfer (also known as resonance energy transfer) is the key
bit of physics responsible for moving energy from a donor system to a
receiver system without the emission/absorption of radiation.
\href{https://doi.org/10.1088/1367-2630/12/7/075020}{Supertransfer} is
the Dicke acceleration of this process due to the coupling of many
donors/receivers to a shared oscillator.

There are numerous examples of excitation transfer dynamics at the
atomic scale such as exciton diffusion at room temperature e.g.,
\href{https://www.annualreviews.org/content/journals/10.1146/annurev-physchem-040214-121713}{photosynthetic
systems} and
\href{https://www.nature.com/articles/s41467-022-30308-5}{organic
semiconductors}. Supertransfer in engineered systems has also been
demonstrated by Park and colleagues in a
\href{https://www.nature.com/articles/nmat4448}{2015 Nature Materials
paper} at the atomic scale at room temperature.

Nuclear supertransfer has not yet been demonstrated.

We seek to adapt the
\href{https://www.nature.com/articles/s41567-017-0001-z}{Chumakov et.al
2017} experiment to make it work for supertransfer instead of a
superradiance. More specifically, we propose exciting \(\rm ^{57}Fe\)
nuclei and transferring the \(14 \ \rm keV\) of nuclear energy to
\(\rm Pb\) atoms. The transfer needs to be achieved in a very short
timescale in order to beat the natural radiative decay of the excited
\(\rm ^{57}Fe\). In other words, we need supertransfer.

In order to couple the \(\rm ^{57}Fe\) and \(\rm Pb\) together
non-radiatively, we need a mismatched oscillator and nuclei.
Supertransfer does not suffer the same challenges as superradiance when
using mismatched oscillator and nuclei. This is because superradiance
involves the radiation of many small quanta, whereas supertransfer
involves none - it is radiationless. We propose using a low frequency
(long wavelength) oscillating magnetic field provided by a solenoid to
involve a macroscopically large number of nuclei that will produce the
Dicke enhancements that we need.

The advantage of using a magnetic field is that it's a well established
coupling mechanism that's suitable for demonstration purposes. The
disadvantages are:

\begin{itemize}
\tightlist
\item
  Magnetic dipole coupling is weak and so limits how practical this form
  of coupling can be
\item
  Excitation transfer rates with this coupling can be insensitive to
  frequency (depending on the details) so it does not give us a reliable
  way to control the transition rates.
\end{itemize}

\subsection{Controlling nuclear transitions with low frequency
stimulation}\label{controlling-nuclear-transitions-with-low-frequency-stimulation}

For energy applications, it is necessary to have a type of nuclear
coupling that's both strong and tuneable. For practical purposes such
tunability more easily achieved via low frequency stimulation because it
typically requires smaller and less expensive equipment.

Peter Hagelstein \href{https://arxiv.org/abs/1201.4377}{first proposed}
a novel form of relativistic phonon nuclear coupling in 2012. It relies
on a previously neglected aspect of relativity that couples vibrational
energy (aka phonons) to nuclear energy. Relativistic phonon nuclear
coupling is many orders of magnitude stronger than the electromagnetic
coupling that's typically considered for nuclear interactions. It's also
frequency dependent and so represents a controllable form nuclear
coupling.

Although the most developed form of relativistic phonon nuclear coupling
was
\href{https://iopscience.iop.org/article/10.1088/1361-6455/acf3be}{peer
reviewed in 2023}, it has yet to be experimentally verified.

We propose a nuclear excitation transfer experiment in which we seek to
observe the spreading out of emission of excited \(\rm ^{57}Fe\) as the
nuclear energy is transferred from the excited \(\rm ^{57}Fe\) to
neighbouring ground state \(\rm ^{57}Fe\). The excited nuclei are to
generated from the decay of \(\rm ^{57}Co\) that's deposited on the
\(\rm ^{57}Fe\) surface and the phonon coupling is generated via
stimulation of the surface with a \(\rm THz\) laser.

\subsection{Extracting nuclear energy in benign
forms}\label{extracting-nuclear-energy-in-benign-forms}

Relativistic phonon nuclear coupling is so strong that experiments will
be in a form of deep strong coupling regime that was first proposed as
an academic curiosity in 2010 by
\href{https://journals.aps.org/prl/abstract/10.1103/PhysRevLett.105.263603}{Casanova
at.al} and only recently experimentally observed in ``artificial atoms''
in \href{https://pubs.acs.org/doi/10.1021/acs.nanolett.7b03103}{2017 by
Bayer at.al}. The behaviour in this regime is very different from the
usual weak coupling regime that informs peoples' intuitions via
``perturbation theory''.

Deep strong coupling allows a free exchange of energy between the
particles and the field. For the phonon-nuclear system this represents a
free exchange of nuclear and vibrational energy. This presents an
opportunity to ``bleed off'' nuclear energy into vibrational energy and
ultimately into heat. It also presents the reverse opportunity to turn
vibrational energy into nuclear energy, e.g.~creating a controllable
x-ray laser.

Unlike artificial atoms, deep strong coupling for real atoms and nuclei
requires a minimum field energy. Electromagnetic coupling is too small
to make deep strong coupling possible for real atoms and nuclear.
However, relativistic phonon nuclear coupling is so strong that only a
very modest field energy of \(\sim \rm 10 \ mJ\) is needed.

We propose to test the free exchange of nuclear and vibrational energy
by creating \(>10 \ \rm mJ\) of phonon energy inside an otherwise stable
material using a \(\rm GHz\) piezoelectric driver and then detecting the
gamma ray emission. Heavier elements give the largest coupling and
header materials produce longer lasting phonons and so tungsten is an
ideal choice.

\printbibliography


\end{document}
