% Options for packages loaded elsewhere
\PassOptionsToPackage{unicode}{hyperref}
\PassOptionsToPackage{hyphens}{url}
\documentclass[
]{article}
\usepackage{xcolor}
\usepackage{amsmath,amssymb}

\usepackage{hyperref}
\usepackage{ulem}  % For underlining links
\normalem

\hypersetup{
    colorlinks=false,         % Disable colored links (boxes will appear)
    citebordercolor=red,      % Border color for citation links
    linkbordercolor=red,      % Border color for internal links
    urlbordercolor=blue,      % Border color for external links (URLs)
    pdfborder={0 0 2}         % Specifies the border thickness: {horizontal vertical thickness}
}


% Only underline external links (URLs)
\let\oldhref\href
\renewcommand{\href}[2]{\ifx#1\urlprefix\oldhref{#1}{#2}\else\uline{\oldhref{#1}{#2}}\fi}


\renewcommand{\[}{\begin{equation}}
\renewcommand{\]}{\end{equation}}


\setcounter{secnumdepth}{5}
\usepackage{iftex}
\ifPDFTeX
  \usepackage[T1]{fontenc}
  \usepackage[utf8]{inputenc}
  \usepackage{textcomp} % provide euro and other symbols
\else % if luatex or xetex
  \usepackage{unicode-math} % this also loads fontspec
  \defaultfontfeatures{Scale=MatchLowercase}
  \defaultfontfeatures[\rmfamily]{Ligatures=TeX,Scale=1}
\fi
\usepackage{lmodern}
\ifPDFTeX\else
  % xetex/luatex font selection
\fi
% Use upquote if available, for straight quotes in verbatim environments
\IfFileExists{upquote.sty}{\usepackage{upquote}}{}
\IfFileExists{microtype.sty}{% use microtype if available
  \usepackage[]{microtype}
  \UseMicrotypeSet[protrusion]{basicmath} % disable protrusion for tt fonts
}{}
\makeatletter
\@ifundefined{KOMAClassName}{% if non-KOMA class
  \IfFileExists{parskip.sty}{%
    \usepackage{parskip}
  }{% else
    \setlength{\parindent}{0pt}
    \setlength{\parskip}{6pt plus 2pt minus 1pt}}
}{% if KOMA class
  \KOMAoptions{parskip=half}}
\makeatother
\setlength{\emergencystretch}{3em} % prevent overfull lines
\providecommand{\tightlist}{%
  \setlength{\itemsep}{0pt}\setlength{\parskip}{0pt}}
\usepackage[]{biblatex}
\addbibresource{refs.bib}
\usepackage{bookmark}
\IfFileExists{xurl.sty}{\usepackage{xurl}}{} % add URL line breaks if available
\urlstyle{same}

\title{Phase in electromagnetic coupling}
  \date{\today}  % Default to today if no date is provided

\begin{document}
\maketitle

\section{Phase in Electomagnetic
coupling}\label{phase-in-electomagnetic-coupling}

These notes build up a clean ``mental model'' for how phases arise in
light--matter coupling when two \textbf{identical} two-level systems
(TLS) interact with a \textbf{single quantised electromagnetic mode},
including both \textbf{electric} and \textbf{magnetic} dipole
contributions. The focus is on:

\begin{itemize}
\tightlist
\item
  how we write \(H_{\mathrm{int}}\) from general \(E\) and \(B\)
  couplings,
\item
  how this reduces to coupling to a \textbf{single oscillator
  quadrature},
\item
  where the \textbf{phase factors} come from,
\item
  how phases can be \textbf{gauged away} (factored out),
\item
  and when a \textbf{relative phase difference} between donor/acceptor
  is physically meaningful.
\end{itemize}

\begin{center}\rule{0.5\linewidth}{0.5pt}\end{center}

\subsection{1. The setup: two identical TLS and a single
mode}\label{the-setup-two-identical-tls-and-a-single-mode}

We have two identical TLS labelled \(j \in \{D, A\}\) (donor/acceptor),
transition frequency \(\omega_0\), and a single bosonic mode
(e.g.~cavity/waveguide mode) with frequency \(\omega_c\):

\[
H_0
=
\hbar \omega_c a^\dagger a
+
\sum_{j=D,A}\frac{\hbar \omega_0}{2}\,\sigma_z^{(j)}.
\label{eq:H0}
\]

We will mostly care about the \textbf{interaction} Hamiltonian.

\begin{center}\rule{0.5\linewidth}{0.5pt}\end{center}

\subsection{\texorpdfstring{2. General dipole interaction with \(E\) and
\(B\)}{2. General dipole interaction with E and B}}\label{general-dipole-interaction-with-e-and-b}

In the dipole approximation (TLS size \(\ll \lambda\)), the interaction
for TLS \(j\) is:

\[
H_{\mathrm{int}}^{(j)}
=
-\hat{\mathbf d}^{(j)}\cdot \hat{\mathbf E}(\mathbf r_j)
-\hat{\boldsymbol\mu}^{(j)}\cdot \hat{\mathbf B}(\mathbf r_j),
\label{eq:Hint_general}
\]

where:

\begin{itemize}
\tightlist
\item
  \(\hat{\mathbf d}^{(j)}\) is the electric dipole operator of TLS
  \(j\),
\item
  \(\hat{\boldsymbol\mu}^{(j)}\) is the magnetic dipole operator,
\item
  \(\mathbf r_j\) is the TLS position.
\end{itemize}

Total interaction is:

\[
H_{\mathrm{int}}=\sum_{j=D,A} H_{\mathrm{int}}^{(j)}.
\label{eq:Hint_sum}
\]

\subsubsection{2.1 Identical TLS
operators}\label{identical-tls-operators}

For identical TLS, the transition operators have the same matrix
elements, but geometry/projections can differ with position/orientation.

A common simplification is to take:

\[
\hat{\mathbf d}^{(j)} = \mathbf d\,(\sigma_+^{(j)}+\sigma_-^{(j)}) = \mathbf d\,\sigma_x^{(j)},
\qquad
\hat{\boldsymbol\mu}^{(j)} = \boldsymbol\mu\,(\sigma_+^{(j)}+\sigma_-^{(j)}) = \boldsymbol\mu\,\sigma_x^{(j)}.
\label{eq:dipole_ops}
\]

More generally, \(\mathbf d\) and \(\boldsymbol\mu\) can be complex
vectors, but the structural results below remain the same.

\begin{center}\rule{0.5\linewidth}{0.5pt}\end{center}

\subsection{3. One quantised mode: field operators and complex mode
functions}\label{one-quantised-mode-field-operators-and-complex-mode-functions}

For a \textbf{single mode}, the electric and magnetic field operators at
position \(\mathbf r\) can always be written as:

\[
\hat{\mathbf E}(\mathbf r)
=
\mathbf E_{\mathrm{zpf}}(\mathbf r)\,a + \mathbf E_{\mathrm{zpf}}^*(\mathbf r)\,a^\dagger,
\qquad
\hat{\mathbf B}(\mathbf r)
=
\mathbf B_{\mathrm{zpf}}(\mathbf r)\,a + \mathbf B_{\mathrm{zpf}}^*(\mathbf r)\,a^\dagger.
\label{eq:fields_single_mode}
\]

Here \(\mathbf E_{\mathrm{zpf}}(\mathbf r)\) and
\(\mathbf B_{\mathrm{zpf}}(\mathbf r)\) are \textbf{complex
vector-valued mode amplitudes} (including polarisation and spatial
dependence), and ``zpf'' means ``zero-point field''.

A travelling-wave-like mode often has:

\[
\mathbf E_{\mathrm{zpf}}(\mathbf r)\propto e^{i\mathbf k\cdot\mathbf r},\qquad
\mathbf B_{\mathrm{zpf}}(\mathbf r)\propto e^{i\mathbf k\cdot\mathbf r}.
\label{eq:travelling_mode_phase}
\]

A lossless standing-wave cavity mode can often be chosen real up to sign
(so phases are \(0\) or \(\pi\)).

\begin{center}\rule{0.5\linewidth}{0.5pt}\end{center}

\subsection{\texorpdfstring{4. The simplest case: \textbf{magnetic
dipole only}
(B-only)}{4. The simplest case: magnetic dipole only (B-only)}}\label{the-simplest-case-magnetic-dipole-only-b-only}

Assume \(\mathbf d = 0\) and only magnetic coupling:

\[
H_{\mathrm{int}}^{(j)}
=
-\hat{\boldsymbol\mu}^{(j)}\cdot \hat{\mathbf B}(\mathbf r_j).
\label{eq:Hint_Bonly_start}
\]

Insert Eqs. \(\ref{eq:dipole_ops}\) and \(\ref{eq:fields_single_mode}\):

\[
H_{\mathrm{int}}^{(j)}
=
-\sigma_x^{(j)}\Big[
\boldsymbol\mu\cdot \mathbf B_{\mathrm{zpf}}(\mathbf r_j)\,a
+
\boldsymbol\mu\cdot \mathbf B_{\mathrm{zpf}}^*(\mathbf r_j)\,a^\dagger
\Big].
\label{eq:Hint_Bonly_expand}
\]

Define the complex scalar coupling coefficient:

\[
g_j \equiv -\frac{1}{\hbar}\,\boldsymbol\mu\cdot \mathbf B_{\mathrm{zpf}}(\mathbf r_j),
\qquad
g_j^* = -\frac{1}{\hbar}\,\boldsymbol\mu\cdot \mathbf B_{\mathrm{zpf}}^*(\mathbf r_j).
\label{eq:gj_def}
\]

Then:

\[
H_{\mathrm{int}}^{(j)}
=
\hbar\,\sigma_x^{(j)}\big(g_j a + g_j^* a^\dagger\big).
\label{eq:Hint_Bonly_g}
\]

\subsubsection{4.1 Writing the phase
explicitly}\label{writing-the-phase-explicitly}

Write \(g_j\) in polar form:

\[
g_j = |g_j|e^{i\theta_j}.
\label{eq:gj_polar}
\]

Then:

\[
g_j a + g_j^* a^\dagger
=
|g_j|\big(e^{i\theta_j}a + e^{-i\theta_j}a^\dagger\big),
\label{eq:phase_factor_form}
\]

so:

\[
H_{\mathrm{int}}^{(j)}
=
\hbar |g_j|\,\sigma_x^{(j)}\big(e^{i\theta_j}a + e^{-i\theta_j}a^\dagger\big).
\label{eq:Hint_Bonly_phase}
\]

\subsubsection{\texorpdfstring{4.2 Where does \(\theta_j\) come
from?}{4.2 Where does \textbackslash theta\_j come from?}}\label{where-does-theta_j-come-from}

It is simply:

\[
\theta_j = \arg\!\big(\boldsymbol\mu\cdot \mathbf B_{\mathrm{zpf}}(\mathbf r_j)\big).
\label{eq:theta_origin}
\]

For a travelling wave with
\(\mathbf B_{\mathrm{zpf}}(\mathbf r)\propto e^{i\mathbf k\cdot\mathbf r}\)
and fixed polarisation,

\[
\theta_j \simeq \mathbf k\cdot \mathbf r_j + \text{(constant)}.
\label{eq:theta_travelling}
\]

\begin{center}\rule{0.5\linewidth}{0.5pt}\end{center}

\subsection{5. Factoring out a global phase (why only relative phase
matters)}\label{factoring-out-a-global-phase-why-only-relative-phase-matters}

The free Hamiltonian \(\hbar\omega_c a^\dagger a\) is invariant under:

\[
a \to a\,e^{-i\chi},
\qquad
a^\dagger \to a^\dagger e^{i\chi}.
\label{eq:mode_rephase}
\]

This is a ``phase convention'' for the mode. Use it to make one
emitter's phase vanish.

Take \(\chi=\theta_D\). Then donor's factor becomes:

\[
e^{i\theta_D}a + e^{-i\theta_D}a^\dagger
\to
a + a^\dagger.
\label{eq:donor_phase_removed}
\]

Acceptor keeps only the \textbf{relative phase}:

\[
\Delta\phi \equiv \theta_A-\theta_D.
\label{eq:delta_phi_def}
\]

So the two-emitter B-only interaction can be written as:

\[
H_{\mathrm{int}}
=
\hbar |g_D|\sigma_x^{(D)}(a+a^\dagger)
+
\hbar |g_A|\sigma_x^{(A)}\big(e^{i\Delta\phi}a+e^{-i\Delta\phi}a^\dagger\big).
\label{eq:Hint_Bonly_final}
\]

\begin{center}\rule{0.5\linewidth}{0.5pt}\end{center}

\subsection{\texorpdfstring{6. When do you get a \textbf{nontrivial}
relative phase
\(\Delta\phi\)?}{6. When do you get a nontrivial relative phase \textbackslash Delta\textbackslash phi?}}\label{when-do-you-get-a-nontrivial-relative-phase-deltaphi}

\subsubsection{6.1 Travelling-wave / running-wave
mode}\label{travelling-wave-running-wave-mode}

If
\(\mathbf B_{\mathrm{zpf}}(\mathbf r)\propto e^{i\mathbf k\cdot\mathbf r}\)
then:

\[
\Delta\phi = \mathbf k\cdot(\mathbf r_A-\mathbf r_D).
\label{eq:delta_phi_kdr}
\]

In 1D, \(\Delta\phi = k(x_A-x_D)\), and \(\Delta\phi=\pi/2\) corresponds
to separation \(\Delta x=\lambda/4\).

\subsubsection{6.2 Ideal standing-wave cavity
mode}\label{ideal-standing-wave-cavity-mode}

If the mode function can be chosen real everywhere, then \(g_j\) is real
up to a sign, so:

\begin{itemize}
\tightlist
\item
  phases are typically \(0\) or \(\pi\),
\item
  you cannot get a continuous \(\Delta\phi\) unless you use a
  running-wave basis or the mode is intrinsically complex (lossy/open
  cavity, quasi-normal mode, etc.).
\end{itemize}

\subsubsection{6.3 Extra ways to get complex spatial
phase}\label{extra-ways-to-get-complex-spatial-phase}

Even for ``one mode'', \(\mathbf B_{\mathrm{zpf}}(\mathbf r)\) can be
intrinsically complex in:

\begin{itemize}
\tightlist
\item
  leaky/open cavities (quasi-normal modes),
\item
  waveguides/rings (naturally travelling-wave eigenmodes),
\item
  strongly confined/evanescent structures where phase varies over
  shorter effective wavelengths.
\end{itemize}

\begin{center}\rule{0.5\linewidth}{0.5pt}\end{center}

\subsection{7. Quadratures: what they are and why they
appear}\label{quadratures-what-they-are-and-why-they-appear}

For one harmonic oscillator mode, define quadratures:

\[
X \equiv a+a^\dagger,
\qquad
P \equiv i(a^\dagger-a).
\label{eq:XP_def}
\]

A useful identity is:

\[
e^{i\phi}a + e^{-i\phi}a^\dagger
=
X\cos\phi + P\sin\phi.
\label{eq:quadrature_identity}
\]

So the ``phase factor'' form is exactly ``a rotated quadrature'':

\[
X_\phi \equiv X\cos\phi + P\sin\phi.
\label{eq:rotated_quadrature}
\]

\subsubsection{7.1 Important conceptual
separation}\label{important-conceptual-separation}

\begin{itemize}
\tightlist
\item
  In the travelling-wave B-only case, \(\Delta\phi\) comes from the
  \textbf{complex mode function} \(e^{ikx}\).
\item
  Rewriting it as quadratures via Eq. \(\ref{eq:quadrature_identity}\)
  is \textbf{pure algebra}, not new physics.
\item
  It does \textbf{not} mean the acceptor is coupling to \(E\) instead of
  \(B\).
\end{itemize}

\begin{center}\rule{0.5\linewidth}{0.5pt}\end{center}

\subsection{\texorpdfstring{8. Coupling to both \(E\) and \(B\) at a
single point: how a quadrature angle arises \emph{without spatial
separation}}{8. Coupling to both E and B at a single point: how a quadrature angle arises without spatial separation}}\label{coupling-to-both-e-and-b-at-a-single-point-how-a-quadrature-angle-arises-without-spatial-separation}

Now allow both dipoles:

\[
H_{\mathrm{int}}^{(j)}=
-\hat{\mathbf d}^{(j)}\cdot \hat{\mathbf E}(\mathbf r_j)
-\hat{\boldsymbol\mu}^{(j)}\cdot \hat{\mathbf B}(\mathbf r_j).
\label{eq:Hint_EB_start}
\]

At a single point, in many common quantisation conventions (notably
those built from \(\mathbf A\)), one finds that for a single mode:

\begin{itemize}
\tightlist
\item
  \(\hat{\mathbf B}(\mathbf r)\) is proportional to one quadrature
  (often \(X\)),
\item
  \(\hat{\mathbf E}(\mathbf r)\) is proportional to the orthogonal
  quadrature (often \(P\)).
\end{itemize}

A toy model is:

\[
\hat{\mathbf B}(\mathbf r_j)=\mathbf B_{\mathrm{zpf}}(\mathbf r_j)\,X,
\qquad
\hat{\mathbf E}(\mathbf r_j)=\mathbf E_{\mathrm{zpf}}(\mathbf r_j)\,P.
\label{eq:EB_as_quadratures}
\]

Project onto dipoles and define real scalar couplings:

\[
g_{B,j}\equiv \boldsymbol\mu\cdot \mathbf B_{\mathrm{zpf}}(\mathbf r_j),\qquad
g_{E,j}\equiv \mathbf d\cdot \mathbf E_{\mathrm{zpf}}(\mathbf r_j).
\label{eq:gE_gB_def}
\]

Then for TLS \(j\):

\[
H_{\mathrm{int}}^{(j)} = -\sigma_x^{(j)}\big(g_{B,j}X + g_{E,j}P\big).
\label{eq:Hint_EB_XP}
\]

Now combine into one rotated quadrature using:

\[
V_j = \sqrt{g_{B,j}^2+g_{E,j}^2},\qquad
\phi_j=\arctan\!\left(\frac{g_{E,j}}{g_{B,j}}\right),
\label{eq:V_phi_def}
\]

giving:

\[
g_{B,j}X + g_{E,j}P
=
V_j\big(X\cos\phi_j + P\sin\phi_j\big)
=
V_j\big(e^{i\phi_j}a + e^{-i\phi_j}a^\dagger\big).
\label{eq:EB_to_phase}
\]

So:

\[
H_{\mathrm{int}}^{(j)}
=
- V_j\,\sigma_x^{(j)}\big(e^{i\phi_j}a + e^{-i\phi_j}a^\dagger\big).
\label{eq:Hint_EB_phase}
\]

\subsubsection{\texorpdfstring{8.1 What is the meaning of \(\phi_j\)
here?}{8.1 What is the meaning of \textbackslash phi\_j here?}}\label{what-is-the-meaning-of-phi_j-here}

It is the \textbf{quadrature angle} selected by the ratio of electric to
magnetic coupling at that point:

\begin{itemize}
\tightlist
\item
  \(g_{E,j}=0 \Rightarrow \phi_j=0\) (pure \(X\) coupling),
\item
  \(g_{B,j}=0 \Rightarrow \phi_j=\pi/2\) (pure \(P\) coupling),
\item
  both nonzero \(\Rightarrow\) rotated quadrature.
\end{itemize}

\subsubsection{\texorpdfstring{8.2 Does this produce a \emph{relative}
phase between two identical
TLS?}{8.2 Does this produce a relative phase between two identical TLS?}}\label{does-this-produce-a-relative-phase-between-two-identical-tls}

If the TLS are identical and sit in identical conditions, they have the
same \(g_{E}/g_{B}\), hence the same \(\phi_j\). In that case, a global
rephasing can remove the common phase from both simultaneously.

A relative phase arises only if:

\[
\phi_A \neq \phi_D,
\label{eq:relative_phase_condition}
\]

which requires a difference in:

\begin{itemize}
\tightlist
\item
  position-dependent field ratios
  \(E_{\mathrm{zpf}}(\mathbf r)/B_{\mathrm{zpf}}(\mathbf r)\),
\item
  orientation/projection of dipoles on local polarisation,
\item
  or (more exotic) intrinsic complex relative phase between electric and
  magnetic transition moments.
\end{itemize}

\begin{center}\rule{0.5\linewidth}{0.5pt}\end{center}

\subsection{9. Combining both mechanisms (general
picture)}\label{combining-both-mechanisms-general-picture}

In the most general single-mode case, each emitter's interaction can be
written as:

\[
H_{\mathrm{int}}^{(j)}
=
\hbar\,\sigma_x^{(j)}\Big(\alpha_j a + \alpha_j^* a^\dagger\Big),
\label{eq:alpha_general}
\]

where the complex coefficient \(\alpha_j\) contains \emph{everything}:

\begin{itemize}
\tightlist
\item
  electric contribution via \(\mathbf E_{\mathrm{zpf}}(\mathbf r_j)\)
  and \(\mathbf d\),
\item
  magnetic contribution via \(\mathbf B_{\mathrm{zpf}}(\mathbf r_j)\)
  and \(\boldsymbol\mu\),
\item
  spatial dependence (e.g.~\(e^{i\mathbf k\cdot\mathbf r_j}\)),
\item
  polarisation/projection factors.
\end{itemize}

Write \(\alpha_j = |\alpha_j|e^{i\theta_j}\) and you get:

\[
H_{\mathrm{int}}^{(j)}
=
\hbar|\alpha_j|\sigma_x^{(j)}\big(e^{i\theta_j}a + e^{-i\theta_j}a^\dagger\big).
\label{eq:alpha_phase}
\]

Then only the \textbf{relative phase} \(\Delta\phi=\theta_A-\theta_D\)
survives after choosing the mode phase convention.

\begin{center}\rule{0.5\linewidth}{0.5pt}\end{center}

\subsection{10. What is physically observable about these
phases?}\label{what-is-physically-observable-about-these-phases}

\subsubsection{10.1 Global phase is not observable in the single-mode
Hamiltonian}\label{global-phase-is-not-observable-in-the-single-mode-hamiltonian}

You can always choose the phase of \(a\) to remove one emitter's phase
entirely. That's why the literature often writes the donor with
\((a+a^\dagger)\) and puts the phase on the acceptor.

\subsubsection{10.2 Relative phase becomes meaningful when there is
interference}\label{relative-phase-becomes-meaningful-when-there-is-interference}

A relative phase matters when the physics compares ``paths'' or
``collective combinations'', e.g.:

\begin{itemize}
\tightlist
\item
  superradiant vs subradiant (bright/dark) collective states in
  two-emitter coupling,
\item
  direct donor--acceptor coupling terms,
\item
  multiple modes / continuum (propagation and retardation),
\item
  coherent drives with a fixed phase reference.
\end{itemize}

Even in a single-mode model, the relative phase influences which
collective TLS superposition couples most strongly to the field.

\begin{center}\rule{0.5\linewidth}{0.5pt}\end{center}

\subsection{11. Summary ``rules of
thumb''}\label{summary-rules-of-thumb}

\begin{itemize}
\tightlist
\item
  \textbf{Single travelling wave}: \(E\) and \(B\) are in phase in time
  for that wave.
\item
  \textbf{Standing wave}: locally, \(E\) and \(B\) are often \(\pi/2\)
  out of phase in time (energy sloshing).
\item
  \textbf{B-only coupling}: phases arise from the complex spatial mode
  function \(u(\mathbf r)\) (e.g.~\(e^{ikx}\)).
\item
  \textbf{E + B coupling}: even at one point, the TLS couples to a
  linear combination of quadratures; this is a ``quadrature rotation''
  with angle \(\phi=\arctan(g_E/g_B)\).
\item
  \textbf{Global phase} can be removed by redefining \(a\); only
  \textbf{relative phase} between emitters or channels matters.
\item
  \textbf{Relative phase difference} requires emitters to experience
  different complex coefficients: either different spatial phase
  (travelling wave) or different \(g_E/g_B\) ratio (varying
  geometry/orientation/field ratios).
\end{itemize}

\begin{center}\rule{0.5\linewidth}{0.5pt}\end{center}

\subsection{12. Minimal ``canonical forms'' to
remember}\label{minimal-canonical-forms-to-remember}

\subsubsection{12.1 B-only (travelling
wave)}\label{b-only-travelling-wave}

\[
H_{\mathrm{int}}
=
\hbar\sum_{j=D,A}\sigma_x^{(j)}\big(g_0 e^{i\mathbf k\cdot\mathbf r_j}a + g_0 e^{-i\mathbf k\cdot\mathbf r_j}a^\dagger\big).
\label{eq:canonical_Bonly}
\]

After gauging away the donor phase:

\[
H_{\mathrm{int}}
=
\hbar g_0\sigma_x^{(D)}(a+a^\dagger)
+
\hbar g_0\sigma_x^{(A)}\big(e^{i\Delta\phi}a+e^{-i\Delta\phi}a^\dagger\big),
\quad
\Delta\phi=\mathbf k\cdot(\mathbf r_A-\mathbf r_D).
\label{eq:canonical_Bonly_relative}
\]

\subsubsection{12.2 E + B at one point (quadrature
rotation)}\label{e-b-at-one-point-quadrature-rotation}

\[
H_{\mathrm{int}}^{(j)} = -\sigma_x^{(j)}\big(g_{B,j}X + g_{E,j}P\big)
= -V_j\sigma_x^{(j)}\big(e^{i\phi_j}a+e^{-i\phi_j}a^\dagger\big),
\quad
\phi_j=\arctan\!\left(\frac{g_{E,j}}{g_{B,j}}\right).
\label{eq:canonical_EB_rotation}
\]

\begin{center}\rule{0.5\linewidth}{0.5pt}\end{center}

\printbibliography


\end{document}
