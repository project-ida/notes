% Options for packages loaded elsewhere
\PassOptionsToPackage{unicode}{hyperref}
\PassOptionsToPackage{hyphens}{url}
\documentclass[
]{article}
\usepackage{xcolor}
\usepackage{amsmath,amssymb}

\usepackage{hyperref}
\usepackage{ulem}  % For underlining links
\normalem

\hypersetup{
    colorlinks=false,         % Disable colored links (boxes will appear)
    citebordercolor=red,      % Border color for citation links
    linkbordercolor=red,      % Border color for internal links
    urlbordercolor=blue,      % Border color for external links (URLs)
    pdfborder={0 0 2}         % Specifies the border thickness: {horizontal vertical thickness}
}


% Only underline external links (URLs)
\let\oldhref\href
\renewcommand{\href}[2]{\ifx#1\urlprefix\oldhref{#1}{#2}\else\uline{\oldhref{#1}{#2}}\fi}


\renewcommand{\[}{\begin{equation}}
\renewcommand{\]}{\end{equation}}


\setcounter{secnumdepth}{5}
\usepackage{iftex}
\ifPDFTeX
  \usepackage[T1]{fontenc}
  \usepackage[utf8]{inputenc}
  \usepackage{textcomp} % provide euro and other symbols
\else % if luatex or xetex
  \usepackage{unicode-math} % this also loads fontspec
  \defaultfontfeatures{Scale=MatchLowercase}
  \defaultfontfeatures[\rmfamily]{Ligatures=TeX,Scale=1}
\fi
\usepackage{lmodern}
\ifPDFTeX\else
  % xetex/luatex font selection
\fi
% Use upquote if available, for straight quotes in verbatim environments
\IfFileExists{upquote.sty}{\usepackage{upquote}}{}
\IfFileExists{microtype.sty}{% use microtype if available
  \usepackage[]{microtype}
  \UseMicrotypeSet[protrusion]{basicmath} % disable protrusion for tt fonts
}{}
\makeatletter
\@ifundefined{KOMAClassName}{% if non-KOMA class
  \IfFileExists{parskip.sty}{%
    \usepackage{parskip}
  }{% else
    \setlength{\parindent}{0pt}
    \setlength{\parskip}{6pt plus 2pt minus 1pt}}
}{% if KOMA class
  \KOMAoptions{parskip=half}}
\makeatother
\setlength{\emergencystretch}{3em} % prevent overfull lines
\providecommand{\tightlist}{%
  \setlength{\itemsep}{0pt}\setlength{\parskip}{0pt}}
\usepackage[]{biblatex}
\addbibresource{refs.bib}
\usepackage{bookmark}
\IfFileExists{xurl.sty}{\usepackage{xurl}}{} % add URL line breaks if available
\urlstyle{same}

\title{Phase in electromagnetic coupling}
  \date{\today}  % Default to today if no date is provided

\begin{document}
\maketitle

\section{Phase in electomagnetic
coupling}\label{phase-in-electomagnetic-coupling}

These notes build a clean toy-model Hamiltonian for two identical
two-level systems (TLS) coupled to a single quantised electromagnetic
(EM) mode, focusing on:

\begin{itemize}
\tightlist
\item
  How to write the interaction for \textbf{electric} and
  \textbf{magnetic} dipole transitions.
\item
  Why and how \textbf{phase factors} like \(e^{\pm i\phi}\) appear in
  cavity/waveguide Hamiltonians.
\item
  The difference between:

  \begin{itemize}
  \tightlist
  \item
    \textbf{spatial phase} (from \(e^{ikx}\) in a travelling wave), and
  \item
    \textbf{quadrature rotation} (coupling to \(X\) vs \(P\) of the
    oscillator).
  \end{itemize}
\item
  A conceptual clarification of ``phase'' in EM waves (travelling vs
  standing waves).
\end{itemize}

Throughout, we use a \emph{single mode} (one harmonic oscillator) and
two TLS labelled donor \(D\) and acceptor \(A\).

\begin{center}\rule{0.5\linewidth}{0.5pt}\end{center}

\subsection{The basic ingredients}\label{the-basic-ingredients}

\subsubsection{TLS Hamiltonian}\label{tls-hamiltonian}

For TLS \(j\in\{D,A\}\) with transition frequency \(\omega_j\),

\[
H_{\mathrm{TLS}}=\sum_{j=D,A}\frac{\hbar\omega_j}{2}\,\sigma_z^{(j)}. \label{eq:HTLS}
\]

\begin{itemize}
\tightlist
\item
  \(\sigma_z^{(j)}\) is the Pauli \(z\) operator acting on TLS \(j\).
\item
  \(\sigma_\pm^{(j)}\) are raising/lowering operators, with
  \(\sigma_x^{(j)}=\sigma_+^{(j)}+\sigma_-^{(j)}\).
\end{itemize}

\subsubsection{Single quantised mode
Hamiltonian}\label{single-quantised-mode-hamiltonian}

One EM mode of frequency \(\omega_c\) is described by bosonic operators
\(a,a^\dagger\):

\[
H_{\mathrm{mode}}=\hbar\omega_c\,a^\dagger a. \label{eq:Hmode}
\]

Total free Hamiltonian:

\[
H_0 = H_{\mathrm{mode}} + H_{\mathrm{TLS}}. \label{eq:H0}
\]

\begin{center}\rule{0.5\linewidth}{0.5pt}\end{center}

\subsection{Dipole interactions: the starting
point}\label{dipole-interactions-the-starting-point}

A TLS transition can couple to:

\begin{itemize}
\tightlist
\item
  the \textbf{electric field} via an electric transition dipole
  \(\hat{\mathbf d}\), and/or
\item
  the \textbf{magnetic field} via a magnetic transition dipole
  \(\hat{\boldsymbol\mu}\).
\end{itemize}

In the dipole approximation, at the TLS position \(\mathbf r_j\):

\[
H_{\mathrm{int}}
=
-\sum_{j=D,A}
\left(
\hat{\mathbf d}^{(j)}\cdot\hat{\mathbf E}(\mathbf r_j)
+
\hat{\boldsymbol\mu}^{(j)}\cdot\hat{\mathbf B}(\mathbf r_j)
\right). \label{eq:Hint_general}
\]

\subsubsection{Transition operators (minimal, aligned
case)}\label{transition-operators-minimal-aligned-case}

To keep the algebra transparent, assume each transition dipole is
aligned with the local field polarisation so dot products reduce to
scalars:

\begin{itemize}
\tightlist
\item
  \(\hat d^{(j)} = d\,\sigma_x^{(j)}\)
\item
  \(\hat\mu^{(j)} = \mu\,\sigma_x^{(j)}\)
\end{itemize}

Then Eq. \(\ref{eq:Hint_general}\) becomes

\[
H_{\mathrm{int}}
=
-\sum_{j=D,A}
\sigma_x^{(j)}
\left(
d\,\hat E(\mathbf r_j)
+
\mu\,\hat B(\mathbf r_j)
\right). \label{eq:Hint_scalar}
\]

(General vector/complex matrix-element versions give the same structure
but with additional projection factors and complex conjugation; the key
ideas below do not rely on the simplification.)

\begin{center}\rule{0.5\linewidth}{0.5pt}\end{center}

\subsection{Quantising the fields for a single
mode}\label{quantising-the-fields-for-a-single-mode}

A single mode at position \(\mathbf r\) can always be written as

\[
\hat{\mathbf E}(\mathbf r) = \mathbf E_{\mathrm{zpf}}(\mathbf r)\,a + \mathbf E_{\mathrm{zpf}}^*(\mathbf r)\,a^\dagger, \qquad
\hat{\mathbf B}(\mathbf r) = \mathbf B_{\mathrm{zpf}}(\mathbf r)\,a + \mathbf B_{\mathrm{zpf}}^*(\mathbf r)\,a^\dagger. \label{eq:fields_singlemode}
\]

\begin{itemize}
\tightlist
\item
  ``zpf'' means zero-point field amplitude (including polarisation and
  spatial mode function).
\item
  The complex nature of \(\mathbf E_{\mathrm{zpf}}(\mathbf r)\) or
  \(\mathbf B_{\mathrm{zpf}}(\mathbf r)\) is exactly where phase factors
  come from.
\end{itemize}

A useful scalar shorthand at each TLS position is:

\[
\hat B(\mathbf r_j)= B_{\mathrm{zpf}}\big(u_j a + u_j^* a^\dagger\big), \qquad
u_j \equiv \text{(complex mode factor at } \mathbf r_j\text{)}. \label{eq:Bmodefactor}
\]

Write \(u_j=|u_j|e^{i\theta_j}\):

\[
u_j a + u_j^* a^\dagger
=
|u_j|\left(e^{i\theta_j}a+e^{-i\theta_j}a^\dagger\right). \label{eq:phaseform}
\]

This algebraic form is the origin of Hamiltonians containing
\((e^{i\phi}a+e^{-i\phi}a^\dagger)\).

\begin{center}\rule{0.5\linewidth}{0.5pt}\end{center}

\subsection{Phase factors from spatial separation in a travelling wave
(B-only)}\label{phase-factors-from-spatial-separation-in-a-travelling-wave-b-only}

\subsubsection{Travelling-wave mode
function}\label{travelling-wave-mode-function}

For a running (travelling) wave along \(x\), a standard mode factor is

\[
u(x)=e^{ikx}. \label{eq:ux}
\]

Then the magnetic field operator is

\[
\hat B(x)=B_{\mathrm{zpf}}\left(e^{ikx}a+e^{-ikx}a^\dagger\right). \label{eq:Btravel}
\]

\subsubsection{\texorpdfstring{Two identical TLS at positions \(x_D\)
and
\(x_A\)}{Two identical TLS at positions x\_D and x\_A}}\label{two-identical-tls-at-positions-x_d-and-x_a}

Insert Eq. \(\ref{eq:Btravel}\) into the magnetic-only part of Eq.
\(\ref{eq:Hint_scalar}\) (set \(d=0\)):

\[
H_{\mathrm{int}}^{(B)}
=
-\mu B_{\mathrm{zpf}}
\sum_{j=D,A}
\sigma_x^{(j)}
\left(e^{ikx_j}a+e^{-ikx_j}a^\dagger\right). \label{eq:Hint_Btravel}
\]

Define the couplings \(V_j\equiv \mu B_{\mathrm{zpf}}\) (or include
\(|u(x_j)|\) if the mode has spatial envelope). Then

\[
H_{\mathrm{int}}^{(B)}
=
-\sum_{j=D,A}
V_j\,\sigma_x^{(j)}
\left(e^{ikx_j}a+e^{-ikx_j}a^\dagger\right). \label{eq:Hint_Btravel2}
\]

\subsubsection{Only the relative phase
matters}\label{only-the-relative-phase-matters}

You can rephase the oscillator without changing \(H_{\mathrm{mode}}\):

\begin{itemize}
\tightlist
\item
  Transform \(a \rightarrow a\,e^{-ikx_D}\) (and
  \(a^\dagger \rightarrow a^\dagger e^{ikx_D}\)).
\end{itemize}

This makes the donor term purely \((a+a^\dagger)\), and the acceptor
term keeps the \textbf{relative} phase

\[
\Delta\phi = k(x_A-x_D). \label{eq:deltaphi}
\]

So you obtain the familiar-looking structure

\[
H_{\mathrm{int}}^{(B)}
=
-(a+a^\dagger)\,V_D\,\sigma_x^{(D)}
-
\left(e^{i\Delta\phi}a+e^{-i\Delta\phi}a^\dagger\right)\,V_A\,\sigma_x^{(A)}. \label{eq:Hint_phasefactor}
\]

\textbf{Key point:} in this B-only travelling-wave case, \(\Delta\phi\)
is simply the \textbf{spatial phase} sampled by the two TLS.

\subsubsection{\texorpdfstring{Special case: \(\lambda/4\)
separation}{Special case: \textbackslash lambda/4 separation}}\label{special-case-lambda4-separation}

If \(x_A-x_D=\lambda/4\), then
\(k(x_A-x_D)=2\pi(\lambda/4)/\lambda=\pi/2\), so \(\Delta\phi=\pi/2\).

This is the cleanest route to a \(\pi/2\) factor with B-only coupling: a
travelling wave and a quarter-wavelength separation.

\begin{center}\rule{0.5\linewidth}{0.5pt}\end{center}

\subsection{Common confusion: is this ``E vs B
quadratures''?}\label{common-confusion-is-this-e-vs-b-quadratures}

\subsubsection{The simple answer}\label{the-simple-answer}

\begin{itemize}
\tightlist
\item
  In the travelling-wave, B-only story above, the phase \(\Delta\phi\)
  comes from \textbf{\(e^{ikx}\)}, i.e.~different complex amplitudes at
  different positions.
\item
  Writing \((e^{i\Delta\phi}a+e^{-i\Delta\phi}a^\dagger)\) as a
  ``rotated quadrature'' is \textbf{just a re-expression} of that same
  spatial phase.
\end{itemize}

There is no need to invoke electric coupling to explain the B-only
relative phase.

\begin{center}\rule{0.5\linewidth}{0.5pt}\end{center}

\subsection{\texorpdfstring{Quadratures: what the \(e^{\pm i\phi}\)
combination really
means}{Quadratures: what the e\^{}\{\textbackslash pm i\textbackslash phi\} combination really means}}\label{quadratures-what-the-epm-iphi-combination-really-means}

Define oscillator quadratures

\[
X \equiv a+a^\dagger, \qquad P \equiv i(a^\dagger-a). \label{eq:XP}
\]

Then the identity

\[
e^{i\phi}a+e^{-i\phi}a^\dagger
=
X\cos\phi + P\sin\phi \label{eq:rotated_quadrature}
\]

shows that \((e^{i\phi}a+e^{-i\phi}a^\dagger)\) is a \textbf{rotated
quadrature}.

\subsubsection{\texorpdfstring{What does \(\phi=\pi/2\)
mean?}{What does \textbackslash phi=\textbackslash pi/2 mean?}}\label{what-does-phipi2-mean}

Set \(\phi=\pi/2\) in Eq. \(\ref{eq:rotated_quadrature}\):

\[
e^{i\pi/2}a+e^{-i\pi/2}a^\dagger
=
P \times (\text{sign}). \label{eq:phi_pi_over_2}
\]

So ``\(\pi/2\) in the Hamiltonian'' is shorthand for ``coupling to the
orthogonal quadrature''.

In the travelling-wave case, that orthogonal-quadrature appearance is
simply because sampling the mode at \(x=\lambda/4\) rotates the
combination of \(a\) and \(a^\dagger\).

\begin{center}\rule{0.5\linewidth}{0.5pt}\end{center}

\subsection{How coupling to both E and B rotates the quadrature at a
single
point}\label{how-coupling-to-both-e-and-b-rotates-the-quadrature-at-a-single-point}

This is a different mechanism: it can produce an effective phase angle
even without spatial separation.

\subsubsection{\texorpdfstring{Step 1: In one common convention, \(E\)
and \(B\) correspond to different
quadratures}{Step 1: In one common convention, E and B correspond to different quadratures}}\label{step-1-in-one-common-convention-e-and-b-correspond-to-different-quadratures}

A standard quantisation choice uses the vector potential
\(\hat{\mathbf A}\propto X\), then:

\begin{itemize}
\tightlist
\item
  \(\hat{\mathbf B}=\nabla\times\hat{\mathbf A}\propto X\)
\item
  \(\hat{\mathbf E}=-\partial_t\hat{\mathbf A}\propto P\)
\end{itemize}

At a fixed position \(\mathbf r_0\) (suppress vector details):

\[
\hat E(\mathbf r_0)=E_{\mathrm{zpf}}\,P, \qquad
\hat B(\mathbf r_0)=B_{\mathrm{zpf}}\,X. \label{eq:EB_quadratures}
\]

\subsubsection{Step 2: Couple a single TLS to
both}\label{step-2-couple-a-single-tls-to-both}

Use Eq. \(\ref{eq:Hint_scalar}\) at \(\mathbf r_0\):

\[
H_{\mathrm{int}}(\mathbf r_0)
=
-\sigma_x\left(d\,\hat E(\mathbf r_0)+\mu\,\hat B(\mathbf r_0)\right). \label{eq:Hint_singlepoint}
\]

Insert Eq. \(\ref{eq:EB_quadratures}\):

\[
H_{\mathrm{int}}(\mathbf r_0)
=
-\sigma_x\left(g_E\,P + g_B\,X\right), \label{eq:Hint_XP}
\]

where

\begin{itemize}
\tightlist
\item
  \(g_E \equiv d E_{\mathrm{zpf}}\)
\item
  \(g_B \equiv \mu B_{\mathrm{zpf}}\)
\end{itemize}

\subsubsection{Step 3: Rewrite as a rotated
quadrature}\label{step-3-rewrite-as-a-rotated-quadrature}

Define

\[
V=\sqrt{g_B^2+g_E^2}, \qquad
\phi=\arctan\left(\frac{g_E}{g_B}\right). \label{eq:Vphi}
\]

Then

\[
g_B X + g_E P = V\left(X\cos\phi + P\sin\phi\right). \label{eq:combine_XP}
\]

Using Eq. \(\ref{eq:rotated_quadrature}\):

\[
H_{\mathrm{int}}(\mathbf r_0)
=
- V\,\sigma_x\left(e^{i\phi}a+e^{-i\phi}a^\dagger\right). \label{eq:Hint_rotated}
\]

\textbf{Interpretation:} coupling to both \(E\) and \(B\) gives two
independent linear couplings to the same oscillator, hence a rotated
quadrature.

\subsubsection{Important caveat (identical TLS at the same
point)}\label{important-caveat-identical-tls-at-the-same-point}

If two TLS are identical and co-located (and oriented the same way),
they have the same ratio \(g_E/g_B\) and therefore the same \(\phi\). In
that case:

\begin{itemize}
\tightlist
\item
  there is no \emph{relative} phase between them arising from this
  mechanism alone,
\item
  you can rephase the mode to remove the common \(\phi\) from both
  simultaneously.
\end{itemize}

To get a \emph{relative} phase from this mechanism, something must make
\(g_E/g_B\) differ between the TLS (different position in a standing
wave where \(E\) and \(B\) vary differently, different orientation,
etc.).

\begin{center}\rule{0.5\linewidth}{0.5pt}\end{center}

\subsection{\texorpdfstring{Conceptual interlude: are \(E\) and \(B\)
``in phase'' by
Maxwell?}{Conceptual interlude: are E and B ``in phase'' by Maxwell?}}\label{conceptual-interlude-are-e-and-b-in-phase-by-maxwell}

A major conceptual hurdle is distinguishing:

\begin{itemize}
\tightlist
\item
  \textbf{travelling waves} (propagating plane waves), and
\item
  \textbf{standing waves} (superpositions of counter-propagating waves,
  e.g.~cavity modes).
\end{itemize}

\subsubsection{\texorpdfstring{Travelling plane wave: \(E\) and \(B\)
are in
phase}{Travelling plane wave: E and B are in phase}}\label{travelling-plane-wave-e-and-b-are-in-phase}

A standard vacuum plane wave propagating in \(+x\):

\[
E_y(x,t)=E_0\cos(kx-\omega t), \qquad
B_z(x,t)=\frac{E_0}{c}\cos(kx-\omega t). \label{eq:plane_inphase}
\]

At fixed \(x\), the maxima/minima occur at the same times: no relative
\(\pi/2\) time shift.

A compact statement in complex-amplitude form is:

\[
\mathbf B_0=\frac{1}{\omega}\,\mathbf k\times \mathbf E_0, \label{eq:B0kE0}
\]

so both share the same space--time factor
\(e^{i(\mathbf k\cdot\mathbf r-\omega t)}\).

\subsubsection{Why the magnetic field ``subtracts'' in a standing-wave
construction}\label{why-the-magnetic-field-subtracts-in-a-standing-wave-construction}

A standing wave is built from two travelling waves of opposite
propagation direction. For the same \(E\) polarisation, reversing
propagation flips the sign of \(B\) because

\begin{itemize}
\tightlist
\item
  \(\mathbf B \propto \hat{\mathbf k}\times \mathbf E\),
\item
  \(\hat{\mathbf k}\rightarrow -\hat{\mathbf k}\) implies
  \(\mathbf B\rightarrow -\mathbf B\).
\end{itemize}

This is why, when you add counter-propagating waves, the \(E\) fields
add but the \(B\) fields subtract.

\subsubsection{\texorpdfstring{Standing wave: \(E\) and \(B\) can be
\(\pi/2\) out of phase in time at a
point}{Standing wave: E and B can be \textbackslash pi/2 out of phase in time at a point}}\label{standing-wave-e-and-b-can-be-pi2-out-of-phase-in-time-at-a-point}

Adding the two waves yields (one common form):

\[
E_y(x,t)=2E_0\cos(kx)\cos(\omega t), \qquad
B_z(x,t)=2\frac{E_0}{c}\sin(kx)\sin(\omega t). \label{eq:standing_quadrature}
\]

At a fixed \(x\) where both \(\cos(kx)\) and \(\sin(kx)\) are nonzero:

\begin{itemize}
\tightlist
\item
  \(E\propto \cos(\omega t)\)
\item
  \(B\propto \sin(\omega t)\)
\end{itemize}

This is a genuine \(\pi/2\) time shift.

\subsubsection{\texorpdfstring{Common confusion: ``cos vs sin means
\(\pi/2\)''}{Common confusion: ``cos vs sin means \textbackslash pi/2''}}\label{common-confusion-cos-vs-sin-means-pi2}

\begin{itemize}
\tightlist
\item
  Yes: \(\sin(\omega t)=\cos(\omega t-\pi/2)\).
\item
  But you must compare \textbf{two signals with the same argument} at
  the same location/time origin.
\end{itemize}

In a travelling wave, you can write both \(E\) and \(B\) using
\(\cos(kx-\omega t)\) (or both using \(\sin\)) consistently; switching
\(\cos \leftrightarrow \sin\) for one but not the other would correspond
to shifting reference phase for one field only, which is not the
physical plane-wave relation.

\begin{center}\rule{0.5\linewidth}{0.5pt}\end{center}

\subsection{Putting it all together: what ``phase'' means in the
Hamiltonian}\label{putting-it-all-together-what-phase-means-in-the-hamiltonian}

There are two distinct (but algebraically similar) appearances of
phases:

\subsubsection{1) Spatial phase (travelling wave, even with
B-only)}\label{spatial-phase-travelling-wave-even-with-b-only}

\begin{itemize}
\tightlist
\item
  The mode function has \(u(x)=e^{ikx}\).
\item
  Different TLS positions sample different complex coefficients.
\item
  This yields a relative phase \(\Delta\phi=k(x_A-x_D)\) in the
  coupling, Eq. \(\ref{eq:deltaphi}\).
\item
  \(\lambda/4\) separation gives \(\Delta\phi=\pi/2\).
\end{itemize}

\subsubsection{2) Quadrature rotation (coupling to both E and B at one
point)}\label{quadrature-rotation-coupling-to-both-e-and-b-at-one-point}

\begin{itemize}
\tightlist
\item
  In a common quantisation convention, \(B\sim X\) and \(E\sim P\).
\item
  Coupling to both gives \(g_BX+g_EP=V(X\cos\phi+P\sin\phi)\) with
  \(\phi=\arctan(g_E/g_B)\).
\item
  This produces an \(e^{\pm i\phi}\) structure even without spatial
  separation, Eq. \(\ref{eq:Hint_rotated}\).
\item
  For identical TLS at the same point, this does not create a relative
  phase by itself.
\end{itemize}

\begin{center}\rule{0.5\linewidth}{0.5pt}\end{center}

\subsection{Takeaways}\label{takeaways}

\begin{itemize}
\tightlist
\item
  A travelling-wave spatial mode \(u(x)=e^{ikx}\) gives
  \textbf{position-dependent complex coupling}; the relative phase is
  \(\Delta\phi=k\Delta x\) (Eq. \(\ref{eq:deltaphi}\)).
\item
  A \(\pi/2\) phase factor in \((e^{i\phi}a+e^{-i\phi}a^\dagger)\)
  corresponds to coupling to the \textbf{orthogonal oscillator
  quadrature} (Eq. \(\ref{eq:rotated_quadrature}\)).
\item
  With \textbf{B-only coupling}, a \(\pi/2\) \emph{relative} phase
  arises cleanly from \textbf{\(\lambda/4\) spatial separation} in a
  travelling wave (Eq. \(\ref{eq:Hint_phasefactor}\)).
\item
  Coupling to both \textbf{E and B} at a single point yields a
  \textbf{quadrature rotation} with \(\phi=\arctan(g_E/g_B)\) (Eq.
  \(\ref{eq:Vphi}\)), even without spatial separation.
\item
  Maxwell's equations allow \(E\) and \(B\) to be \textbf{in phase in
  travelling waves} (Eq. \(\ref{eq:plane_inphase}\)) but often
  \textbf{\(\pi/2\) out of phase in standing waves} (Eq.
  \(\ref{eq:standing_quadrature}\)).
\end{itemize}

\printbibliography


\end{document}
