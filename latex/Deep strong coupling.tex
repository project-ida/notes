% Options for packages loaded elsewhere
\PassOptionsToPackage{unicode}{hyperref}
\PassOptionsToPackage{hyphens}{url}
\documentclass[
]{article}
\usepackage{xcolor}
\usepackage{amsmath,amssymb}

\usepackage{hyperref}
\usepackage{ulem}  % For underlining links

\hypersetup{
    colorlinks=false,         % Disable colored links (boxes will appear)
    citebordercolor=red,      % Border color for citation links
    linkbordercolor=red,      % Border color for internal links
    urlbordercolor=blue,      % Border color for external links (URLs)
    pdfborder={0 0 2}         % Specifies the border thickness: {horizontal vertical thickness}
}


% Only underline external links (URLs)
\let\oldhref\href
\renewcommand{\href}[2]{\ifx#1\urlprefix\oldhref{#1}{#2}\else\uline{\oldhref{#1}{#2}}\fi}


\renewcommand{\[}{\begin{equation}}
\renewcommand{\]}{\end{equation}}


\setcounter{secnumdepth}{5}
\usepackage{iftex}
\ifPDFTeX
  \usepackage[T1]{fontenc}
  \usepackage[utf8]{inputenc}
  \usepackage{textcomp} % provide euro and other symbols
\else % if luatex or xetex
  \usepackage{unicode-math} % this also loads fontspec
  \defaultfontfeatures{Scale=MatchLowercase}
  \defaultfontfeatures[\rmfamily]{Ligatures=TeX,Scale=1}
\fi
\usepackage{lmodern}
\ifPDFTeX\else
  % xetex/luatex font selection
\fi
% Use upquote if available, for straight quotes in verbatim environments
\IfFileExists{upquote.sty}{\usepackage{upquote}}{}
\IfFileExists{microtype.sty}{% use microtype if available
  \usepackage[]{microtype}
  \UseMicrotypeSet[protrusion]{basicmath} % disable protrusion for tt fonts
}{}
\makeatletter
\@ifundefined{KOMAClassName}{% if non-KOMA class
  \IfFileExists{parskip.sty}{%
    \usepackage{parskip}
  }{% else
    \setlength{\parindent}{0pt}
    \setlength{\parskip}{6pt plus 2pt minus 1pt}}
}{% if KOMA class
  \KOMAoptions{parskip=half}}
\makeatother
\setlength{\emergencystretch}{3em} % prevent overfull lines
\providecommand{\tightlist}{%
  \setlength{\itemsep}{0pt}\setlength{\parskip}{0pt}}
\usepackage[]{biblatex}
\addbibresource{refs.bib}
\usepackage{bookmark}
\IfFileExists{xurl.sty}{\usepackage{xurl}}{} % add URL line breaks if available
\urlstyle{same}

\title{Deep strong coupling}
  \author{Matt Lilley}
  \date{\today}  % Default to today if no date is provided

\begin{document}
\maketitle

\section{Introduction}\label{introduction}

Coupling refers to the interaction between two systems, where the total
energy is not a simple sum of the energies of each system. Instead, the
total energy also depends on the combined states of both, with each
system influencing the other in ways that cannot be separated. We often
express the ideas through a Lagrangian or Hamiltonian.

The aim of these notes is to build some intuition for a quantum systems
that have extremely strong coupling. In the
\href{https://www.nature.com/articles/s42254-018-0006-2}{quantum optics
literature}, the use the terms:

\begin{itemize}
\tightlist
\item
  Weak
\item
  Strong
\item
  Ultra strong
\item
  Deep strong
\end{itemize}

To describe the different regimes. We'll begin with a classical example
and use the quantum optics language

\section{A Classical example}\label{a-classical-example}

For example, consider two identical pendulums of length \(l\) and mass
\(m\) connected by a spring whose stiffness is characterised by \(k\).
In the small angle approximation (\(\theta_1, \theta_1 \ll 1\)), the
Hamiltonian is: \[
H = \frac{m l^2}{2} \dot{\theta}_1^2 + \frac{m g l}{2} \theta_1^2 + \frac{m l^2}{2} \dot{\theta}_2^2 + \frac{m g l}{2} \theta_2^2 + \frac{1}{2} k l^2(\theta_1 - \theta_2)^2
\] The first four terms are the simple sum of each individual pendulum.
The third term arrises due to the coupling.

More abstractly, we can write: \[
H = H_1 + H_2+ V_{\text{coupling}}
\]

and associate frequencies to the different parts:

\begin{itemize}
\tightlist
\item
  \(\omega_1 = \sqrt{g/l}\)
\item
  \(\omega_2 = \sqrt{g/l}\)
\item
  \(\omega_{\rm coupling} = \sqrt{k/m}\)
\end{itemize}

The system behaves quite differently depending on the strength of the
coupling which is proportional to the different frequencies.

\subsection{Weak coupling}\label{weak-coupling}

In reality, there is always dissipation which cannot be properly
captured in a Hamiltonian description. We can however define a
dissipation rate \(\gamma_{\rm diss}\) whose magnitude also changes the
system behaviour.

When
\(\omega_{\rm coupling} \ll \gamma_{\rm diss} \ll \omega_1,\omega_2\),
the coupling is described as weak.

There are very small changes in the natural frequencies of the system as
compared to the uncoupled case:

\begin{itemize}
\tightlist
\item
  \(\omega_1 \rightarrow \omega_+ = \sqrt{g/l}\)
\item
  \(\omega_2 \rightarrow \omega_- = \sqrt{g/l + 2k/m}\)
\end{itemize}

The energy does not however move back and forwards between the pendulums
because of the large dissipation.

\subsection{Strong coupling}\label{strong-coupling}

When the coupling is ``strong'' in the sense that
\(\gamma_{\rm diss} \ll \omega_{\rm coupling} \ll \omega_1,\omega_2\),
dissipation is small enough to allow energy to be slowly
\href{https://www.youtube.com/watch?v=CjJVBvDNxcE&t=57s}{exchanged
between the two pendulums}. The motion is characterised by individual
swings happening with a frequency \(\approx \sqrt{g/l}\) where the
amplitude of those swings gradually undulates on a timescale
characterised by \(\omega_{\rm exchange} = k/2m\omega_p\).

Strong coupling allows us to still conceptually consider the pendulums
as having well defined identities in the sense that they have their own
natural frequencies. As the coupling becomes larger, this is no longer
the case.

\subsection{Ultra strong coupling}\label{ultra-strong-coupling}

When the coupling is ``ultra strong'' in the sense that
\(\gamma_{\rm diss} \ll \omega_{\rm coupling} \sim \omega_1,\omega_2\),
energy exchange happens on the time scale of a single swing of one of
the pendulums. The two natural natural frequencies can be noticeably
discerned, \(\omega_+ = \sqrt{g/l}\) when both pendulums move ``in
phase'' (the spring is not stretched) with one another and
\(\omega_- = \sqrt{g/l + 2k/m}\) when the pendulums move ``out of
phase''.

\subsection{Deep strong coupling}\label{deep-strong-coupling}

When
\(\gamma_{\rm diss} \ll \omega_1,\omega_2 \ll \omega_{\rm coupling}\) ,
the coupling dominates over everything else and we enter into a regime
called ``Deep strong coupling''. Energy transfer between the two
pendulums is so fast that it's almost instantaneous and so it's not
possible to move one pendulum without the other - they act as a single
rigid body.

\section{Two level systems}\label{two-level-systems}

\[
H = \frac{\Delta E}{2} \sigma_z + \hbar\omega_A\left(a^{\dagger}a +\frac{1}{2}\right) + U\left( b^{\dagger} + b \right)\sigma_x
\]

\printbibliography


\end{document}
