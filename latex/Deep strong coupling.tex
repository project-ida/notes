% Options for packages loaded elsewhere
\PassOptionsToPackage{unicode}{hyperref}
\PassOptionsToPackage{hyphens}{url}
\documentclass[
]{article}
\usepackage{xcolor}
\usepackage{amsmath,amssymb}

\usepackage{hyperref}
\usepackage{ulem}  % For underlining links

\hypersetup{
    colorlinks=false,         % Disable colored links (boxes will appear)
    citebordercolor=red,      % Border color for citation links
    linkbordercolor=red,      % Border color for internal links
    urlbordercolor=blue,      % Border color for external links (URLs)
    pdfborder={0 0 2}         % Specifies the border thickness: {horizontal vertical thickness}
}


% Only underline external links (URLs)
\let\oldhref\href
\renewcommand{\href}[2]{\ifx#1\urlprefix\oldhref{#1}{#2}\else\uline{\oldhref{#1}{#2}}\fi}


\renewcommand{\[}{\begin{equation}}
\renewcommand{\]}{\end{equation}}


\setcounter{secnumdepth}{5}
\usepackage{iftex}
\ifPDFTeX
  \usepackage[T1]{fontenc}
  \usepackage[utf8]{inputenc}
  \usepackage{textcomp} % provide euro and other symbols
\else % if luatex or xetex
  \usepackage{unicode-math} % this also loads fontspec
  \defaultfontfeatures{Scale=MatchLowercase}
  \defaultfontfeatures[\rmfamily]{Ligatures=TeX,Scale=1}
\fi
\usepackage{lmodern}
\ifPDFTeX\else
  % xetex/luatex font selection
\fi
% Use upquote if available, for straight quotes in verbatim environments
\IfFileExists{upquote.sty}{\usepackage{upquote}}{}
\IfFileExists{microtype.sty}{% use microtype if available
  \usepackage[]{microtype}
  \UseMicrotypeSet[protrusion]{basicmath} % disable protrusion for tt fonts
}{}
\makeatletter
\@ifundefined{KOMAClassName}{% if non-KOMA class
  \IfFileExists{parskip.sty}{%
    \usepackage{parskip}
  }{% else
    \setlength{\parindent}{0pt}
    \setlength{\parskip}{6pt plus 2pt minus 1pt}}
}{% if KOMA class
  \KOMAoptions{parskip=half}}
\makeatother
\setlength{\emergencystretch}{3em} % prevent overfull lines
\providecommand{\tightlist}{%
  \setlength{\itemsep}{0pt}\setlength{\parskip}{0pt}}
\usepackage[]{biblatex}
\addbibresource{refs.bib}
\usepackage{bookmark}
\IfFileExists{xurl.sty}{\usepackage{xurl}}{} % add URL line breaks if available
\urlstyle{same}

\title{Deep strong coupling}
  \author{Matt Lilley}
  \date{\today}  % Default to today if no date is provided

\begin{document}
\maketitle

\section{Introduction}\label{introduction}

Coupling refers to the interaction between two systems, where the total
energy is not a simple sum of the energies of each system. Instead, the
total energy also depends on the combined states of both, with each
system influencing the other in ways that cannot be separated. We often
express the ideas through a Lagrangian or Hamiltonian.

The aim of these notes is to build some intuition for a quantum systems
that have extremely strong coupling. In the
\href{https://www.nature.com/articles/s42254-018-0006-2}{quantum optics
literature}, the use the terms:

\begin{itemize}
\tightlist
\item
  Weak
\item
  Strong
\item
  Ultra strong
\item
  Deep strong
\end{itemize}

to describe the different regimes.

We'll begin with a classical example and use the quantum optics language
above. We choose the example of coupled pendulums because it turns out
each quantum state with a well defined energy behaves like it's own
pendulum (see
e.g.~\href{https://journals.aps.org/pra/abstract/10.1103/PhysRevA.85.052111}{Briggs
et.al}).

\section{A Classical example}\label{a-classical-example}

For example, consider two identical pendulums of length \(l\) and mass
\(m\) connected by a spring whose stiffness is characterised by \(k\).
In the small angle approximation (\(\theta_1, \theta_1 \ll 1\)), the
Hamiltonian is:

\[
H = \frac{m l^2}{2} \dot{\theta}_1^2 + \frac{m g l}{2} \theta_1^2 + \frac{m l^2}{2} \dot{\theta}_2^2 + \frac{m g l}{2} \theta_2^2 + \frac{1}{2} k l^2(\theta_1 - \theta_2)^2
\] The first four terms are the simple sum of each individual pendulum.
The third term arrises due to the coupling.

More abstractly, we can write: \[
H = H_1 + H_2+ V_{\text{coupling}}
\]

and associate frequencies to the different parts:

\begin{itemize}
\tightlist
\item
  \(\omega_1 = \sqrt{g/l}\)
\item
  \(\omega_2 = \sqrt{g/l}\)
\item
  \(\omega_{\rm coupling} = \sqrt{k/m}\)
\end{itemize}

The system behaves quite differently depending on the strength of the
coupling which is proportional to the different frequencies.

\subsection{Weak coupling}\label{weak-coupling}

In reality, there is always dissipation which cannot be properly
captured in a Hamiltonian description. We can however define a
dissipation rate \(\gamma_{\rm diss}\) whose magnitude also changes the
system behaviour.

When
\(\omega_{\rm coupling} \ll \gamma_{\rm diss} \ll \omega_1,\omega_2\),
the coupling is described as weak.

There are very small changes in the natural frequencies of the system as
compared to the uncoupled case:

\begin{itemize}
\tightlist
\item
  \(\omega_1 \rightarrow \omega_+ = \sqrt{g/l}\)
\item
  \(\omega_2 \rightarrow \omega_- = \sqrt{g/l + 2k/m}\)
\end{itemize}

The energy does not however move back and forwards between the pendulums
because of the large dissipation.

\subsection{Strong coupling}\label{strong-coupling}

When the coupling is ``strong'' in the sense that
\(\gamma_{\rm diss} \ll \omega_{\rm coupling} \ll \omega_1,\omega_2\),
dissipation is small enough to allow energy to be slowly
\href{https://www.youtube.com/watch?v=CjJVBvDNxcE&t=57s}{exchanged
between the two pendulums}. The motion is characterised by individual
swings happening with a frequency \(\approx \sqrt{g/l}\) where the
amplitude of those swings gradually undulates on a timescale
characterised by \(\omega_{\rm exchange} = k/2m\omega_p\). This exchange
happens most effectively when the pendulums have the same length, so
that their natural frequencies are the same .

Strong coupling allows us to still conceptually consider the pendulums
as having well defined identities in the sense that they have their own
natural frequencies. As the coupling becomes larger, this is no longer
the case.

\subsection{Ultra strong coupling}\label{ultra-strong-coupling}

When the coupling is ``ultra strong'' in the sense that
\(\gamma_{\rm diss} \ll \omega_{\rm coupling} \sim 0.1\times  \omega_1,\omega_2\),
energy exchange happens on the time scale of a single swing of one of
the pendulums. The two natural natural frequencies can be noticeably
discerned, \(\omega_+ = \sqrt{g/l}\) when both pendulums move ``in
phase'' (the spring is not stretched) with one another and
\(\omega_- = \sqrt{g/l + 2k/m}\) when the pendulums move ``out of
phase''.

The coupling is getting strong enough so that more energy can be
exchanged between pendulums of different lengths.

This exact boundary for this regime is somewhat artificial - there is
nothing particularly special about the value
\(0.1 \omega_{\rm 1}, \omega_2\). This value was first used as part of
the quantum optics literature.

\subsection{Deep strong coupling}\label{deep-strong-coupling}

When
\(\gamma_{\rm diss} \ll \omega_1,\omega_2 \lesssim \omega_{\rm coupling}\)
, the coupling begins to dominate over everything else and we enter into
a regime called ``Deep strong coupling''. Energy transfer between the
two pendulums is so fast that it's almost instantaneous and so it's not
possible to move one pendulum without the other - they act as a single
rigid body.

\section{A quantum example}\label{a-quantum-example}

A canonical quantum example is a single two level system (TLS)
interacting with a quantised field. The Hamiltonian can be written as:
\[
H = \frac{\Delta E}{2} \sigma_z + \hbar\omega\left(b^{\dagger}b +\frac{1}{2}\right) + U\left( b^{\dagger} + b \right)\sigma_x
\label{eq:H}
\]

where \(\Delta E\) is the transition energy between the 2 levels of the
TLS, \(\hbar\omega\) is the energy of each quantum of the field, and
\(U\) is the coupling constant between the TLS and the field. The
\(\sigma\) operators are the Pauli matrices and \(b^{\dagger}\), \(b\)
are the field creation and annihilation operators respectively.

Although a TLS has just 2 states (denoted \(|\pm\rangle\)), the
quantised field has infinitely many states (denoted by the number of
quanta \(|n\rangle\)). The combined state of the system (denoted
\(|n, \pm\rangle\)) therefore has infinitely many states and so
conceptually the system behaves like infinitely many pendulums coupled
together. The frequency of these conceptual pendulums is determined by
the energy of the states.

Much like the classical example, the dynamics depend on the relative
sizes of the different terms in the Hamiltonian. For the quantum case
however, it's not enough just to compare the various constants
\(U, \hbar \omega, \Delta E\), we must also consider how many field
quanta \(n\) we have. This is because of how the field operators work:
\[
b^{\dagger} |n,\pm\rangle = \sqrt{n+1}|n+1,\pm\rangle \\
\]

\[
b^{} |n,\pm\rangle = \sqrt{n}|n-1,\pm\rangle \\
\]

\[
b^{\dagger}b |n,\pm\rangle = n|n,\pm\rangle
\]

The more field quanta we have, the larger the field and coupling terms
will be compared to the TLS term.

\subsection{Weak coupling}\label{weak-coupling-1}

Much like the classical pendulums, the quantum states can suffer various
forms of ``dissipation''. As the quantum systems interacts with outside
systems, it can cause:

\begin{itemize}
\tightlist
\item
  Dephasing - where the phase relationship between each of the quantum
  states starts to change over time
\item
  Decoherence - the system is forced out of a superposition state and
  into a well defined state aka ``collapse of the wavefunciton''
\end{itemize}

If we define \(\hbar\gamma_{\rm diss}\) as a characteristic energy
associated with the above processes, then
\(\sqrt{n}U \ll \hbar \gamma_{\rm diss} \ll \Delta E , n\hbar \omega\)
defines the weak coupling regime. Spontaneous emission is the most
characteristic feature of weak coupling where an excited TLS with less
field quanta (e.g.~\(|n,+\rangle\)) is coupled to a ground state TLS
with more field quanta (e.g.~\(|n+1,-\rangle\) ). The coupling is
however not so strong that the field quanta can get reabsorbed.

\begin{quote}
Note that truly irreversible spontaneous emission also relies on there
being a continuum of states instead of discrete ones.
\end{quote}

\subsection{Strong coupling}\label{strong-coupling-1}

As in the classical case, when the coupling is strong in the sense that
\(\hbar\gamma_{\rm diss} \ll \sqrt{n}U \ll \Delta E , n\hbar \omega\),
there is time for slow exchange between the different quantum states
(remember each state is like it's own pendulum). Unlike the classical
case, where it's energy that's exchanged, in the quantum case it's state
occupation probability \(|\psi|^2\) that's exchanged.

In order for exchange to occur effectively, the quantum states have to
have the same energy. This is equivalent to the conceptual pendulums
having the same length. Two such states are often described as
``resonant'' with one another. Whether or not the system has any
resonances depends on the relationship between \(\hbar \omega\) and
\(\Delta E\).

\subsubsection{Matched field and TLS}\label{matched-field-and-tls}

When the field is matched to the TLS, \(\Delta E = \hbar\omega\). This
is the most widely discussed regime in which a transition of the TLS
(often an atomic transition) results in the emission of a single field
quanta (often a photon). In a cavity where discrete states can be
arranged and field quanta can be confined, this results in occupation
probability oscillating between states like \(|n,+\rangle\) and
\(|n+1,-\rangle\). These oscillations are called
\href{https://en.wikipedia.org/wiki/Jaynes\%E2\%80\%93Cummings_model\#Vacuum_Rabi_oscillations}{Rabi
oscillations} which have a frequency
\(\Omega/\hbar\omega \sim \sqrt{n}U/\hbar\omega\)

\subsubsection{Mismatched field and TLS}\label{mismatched-field-and-tls}

When the field is mismatched to the TLS, \(\Delta E \neq \hbar\omega\) .
If the mismatch is arbitrary, e.g.~\(\Delta E/ \hbar\omega = 3.23677\)
then Rabi oscillations won't occur and, in a cavity, a system
initialised in state \(|n,+\rangle\) will stay there forever.

However, if \(\Delta E = m \hbar\omega\) where \(m=3, 5, 7 ...\) then
\(|n,+\rangle\) is resonant with \(|n+3,-\rangle\), \(|n+5,-\rangle\),
\(|n+7,-\rangle ...\) and so Rabi oscillations can once again occur. The
frequency is slower
\(\Omega/\hbar\omega \sim (\sqrt{n}U/\hbar\omega)^m\) and so for larger
\(m\) and so the emission of multiple quanta becomes less and less
likely.

\subsection{Ultra strong coupling}\label{ultra-strong-coupling-1}

When the coupling becomes a sizeable fraction of the TLS and field
energies,
\(\hbar\gamma_{\rm diss} \ll U \sim 0.1 \times \Delta E , \hbar \omega\),
non-resonant states begin to gain significant occupancy. For example, a
system can start out in state \(|0,+\rangle\) with 100\% probability and
overtime a state \(|1,+\rangle\) can gain a non-trivial amount of
occupation probability. Although this superficially appears to violate
energy conservation, the energy in the coupling is no longer small and
so all terms in the Hamiltonian need to be considered when thinking
about energy conservation.

\subsection{Deep strong coupling}\label{deep-strong-coupling-1}

This regime was first theoretically explored in
\href{https://journals.aps.org/prl/abstract/10.1103/PhysRevLett.105.263603}{2010
by Casanova at.al}. It's defined by
\(\hbar\gamma_{\rm diss} \ll \Delta E , \hbar \omega \lesssim U\) where
the coupling is now plays an equal role as the TLS and the oscillator.
In this regime, the TLS and the oscillator can't conceptually be
separated - they don't have well defined identities anymore. Energy can
move between the oscillator and TLS freely in the sense that we're no
longer bound by strict resonance requirements between the TLS and the
oscillator.

\printbibliography


\end{document}
