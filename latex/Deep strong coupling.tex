% Options for packages loaded elsewhere
\PassOptionsToPackage{unicode}{hyperref}
\PassOptionsToPackage{hyphens}{url}
\documentclass[
]{article}
\usepackage{xcolor}
\usepackage{amsmath,amssymb}

\usepackage{hyperref}
\usepackage{ulem}  % For underlining links
\normalem

\hypersetup{
    colorlinks=false,         % Disable colored links (boxes will appear)
    citebordercolor=red,      % Border color for citation links
    linkbordercolor=red,      % Border color for internal links
    urlbordercolor=blue,      % Border color for external links (URLs)
    pdfborder={0 0 2}         % Specifies the border thickness: {horizontal vertical thickness}
}


% Only underline external links (URLs)
\let\oldhref\href
\renewcommand{\href}[2]{\ifx#1\urlprefix\oldhref{#1}{#2}\else\uline{\oldhref{#1}{#2}}\fi}


\renewcommand{\[}{\begin{equation}}
\renewcommand{\]}{\end{equation}}


\setcounter{secnumdepth}{5}
\usepackage{iftex}
\ifPDFTeX
  \usepackage[T1]{fontenc}
  \usepackage[utf8]{inputenc}
  \usepackage{textcomp} % provide euro and other symbols
\else % if luatex or xetex
  \usepackage{unicode-math} % this also loads fontspec
  \defaultfontfeatures{Scale=MatchLowercase}
  \defaultfontfeatures[\rmfamily]{Ligatures=TeX,Scale=1}
\fi
\usepackage{lmodern}
\ifPDFTeX\else
  % xetex/luatex font selection
\fi
% Use upquote if available, for straight quotes in verbatim environments
\IfFileExists{upquote.sty}{\usepackage{upquote}}{}
\IfFileExists{microtype.sty}{% use microtype if available
  \usepackage[]{microtype}
  \UseMicrotypeSet[protrusion]{basicmath} % disable protrusion for tt fonts
}{}
\makeatletter
\@ifundefined{KOMAClassName}{% if non-KOMA class
  \IfFileExists{parskip.sty}{%
    \usepackage{parskip}
  }{% else
    \setlength{\parindent}{0pt}
    \setlength{\parskip}{6pt plus 2pt minus 1pt}}
}{% if KOMA class
  \KOMAoptions{parskip=half}}
\makeatother
\setlength{\emergencystretch}{3em} % prevent overfull lines
\providecommand{\tightlist}{%
  \setlength{\itemsep}{0pt}\setlength{\parskip}{0pt}}
\usepackage[]{biblatex}
\addbibresource{refs.bib}
\usepackage{bookmark}
\IfFileExists{xurl.sty}{\usepackage{xurl}}{} % add URL line breaks if available
\urlstyle{same}

\title{Deep strong coupling}
  \author{Matt Lilley}
  \date{\today}  % Default to today if no date is provided

\begin{document}
\maketitle

\section{Introduction}\label{introduction}

Coupling refers to the interaction between two systems, where the total
energy is not a simple sum of the energies of each system. Instead, the
total energy also depends on the combined states of both, with each
system influencing the other in ways that cannot be separated. We often
express the ideas through a Lagrangian or Hamiltonian.

The aim of these notes is to build some intuition for quantum systems
that have extremely strong coupling. In the
\href{https://www.nature.com/articles/s42254-018-0006-2}{quantum optics
literature}, the terms:

\begin{itemize}
\tightlist
\item
  Weak
\item
  Strong
\item
  Ultra strong
\item
  Deep strong
\end{itemize}

are used to describe the different regimes.

We will begin with a classical example and use the quantum optics
language above. We choose the example of coupled pendulums because it
turns out each quantum state with a well-defined energy behaves like its
own pendulum (see
e.g.~\href{https://journals.aps.org/pra/abstract/10.1103/PhysRevA.85.052111}{Briggs
et.al}).

\section{Coupled pendulums}\label{coupled-pendulums}

For example, consider two identical pendulums of length \(l\) and mass
\(m\) connected by a spring whose stiffness is characterised by \(k\).
In the small angle approximation (\(\theta_1, \theta_1 \ll 1\)), the
Hamiltonian is:

\[
H_{\rm pen} = \frac{m l^2}{2} \dot{\theta}_1^2 + \frac{m g l}{2} \theta_1^2 + \frac{m l^2}{2} \dot{\theta}_2^2 + \frac{m g l}{2} \theta_2^2 + \frac{1}{2} k l^2(\theta_1 - \theta_2)^2
\]

The first four terms are the simple sum of each individual pendulum. The
last term arises due to the coupling.

More abstractly, we can write:

\[
H = H_1 + H_2+ V_{\text{coupling}}
\]

and we can associate frequencies to the different parts:

\begin{itemize}
\tightlist
\item
  \(\omega_1 = \sqrt{g/l}\)
\item
  \(\omega_2 = \sqrt{g/l}\)
\item
  \(\omega_{\rm coupling} = \sqrt{k/m}\)
\end{itemize}

The system behaves quite differently depending on the strength of the
coupling which is proportional to the different frequencies.

\subsection{Weak coupling}\label{weak-coupling}

In reality, there is always dissipation which cannot be properly
captured in a Hamiltonian description. We can however define a
dissipation rate \(\gamma_{\rm diss}\) whose magnitude also changes the
system behaviour.

The coupling is considered weak when
\(\omega_{\rm coupling} \ll \gamma_{\rm diss} \ll \omega_1,\omega_2\).

There are very small changes in the natural frequencies of the system as
compared to the uncoupled case:

\begin{itemize}
\tightlist
\item
  \(\omega_1 \rightarrow \omega_+ = \sqrt{g/l}\)
\item
  \(\omega_2 \rightarrow \omega_- = \sqrt{g/l + 2k/m}\)
\end{itemize}

The energy does not however move back and forwards between the pendulums
because the large dissipation removes the energy before any transfer can
occur.

\subsection{Strong coupling}\label{strong-coupling}

When the coupling is ``strong'' in the sense that
\(\gamma_{\rm diss} \ll \omega_{\rm coupling} \ll \omega_1,\omega_2\),
dissipation is small enough to allow energy to be slowly
\href{https://www.youtube.com/watch?v=CjJVBvDNxcE&t=57s}{exchanged}
between the two pendulums. The motion is characterised by individual
swings happening with a frequency \(\approx \sqrt{g/l}\) where the
amplitude of those swings gradually undulates on a timescale
characterised by \(\omega_{\rm exchange} =  (k/m) / \omega_+\). This
exchange happens most effectively when the pendulums have the same
length, so that their natural frequencies are the same.

Strong coupling allows us to still conceptually consider the pendulums
as having well defined identities in the sense that they have their own
natural frequencies. As the coupling becomes larger, this is no longer
the case.

\subsection{Ultra strong coupling}\label{ultra-strong-coupling}

When the coupling is ``ultra strong'' in the sense that
\(\gamma_{\rm diss} \ll \omega_{\rm coupling} \sim 0.1\times  \omega_1,\omega_2\),
energy exchange happens on the time scale of a single swing of one of
the pendulums. The two natural frequencies can be noticeably discerned,
\(\omega_+ = \sqrt{g/l}\) when both pendulums move ``in phase'' (the
spring is not stretched) with one another and
\(\omega_- = \sqrt{g/l + 2k/m}\) when the pendulums move ``out of
phase'' (the spring is periodically compressed and expanded)

The coupling is getting strong enough so that more energy can be
exchanged between pendulums of different lengths.

This exact boundary for this regime is somewhat artificial; there is
nothing particularly special about the value
\(0.1 \omega_{\rm 1}, \omega_2\). This value was first used as part of
the quantum optics literature.

\subsection{Deep strong coupling}\label{deep-strong-coupling}

When
\(\gamma_{\rm diss} \ll \omega_1,\omega_2 \lesssim \omega_{\rm coupling}\),
the coupling begins to dominate over everything else and we enter into a
regime called ``Deep strong coupling''. Energy transfer between the two
pendulums is so fast that it is almost instantaneous and so it is not
possible to move one pendulum without the other - they act as a single
rigid body.

\section{Rabi model}\label{rabi-model}

A canonical quantum example is a single two level system (TLS)
interacting with a single mode (i.e.~single frequency/wavelength) of a
quantised field. This is often called the Rabi model and its Hamiltonian
can be written as: \[
H_{\rm Rabi} = \frac{\Delta E}{2} \sigma_z + \hbar\omega\left(a^{\dagger}a +\frac{1}{2}\right) + U\left( a^{\dagger} + a \right)(\sigma_+ + \sigma_-)
\label{eq:rabiH}
\]

where \(\Delta E\) is the transition energy between the 2 levels of the
TLS, \(\hbar\omega\) is the energy of each quantum of the field, and
\(U\) is the coupling constant between the TLS and the field. The
\(\sigma\) operators are the
\href{https://ocw.mit.edu/courses/5-61-physical-chemistry-fall-2007/3b1fb40c61e7f939861b190bedbc57a7_lecture24.pdf}{Pauli
spin matrices} that act on the TLS, where \(\sigma_+\) and \(\sigma_-\)
act as raising and lowering operators. The \(a^{\dagger}\), \(a\) are
the field creation and annihilation operators respectively.

It is worth noting that we are using the Pauli spin matrices as a
mathematical tool to describe two levels. Just keep in mind that we are
not really talking about spin angular momentum here.

Although a TLS has just 2 states (denoted \(|\pm\rangle\)), the
quantised field has infinitely many states (denoted by the number of
quanta \(|n\rangle\)). The combined state of the system (denoted
\(|n, \pm\rangle\)) therefore has infinitely many states and so
conceptually the system behaves like infinitely many pendulums coupled
together. The frequency of these conceptual pendulums is determined by
the energy of the states.

Much like the classical example, the dynamics depend on the relative
sizes of the different terms in the Hamiltonian. For the quantum case
however, it is not enough to just compare the various constants
\(U, \hbar \omega, \Delta E\). We must also consider how many field
quanta \(n\) we have. This is because of how the field operators work:

\[
a^{\dagger} |n,\pm\rangle = \sqrt{n+1}|n+1,\pm\rangle \\
\]

\[
a |n,\pm\rangle = \sqrt{n}|n-1,\pm\rangle \\
\]

\[
a^{\dagger}a |n,\pm\rangle = n|n,\pm\rangle
\]

The more field quanta we have, the larger the field and coupling terms
will be.

\subsection{Weak coupling}\label{weak-coupling-1}

Much like the classical pendulums, the quantum states can suffer various
forms of ``dissipation''. As the quantum systems interacts with outside
systems, it can cause:

\begin{itemize}
\tightlist
\item
  Dephasing - where the phase relationship between each of the quantum
  states starts to change over time
\item
  Decoherence - the system is forced out of a superposition state and
  into a well defined state aka ``collapse of the wavefunciton''
\end{itemize}

If we define \(\hbar\gamma_{\rm diss}\) as a characteristic energy
associated with the above processes, then
\(\sqrt{n}U \ll \hbar \gamma_{\rm diss} \ll \Delta E , n\hbar \omega\)
defines the weak coupling regime. Spontaneous emission is the most
characteristic feature of weak coupling where an excited TLS with less
field quanta (e.g.~\(|n,+\rangle\)) is coupled to a ground state TLS
with more field quanta (e.g.~\(|n+1,-\rangle\) ). The coupling is
however not so strong that the field quanta can get reabsorbed.

\begin{quote}
Note that truly irreversible spontaneous emission also relies on there
being a continuum of states instead of discrete ones.
\end{quote}

\subsection{Strong coupling}\label{strong-coupling-1}

As in the classical case, when the coupling is strong in the sense that
\(\hbar\gamma_{\rm diss} \ll \sqrt{n}U \ll \Delta E , \hbar \omega\),
there is time for slow exchange between the different quantum states
(remember each state is like it's own pendulum). Unlike the classical
case, where it is energy that is exchanged, in the quantum case it is
state occupation probability \(|\psi|^2\) that is exchanged.

In order for exchange to occur effectively, the quantum states need to
have the same energy. This is equivalent to the conceptual pendulums
having the same length. Two such states are often described as
``resonant'' with one another. Whether or not the system has any
resonances depends on the relationship between \(\hbar \omega\) and
\(\Delta E\).

\subsubsection{Matched field and TLS}\label{matched-field-and-tls}

The field is matched to the TLS when \(\Delta E = \hbar\omega\). This is
the most widely discussed regime in which a transition of the TLS (often
an atomic transition) results in the emission of a single field quantum
(often a photon). In a cavity where discrete states can be arranged and
field quanta can be confined, this results in occupation probability
oscillating between states like \(|n,+\rangle\) and \(|n+1,-\rangle\).
These oscillations are called
\href{https://en.wikipedia.org/wiki/Jaynes\%E2\%80\%93Cummings_model\#Vacuum_Rabi_oscillations}{Rabi
oscillations} which have a frequency
\(\Omega/\hbar\omega \sim \sqrt{n}U/\hbar\omega\)

\subsubsection{Mismatched field and TLS}\label{mismatched-field-and-tls}

The field is mismatched to the TLS when \(\Delta E \neq \hbar\omega\) .
If the mismatch is arbitrary, e.g.~\(\Delta E/ \hbar\omega = 2.83677\)
then Rabi oscillations cannot occur because the coupling term
(\(\sim \sqrt{n}U\)) is still very small compared to \(\Delta E\) and
\(\hbar\omega\) and so it cannot accommodate any energy mismatch between
the field and the TLS.

If however, \(\Delta E = m\hbar \omega\) where \(m=3, 5, 7 ...\) then
\(|n,+\rangle\) is ``resonant'' with, \(|n+3,-\rangle\),
\(|n+5,-\rangle\), \(|n+7,-\rangle ...\) and so Rabi oscillations can
once again occur. The frequency is slower
\(\Omega/\hbar\omega \sim (\sqrt{n}U/\hbar\omega)^m\) and so for larger
\(m\) the emission of multiple quanta becomes less and less likely.

\subsection{Ultra strong coupling}\label{ultra-strong-coupling-1}

When the coupling becomes a sizeable fraction of the TLS and field
quantum,
\(\hbar\gamma_{\rm diss} \ll \sqrt{n}U \sim 0.1 \times \Delta E , \hbar \omega\),
non-resonant states begin to gain significant occupancy. For example, a
system can start out in state \(|0,+\rangle\) with 100\% probability and
over time a state \(|1,+\rangle\) can gain a non-trivial amount of
occupation probability. Although this superficially appears to violate
energy conservation, the energy in the coupling is no longer negligible
and so all terms in the Hamiltonian must be considered when thinking
about energy conservation.

The coupling term can also accommodate energy mismatches between the TLS
and the oscillator, e.g.~\(\Delta E/ \hbar\omega = 2.83677\) vs
\(\Delta E/ \hbar\omega = 3\). This makes it easier to observe the
emission of multiple quanta.

\subsection{Deep strong coupling}\label{deep-strong-coupling-1}

When the coupling becomes on the same order or greater than the TLS and
field quantum,
\(\hbar\gamma_{\rm diss} \ll \Delta E , \hbar \omega \lesssim \sqrt{n}U\),
then TLS transitions and creation/annihilation of field quanta can no
longer be understood by simply thinking about the TLS and field
exchanging energy with each other and the coupling as a kind of glue
between the two. The coupling term has an ``identity'' of its own.

This regime was first theoretically explored in
\href{https://journals.aps.org/prl/abstract/10.1103/PhysRevLett.105.263603}{2010
by Casanova at.al} where a simpler definition of ``deep strong
coupling'' was given as:

\[
\frac{U}{\hbar\omega} \gtrsim 1
\label{eq:deepstrongcoupling}
\]

Indeed, if their condition is satisfied then
\(\hbar \omega \lesssim \sqrt{n}U\) is guaranteed.

First, let us consider the case (as Casanova did) where the TLS energy
is small in the sense that \(\Delta E < \hbar \omega\). If the coupling
is in the deep strong regime so that \(U/\hbar\omega \gtrsim 1\) then,
from an energy conservation point of view, the coupling term can
spontaneously create field quanta. The field effectively ``borrows''
energy from interaction energy which can take on a form of energy debt.
There is a limit to how much energy debt that the interaction term can
take on because the energy cost of the quanta grows like \(n\) whereas
the interaction terms grows more slowly like \(\sqrt{n}\). At some level
of quanta, the energy required to make an extra quanta outstrips the
interaction's ability to provide.

From these ideas, we can find out how many quanta get created by
equating the field energy to the coupling energy in the Hamiltonian:

\[
n\hbar\omega \sim \sqrt{n}U
\label{eq:couplingbalancefield}
\]

This gives us:

\[
n \sim \left(\frac{U}{\hbar\omega}\right)^2
\label{eq:selfconsistentn}
\]

When you work out the detailed maths, you end up with
\(n = 4(U/\hbar\omega)^2\).

When the TLS energy is not small (\(\Delta E \gtrsim \hbar \omega\))
then there is an additional energy equivalence to consider: \[
\Delta E \sim \sqrt{n}U
\label{eq:couplingbalancetls}
\]

Eq. \(\ref{eq:couplingbalancefield}\) and Eq.
\(\ref{eq:couplingbalancetls}\) can be solved simultaneously to
eliminate \(n\). This gives us a relationship between
\(U,\Delta E,\hbar\omega\):

\[
\begin{aligned}
\Delta E &\sim \sqrt{\left(\frac{U}{\hbar\omega}\right)^2}U \\
\Delta E &\sim \frac{U^2}{\hbar\omega} \\
U &\sim \sqrt{\Delta E \hbar\omega} \\
\frac{U}{\hbar\omega} &\sim \sqrt{\frac{\Delta E}{\hbar\omega}}
\end{aligned}
\label{eq:uconstraint}
\]

For the case when the TLS energy dominates over the field,
\(\Delta E \gg \hbar \omega\), Eq. \(\ref{eq:uconstraint}\) is a more
appropriate coupling threshold to consider than Eq.
\(\ref{eq:deepstrongcoupling}\). This regime is sometimes called
``dispersive deep strong coupling'' as was first coined by
\href{https://journals.aps.org/pra/abstract/10.1103/PhysRevA.95.013827}{Felicetti
et.al in 2017}. Again, when you do the detailed maths, you get an extra
constant so that:

\[
\frac{U}{\hbar\omega} \gtrsim \frac{1}{2}\sqrt{\frac{\Delta E}{\hbar\omega}}
\label{eq:superradianttransition}
\]

In the extreme case when \(\Delta E/\hbar\omega \rightarrow \infty\)
then Eq. \(\ref{eq:superradianttransition}\) represents a boundary of
what is called a superradiant phase transition (detailed in
\href{https://journals.aps.org/prl/abstract/10.1103/PhysRevLett.115.180404}{2015
by Hwang et.al}). When you go above this critical coupling, the system
undergoes a phase change where the lowest energy state involves a
non-zero amount of field quanta. In other words, above this threshold
the TLS freely exchanges energy with the field and the usual
restrictions around having a matched TLS and field are not important.
Superradiant phase transitions have been discussed for much longer times
\href{https://en.wikipedia.org/wiki/Dicke_model\#Superradiant_transition_and_Dicke_superradiance}{in
relation to the Dicke model} and we'll come back to look at this later
on.

\subsubsection{Relativistic phonon nuclear
coupling}\label{relativistic-phonon-nuclear-coupling}

For acoustic phonons, we choose a notation \(\omega \equiv \omega_A\).
From notes on
\href{https://github.com/project-ida/notes/blob/main/pdf/Coupling\%20constants\%20in\%20nuclear\%20physics.pdf}{Coupling
constants in nuclear physics}, we derived the relativistic phonon
nuclear coupling as:

\[
\frac{U}{\hbar \omega_A} = \sqrt{\frac{2}{N}} \sqrt{\frac{\Delta E}{M c^2}} \sqrt{\frac{\Delta E}{\hbar \omega_A}} \times 10^{-3}
\label{eq:phononcoupling}
\]

where \(N\) is the number of nuclei involved in the phonon motion, and
\(M\) is the mass of the nucleus.

Typically, we imagine a phonon as being the quantised oscillatory motion
of many nuclei in a lattice. However, it is possible to arrange systems
in which the motion of an isolated nucleus is considered (see
e.g.~\href{https://www.nature.com/articles/s41467-021-21425-8}{Cat et.al
2021}) - in which case \(N=1\).

Our example Hamiltonian in Eq. \(\ref{eq:rabiH}\) is for a single TLS,
so we will consider Eq. \(\ref{eq:phononcoupling}\) with \(N=1\) and
extend the Hamiltonian to many TLS later on.

For nuclear transitions mediated by phonons, \(\Delta E \sim \rm MeV\)
and \(\hbar\omega_A \sim 10 \ \rm neV\) and so
\(\Delta E/\hbar\omega \gg 1\). Therefore, Eq.
\(\ref{eq:superradianttransition}\) is the appropriate superradiant
threshold condition.

Substituting the expression for coupling (Eq.
\(\ref{eq:phononcoupling}\)) into the critical coupling expression (Eq.
\(\ref{eq:superradianttransition}\)) gives the following condition:

\[
\begin{aligned}
2\frac{U}{\hbar\omega_A}\sqrt{\frac{\hbar\omega_A}{\Delta E}} &\gtrsim 1 \\
2\sqrt{2} \sqrt{\frac{\Delta E}{M c^2}}  \times 10^{-3} &\ge 1
\end{aligned}
\label{eq:criticalphononcouplingexplicit}
\]

For nuclear transitions \(\Delta E \sim MeV\) and for a single nucleon
\(Mc^2 \sim GeV\) so it is already clear that for any sized nucleus we
will not enter the superradiant regime. For the sake of completeness,
let us evaluate Eq. \(\ref{eq:criticalphononcouplingexplicit}\) for a
transition with \(\Delta E \approx 24  \ \rm MeV\) and for a palladium
nucleus with \(Mc^2 \approx 100  \ \rm GeV\):

\[
2\sqrt{2} \sqrt{\frac{24\times10^6}{10^{11}}}  \times 10^{-3} \approx 4 \times 10^{-5} \ll 1
\label{eq:criticalphononcouplingexplicitnumbers}
\]

This confirms that a single nucleus cannot undergo a superradiant phase
transition using relativistic phonon nuclear coupling. Let us look at
another type of coupling that is also associated with oscillatory phonon
motion.

\subsubsection{Electric dipole coupling}\label{electric-dipole-coupling}

For electric dipole coupling associated with an oscillating electric
field driving phonon motion, we continue to use the notation
\(\omega \equiv \omega_A\). From notes on
\href{https://github.com/project-ida/notes/blob/main/pdf/Coupling\%20constants\%20in\%20nuclear\%20physics.pdf}{Coupling
constants in nuclear physics}, we derived the electric dipole phonon
coupling as:

\[
\frac{U}{\hbar \omega_A} = \frac{2\pi\sqrt{2}}{Z \sqrt{N}} \sqrt{\frac{M c^2}{\hbar \omega_A}} \frac{\hbar \omega_A}{E_L} A^{1/3} \times 6 \times 10^{-3}
\label{eq:dipolecoupling}
\]

If we once again take \(N=1\), then substituting the expression for
coupling (Eq. \(\ref{eq:dipolecoupling}\)) into the critical coupling
expression (Eq. \(\ref{eq:superradianttransition}\)) gives the following
condition:

\[
\begin{aligned}
2\frac{U}{\hbar\omega_A}\sqrt{\frac{\hbar\omega_A}{\Delta E}} &\gtrsim 1 \\
\frac{4\pi\sqrt{2}}{Z} \sqrt{\frac{M c^2}{\Delta E}} \frac{\hbar \omega_A}{E_L} A^{1/3} \times 6 \times 10^{-3} &\gtrsim 1
\end{aligned}
\label{eq:criticaldipolecouplingexplicit}
\]

For palladium nuclear transitions mediated by acoustic phonons:

\begin{itemize}
\tightlist
\item
  \(A \approx 106\)
\item
  \(Z \approx 106\)
\item
  \(M c^2 \approx 10^{11}\) eV
\item
  \(\Delta E \approx 24 \times 10^{6}\) eV\\
\item
  \(\hbar \omega_A \approx 10^{-8}\) eV
\end{itemize}

The localization energy \(E_L\) is:

\[
E_L = \frac{\hbar c}{R_0} = \frac{6.6 \times 10^{-34} \times 3 \times 10^8}{10^{-15}}
\]

\[
= 2 \times 10^{-10} \text{ J} = 1.2 \times 10^9 \text{ eV} \approx 10^9 \text{ eV}
\]

And so evaluating Eq. \(\ref{eq:criticaldipolecouplingexplicit}\) gives:

\[
\frac{4 \pi\sqrt{2} }{106} \times \sqrt{\frac{10^{11}}{24\times10^{6}}} \times \frac{10^{-8}}{10^9} \times 6 \times 10^{-3} \times 106^{1/3} \approx 3\times 10^{-18} \ll 1
\label{eq:dipolewithnumbers}
\]

And so we are even further away from the superradiant regime when
considering the electric dipole coupling associated with the phonon
motion for a single nucleus.

We have so far looked at the Rabi model where the number of TLS is
\(N=1\). How does the story change when we have many TLS?

\section{Dicke model}\label{dicke-model}

The Dicke model describes a system where we have \(N\) identical TLS
coupled to a single mode (i.e.~single frequency/wavelength) of a
quantised field. The Dicke Hamiltonian is a simple extension of the Rabi
Hamiltonian in Eq. \(\ref{eq:rabiH}\) in the sense that we add \(N\)
copies of the TLS terms, as shown below:

\[
H_{\text{Dicke}} = \frac{\Delta E}{2} \sum_{i=1}^N \sigma_z^{(i)} +  \hbar\omega\left(a^{\dagger}a +\frac{1}{2}\right) + U \sum_{i=1}^N (a^\dagger + a) (\sigma_+^{(i)} + \sigma_-^{(i)})
\label{eq:dickeH}
\]

The states of this system are described by
\(|n, \pm, \pm, \pm, \pm, ... \rangle\).

For this ensemble of \(N\) nuclei, the coupling is enhanced by at least
\(\sqrt{N}\) and at most \(N\) due to Dicke superradiance effects. For
more detail on the Dicke model that leads to these enhancements, see the
\href{https://github.com/project-ida/notes/blob/main/pdf/Dicke\%20model.pdf}{notes}
I made on the subject.

\begin{quote}
It should be noted that the effect of Dicke superradiance and
superradiant phase transitions are
\href{https://en.wikipedia.org/wiki/Dicke_model\#Superradiant_transition_and_Dicke_superradiance}{not
the same} phenomenon. The former involves a transient enhancement in
emission of \(N\) TLS which ultimately results in all the TLS in their
ground state and the field quanta escaping to infinity. The latter
involves a permanent change in the ground state of a cavity system in
which field and TLS are both confined.
\end{quote}

If we look back at the coupling terms in Eqs.
\(\ref{eq:phononcoupling}\) and \(\ref{eq:dipolecoupling}\), we can see
a \(1/\sqrt{N}\) term appears to reduce the coupling significantly for
very large numbers of nuclei. However, if we are able to take advantage
of Dicke effects, then the situation is very different:

\begin{itemize}
\tightlist
\item
  For fully excited systems, Dicke enhancement of the coupling scales
  like \(\sqrt{N}\) and so coupling for \(N\) nuclei is the same as for
  a single nuclei (from \(\sqrt{N}/\sqrt{N}\))
\item
  For half excited systems, Dicke enhancement of the coupling scales
  like \(N\) and so coupling for \(N\) nuclei scales like \(\sqrt{N}\)
  (from \(N/\sqrt{N}\))
\end{itemize}

And so, we have two sources of coupling enhancement:

\begin{itemize}
\tightlist
\item
  \(\sqrt{n}\) from the field part
\item
  At least \(\sqrt{N}\) and at most \(N\) from the TLS part due to Dicke
  effects
\end{itemize}

\subsubsection{Deep strong coupling}\label{deep-strong-coupling-2}

If we take the most conservative Dicke enhancement, then the condition
for deep strong coupling can be arrived at by simply replacing \(U\) in
Eq. \(\ref{eq:superradianttransition}\) with \(U\sqrt{N{}}\): \[
\frac{U\sqrt{N}}{\hbar\omega} \gtrsim \frac{1}{2}\sqrt{\frac{\Delta E}{\hbar\omega}}
\label{eq:dickesuperradianttransition}
\] As \(N\rightarrow \infty\) this condition triggers a
\href{https://royalsocietypublishing.org/doi/10.1098/rsta.2010.0333}{superradiant
phase transition} similar to what we saw in the Rabi model when
\(\Delta E / \hbar\omega \rightarrow \infty\).

If we look back at the couplings in Eqs. \(\ref{eq:phononcoupling}\) and
\(\ref{eq:dipolecoupling}\), we can see a \(1/\sqrt{N}\) terms will
cancel with the \(\sqrt{N}\) in Eq.
\(\ref{eq:dickesuperradianttransition}\). This means that our earlier
calculations with \(N=1\) will apply to an arbitrary number of TLS and
so we do not get closer to a superradiant phase transition by having
more TLS involved.

If however, we could use the most optimistic Dicke enhancement, then we
would instead have:

\[
\frac{UN}{\hbar\omega} \gtrsim \frac{1}{2}\sqrt{\frac{\Delta E}{\hbar\omega}}
\label{eq:dickesuperradianttransition2}
\]

\paragraph{Relativistic phonon nuclear
coupling}\label{relativistic-phonon-nuclear-coupling-1}

Using the most optimistic Dicke enhancement, for relativistic phonon
nuclear coupling (Eq. \(\ref{eq:phononcoupling}\)) we would have: \[
2\sqrt{2}\sqrt{N} \sqrt{\frac{\Delta E}{M c^2}}  \times 10^{-3} \ge 1
\]

Using the same numbers as before in Eq.
\(\ref{eq:criticalphononcouplingexplicitnumbers}\) then:

\[
2\sqrt{2} \sqrt{N}\sqrt{\frac{24\times10^6}{10^{11}}}  \times 10^{-3} \approx 4 \times 10^{-5} \sqrt{N} \ge 1
\]

This gives us a condition on the number of nuclei that we need:

\[
N \gtrsim 8 \times 10^8
\]

If we were instead to consider a different transition, e.g.~the
\(\rm 14 \, keV\) transition of \(\rm ^{57}Fe\) then:

\[
2\sqrt{2} \sqrt{N}\sqrt{\frac{14\times10^3}{5\times 10^{10}}}  \times 10^{-3} \approx 1.5 \times 10^{-6} \sqrt{N} \ge 1
\]

Which gives us the following condition on the required number of nuclei:

\[
N \gtrsim 4 \times 10^{11}.
\]

Both the \(\rm Pd\) and \(\rm ^{57}Fe\) numbers are well within
practical limits given that solid number density is about
\(5\times 10^{28} \, \rm m^{-3}\)

\paragraph{Electric dipole coupling}\label{electric-dipole-coupling-1}

Using the most optimistic Dicke enhancement, for electric dipole
coupling (Eq. \(\ref{eq:dipolecoupling}\)) we have (using the same
numbers as in Eq. \(\ref{eq:dipolewithnumbers}\)):

\[
3\times 10^{-18}\sqrt{N}  \ge 1
\]

Which gives us the following condition on the required number of nuclei
in this case:

\[
N \gtrsim 10^{35}
\]

This is well outside of what is practical.

\printbibliography


\end{document}
