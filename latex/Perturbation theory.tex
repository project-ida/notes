% Options for packages loaded elsewhere
\PassOptionsToPackage{unicode}{hyperref}
\PassOptionsToPackage{hyphens}{url}
\documentclass[
]{article}
\usepackage{xcolor}
\usepackage{amsmath,amssymb}

\usepackage{hyperref}
\usepackage{ulem}  % For underlining links
\normalem

\hypersetup{
    colorlinks=false,         % Disable colored links (boxes will appear)
    citebordercolor=red,      % Border color for citation links
    linkbordercolor=red,      % Border color for internal links
    urlbordercolor=blue,      % Border color for external links (URLs)
    pdfborder={0 0 2}         % Specifies the border thickness: {horizontal vertical thickness}
}


% Only underline external links (URLs)
\let\oldhref\href
\renewcommand{\href}[2]{\ifx#1\urlprefix\oldhref{#1}{#2}\else\uline{\oldhref{#1}{#2}}\fi}


\renewcommand{\[}{\begin{equation}}
\renewcommand{\]}{\end{equation}}


\setcounter{secnumdepth}{5}
\usepackage{iftex}
\ifPDFTeX
  \usepackage[T1]{fontenc}
  \usepackage[utf8]{inputenc}
  \usepackage{textcomp} % provide euro and other symbols
\else % if luatex or xetex
  \usepackage{unicode-math} % this also loads fontspec
  \defaultfontfeatures{Scale=MatchLowercase}
  \defaultfontfeatures[\rmfamily]{Ligatures=TeX,Scale=1}
\fi
\usepackage{lmodern}
\ifPDFTeX\else
  % xetex/luatex font selection
\fi
% Use upquote if available, for straight quotes in verbatim environments
\IfFileExists{upquote.sty}{\usepackage{upquote}}{}
\IfFileExists{microtype.sty}{% use microtype if available
  \usepackage[]{microtype}
  \UseMicrotypeSet[protrusion]{basicmath} % disable protrusion for tt fonts
}{}
\makeatletter
\@ifundefined{KOMAClassName}{% if non-KOMA class
  \IfFileExists{parskip.sty}{%
    \usepackage{parskip}
  }{% else
    \setlength{\parindent}{0pt}
    \setlength{\parskip}{6pt plus 2pt minus 1pt}}
}{% if KOMA class
  \KOMAoptions{parskip=half}}
\makeatother
\setlength{\emergencystretch}{3em} % prevent overfull lines
\providecommand{\tightlist}{%
  \setlength{\itemsep}{0pt}\setlength{\parskip}{0pt}}
\usepackage[]{biblatex}
\addbibresource{refs.bib}
\usepackage{bookmark}
\IfFileExists{xurl.sty}{\usepackage{xurl}}{} % add URL line breaks if available
\urlstyle{same}

\title{Perturbation theory}
  \date{\today}  % Default to today if no date is provided

\begin{document}
\maketitle

\section{Notes on Perturbation Theory for a Two-Ensemble Dicke
Model}\label{notes-on-perturbation-theory-for-a-two-ensemble-dicke-model}

\emph{(Donors + Acceptors coupled to a single quantised mode)}

These notes build up the machinery we need to do
\textbf{effective-Hamiltonian perturbation theory} properly, including
the appearance of \textbf{folded-diagram (derivative) terms at 4th
order}. We keep the discussion general first, then later we will plug in
our specific Dicke Hamiltonian and basis.

\begin{center}\rule{0.5\linewidth}{0.5pt}\end{center}

\subsection{1) The model}\label{the-model}

We consider a Hamiltonian split into ``bare'' and ``interaction'' parts:

\(H = H_0 + V\).

A convenient Dicke-like choice is:

\begin{itemize}
\tightlist
\item
  Single bosonic mode: \(a, a^\dagger\) with frequency \(\omega_c\).
\item
  Two ensembles of two-level systems (TLS): donors \(D\) and acceptors
  \(A\).
\item
  Collective spin operators (Dicke operators) for each ensemble:

  \begin{itemize}
  \tightlist
  \item
    \(S_D^\pm, S_D^z\) for donors, size \(N_D\).
  \item
    \(S_A^\pm, S_A^z\) for acceptors, size \(N_A\).
  \end{itemize}
\end{itemize}

A standard ``Rabi/Dicke'' style Hamiltonian is:

\[
H_0
=
\omega_c a^\dagger a
+
\omega_0 \big(S_D^z + S_A^z\big)
\]

\[
V
=
(a+a^\dagger)
\Big[
v_D (S_D^+ + S_D^-)
+
v_A (S_A^+ + S_A^-)
\Big].
\]

Here \(v_D\) and \(v_A\) are the single-particle couplings for donors
and acceptors (not assumed equal).

\begin{center}\rule{0.5\linewidth}{0.5pt}\end{center}

\subsection{2) Choice of basis and the P/Q
split}\label{choice-of-basis-and-the-pq-split}

\subsubsection{Dicke basis (conceptual)}\label{dicke-basis-conceptual}

A convenient label for basis states is:

\(|N_D^*, N_A^*, n\rangle\),

where: - \(N_D^*\) is the number of donor excitations in the symmetric
Dicke ladder (\(0,1,\dots,N_D\)), - \(N_A^*\) is the number of acceptor
excitations (\(0,1,\dots,N_A\)), - \(n\) is the photon number.

The bare energies take the schematic form:

\[
E^{(0)}(N_D^*,N_A^*,n)
=
\omega_0 (N_D^*+N_A^*)
+
\omega_c n
\quad
\text{(up to constant offsets)}.
\]

\subsubsection{The two-dimensional P
space}\label{the-two-dimensional-p-space}

We choose a ``model space'' \(P\) consisting of the two
single-excitation states at a chosen photon sector (often \(k=0\),
i.e.~photon number fixed to some reference \(n\)):

\begin{itemize}
\tightlist
\item
  Donor-like state:
\end{itemize}

\(|D\rangle \equiv |1,0,n\rangle\)

\begin{itemize}
\tightlist
\item
  Acceptor-like state:
\end{itemize}

\(|A\rangle \equiv |0,1,n\rangle\)

So \(P\) is 2D and is spanned by \(\{|D\rangle,|A\rangle\}\). Let \(Q\)
be the complement:

\(Q = 1 - P\).

We will write operator blocks like: - \(H_{PP} \equiv PHP\) -
\(H_{PQ} \equiv PHQ\) - \(H_{QP} \equiv QHP\) - \(H_{QQ} \equiv QHQ\)

and similarly for \(H_0\) and \(V\).

\begin{center}\rule{0.5\linewidth}{0.5pt}\end{center}

\subsection{3) Exact elimination of Q space: Bloch--Horowitz effective
Hamiltonian}\label{exact-elimination-of-q-space-blochhorowitz-effective-hamiltonian}

Start from the Schrödinger equation:

\(H|\Psi\rangle = E|\Psi\rangle\).

Decompose the state:

\(|\Psi\rangle = |\psi_P\rangle + |\psi_Q\rangle\),

where \(|\psi_P\rangle = P|\Psi\rangle\) and
\(|\psi_Q\rangle = Q|\Psi\rangle\).

Projecting onto \(P\) and \(Q\) gives the coupled block equations:

\[
(H_{PP} - E)\,|\psi_P\rangle + H_{PQ}\,|\psi_Q\rangle = 0
\]

\[
H_{QP}\,|\psi_P\rangle + (H_{QQ}-E)\,|\psi_Q\rangle = 0.
\]

Assuming \((E-H_{QQ})\) is invertible on \(Q\), solve the second
equation:

\[
|\psi_Q\rangle
=
(E-H_{QQ})^{-1} H_{QP}\,|\psi_P\rangle.
\]

Insert into the \(P\) equation to obtain the \textbf{exact
energy-dependent effective Hamiltonian}:

\[
H_{\mathrm{eff}}(E)
=
H_{PP}
+
H_{PQ}\,(E-H_{QQ})^{-1}\,H_{QP}.
\]

This is often called the \textbf{Bloch--Horowitz (BH)} or
\textbf{Feshbach} effective Hamiltonian.

\textbf{Key point:} \(H_{\mathrm{eff}}(E)\) depends on \(E\), so the
eigenvalue problem is nonlinear:

\[
H_{\mathrm{eff}}(E)\,|\psi_P\rangle
=
E\,|\psi_P\rangle.
\]

Solving this self-consistently reproduces the exact full-space
eigenvalues associated with the chosen \(P\) sector.

\begin{center}\rule{0.5\linewidth}{0.5pt}\end{center}

\subsection{4) Perturbative expansion: resolvent and ``unfolded''
series}\label{perturbative-expansion-resolvent-and-unfolded-series}

We now expand in powers of \(V\).

Write:

\(H = H_0 + V\),

and correspondingly:

\(H_{QQ} = (H_0)_{QQ} + V_{QQ}\), etc.

Define an ``unperturbed'' \(Q\)-space resolvent:

\[
G_0(E)
\equiv
(E-(H_0)_{QQ})^{-1}.
\]

Then we expand the full resolvent:

\[
(E-H_{QQ})^{-1}
=
\big(E-(H_0)_{QQ}-V_{QQ}\big)^{-1}
=
G_0(E) + G_0(E)V_{QQ}G_0(E) + G_0(E)V_{QQ}G_0(E)V_{QQ}G_0(E)+\cdots
\]

Insert into BH:

\[
H_{\mathrm{eff}}(E)
=
H_{PP}
+
V_{PQ}G_0(E)V_{QP}
+
V_{PQ}G_0(E)V_{QQ}G_0(E)V_{QP}
+
V_{PQ}G_0(E)V_{QQ}G_0(E)V_{QQ}G_0(E)V_{QP}
+\cdots
\]

This produces an expansion in powers of \(V\):

\begin{itemize}
\tightlist
\item
  2nd order: \(V_{PQ}G_0(E)V_{QP}\)
\item
  3rd order: \(V_{PQ}G_0(E)V_{QQ}G_0(E)V_{QP}\)
\item
  4th order: \(V_{PQ}G_0(E)V_{QQ}G_0(E)V_{QQ}G_0(E)V_{QP}\)
\item
  etc.
\end{itemize}

If we evaluate these terms at a fixed reference energy \(E=E_0\)
(typically the degenerate bare energy of the \(P\) states), we get what
we will call the \textbf{unfolded} contributions (no intermediate return
to \(P\) inside the resolvent expansion).

\begin{center}\rule{0.5\linewidth}{0.5pt}\end{center}

\subsection{5) Why energy dependence matters: self-consistency vs
energy-independent
expansions}\label{why-energy-dependence-matters-self-consistency-vs-energy-independent-expansions}

\subsubsection{Self-consistent (BH)
viewpoint}\label{self-consistent-bh-viewpoint}

In BH, you compute \(H_{\mathrm{eff}}(E)\) and solve the nonlinear
problem

\(H_{\mathrm{eff}}(E)\psi_P = E\psi_P\).

This automatically includes ``folding'' effects because you never expand
away the \(E\) dependence: it is treated exactly (within the chosen
truncation of \(Q\)).

\subsubsection{Energy-independent perturbation theory
viewpoint}\label{energy-independent-perturbation-theory-viewpoint}

Often we want an \textbf{energy-independent} effective Hamiltonian
expanded order-by-order (Rayleigh--Schrödinger style). Then we must
handle the fact that \(H_{\mathrm{eff}}\) depends on \(E\).

This is where \textbf{folded-diagram (derivative) terms} appear.

\begin{center}\rule{0.5\linewidth}{0.5pt}\end{center}

\subsection{6) The origin of folded terms: Taylor expanding the
self-energy}\label{the-origin-of-folded-terms-taylor-expanding-the-self-energy}

Define the BH self-energy operator:

\(\Sigma(E) \equiv H_{PQ}(E-H_{QQ})^{-1}H_{QP}\).

Then:

\(H_{\mathrm{eff}}(E) = H_{PP} + \Sigma(E)\).

Choose a reference energy \(E_0\) (for our degenerate \(P\) space this
is the common bare energy of \(|D\rangle\) and \(|A\rangle\)), and
write:

\(E = E_0 + \delta E\).

Now Taylor expand:

\[
\Sigma(E)
=
\Sigma(E_0)
+
\delta E\,\Sigma'(E_0)
+
\frac{(\delta E)^2}{2}\Sigma''(E_0)
+
\cdots
\]

Crucial power counting: - \(\Sigma(E_0)\) starts at order \(V^2\). -
Typically \(\delta E\) is also order \(V^2\) (it is an energy shift
generated by the interaction).

Therefore the term

\(\delta E\,\Sigma'(E_0)\)

is of order \(V^4\).

That is exactly the \textbf{leading folded contribution at 4th order}.

\subsubsection{Why derivatives appear}\label{why-derivatives-appear}

The derivative comes from differentiating the resolvent:

\[
\frac{d}{dE}(E-H_{QQ})^{-1}
=
(E-H_{QQ})^{-2}.
\]

So folded terms correspond to squared denominators in the
intermediate-state sums. Diagrammatically, they are associated with
intermediate returns to the model space \(P\) that would otherwise
produce zero denominators in naive time-ordered perturbation theory.

\begin{center}\rule{0.5\linewidth}{0.5pt}\end{center}

\subsection{7) What ``folded'' means when P is
two-dimensional}\label{what-folded-means-when-p-is-two-dimensional}

Here \(P\) contains two states, \(|D\rangle\) and \(|A\rangle\).

A process can ``return to P'' in the middle by landing on
\textbf{either} state (with the correct photon sector).

\begin{itemize}
\tightlist
\item
  Returning early to \(|D\rangle\) corresponds to ``dressing the donor
  leg''.
\item
  Returning early to \(|A\rangle\) corresponds to ``dressing the
  acceptor leg''.
\end{itemize}

This is why, in a 2D \(P\) space, the folded correction is naturally a
\textbf{matrix product} structure (schematically):

\[
H_{\mathrm{eff}}^{(\le 4)}
\approx
H_{PP}
+
\Sigma^{(2)}(E_0)
+
\Sigma^{(4)}_{\mathrm{unfolded}}(E_0)
+
\Sigma^{(2)\prime}(E_0)\,\Sigma^{(2)}(E_0)
\quad
\text{(up to convention-dependent symmetrisation)}.
\]

When we look specifically at the \textbf{off-diagonal element} \(AD\),
the folded term contains contributions corresponding to ``hit \(D\)
early'' and ``hit \(A\) early'':

\begin{itemize}
\tightlist
\item
  ``hit \(D\) early'' dressing contributes via the \(P\)-index \(p=D\)
\item
  ``hit \(A\) early'' dressing contributes via the \(P\)-index \(p=A\)
\end{itemize}

This is exactly how ``getting to the end point early'' enters
analytically: it is one of the allowed intermediate \(P\) states in the
folded correction.

\begin{center}\rule{0.5\linewidth}{0.5pt}\end{center}

\subsection{8) Stop point for the general
formalism}\label{stop-point-for-the-general-formalism}

At this stage we have:

\begin{enumerate}
\def\labelenumi{\arabic{enumi}.}
\tightlist
\item
  A clear \(P/Q\) reduction to an exact \(H_{\mathrm{eff}}(E)\)
  (BH/Feshbach).
\item
  A perturbative expansion of \(H_{\mathrm{eff}}(E)\) in powers of \(V\)
  that yields ``unfolded'' contributions at fixed \(E_0\).
\item
  A Taylor expansion in \(E\) showing that, at 4th order, we must
  include derivative terms (folded diagrams) because \(\delta E\) is
  itself order \(V^2\).
\end{enumerate}

Next step (in our Dicke model): - Choose the explicit Dicke basis states
and compute \(\Sigma^{(2)}_{DD}(E_0)\), \(\Sigma^{(2)}_{AA}(E_0)\),
\(\Sigma^{(2)}_{AD}(E_0)\) keeping full \((n,n+1)\) dependence. - Then
compute \(\Sigma^{(2)\prime}(E_0)\) and assemble the 4th-order folded
correction. - Compare ``static at \(E_0\)'' vs ``self-consistent BH''
and show the cancellations we observed numerically.

\begin{center}\rule{0.5\linewidth}{0.5pt}\end{center}

\printbibliography


\end{document}
