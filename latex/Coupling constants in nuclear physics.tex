% Options for packages loaded elsewhere
\PassOptionsToPackage{unicode}{hyperref}
\PassOptionsToPackage{hyphens}{url}
\documentclass[
]{article}
\usepackage{xcolor}
\usepackage{amsmath,amssymb}

\usepackage{hyperref}
\usepackage{ulem}  % For underlining links
\normalem

\hypersetup{
    colorlinks=false,         % Disable colored links (boxes will appear)
    citebordercolor=red,      % Border color for citation links
    linkbordercolor=red,      % Border color for internal links
    urlbordercolor=blue,      % Border color for external links (URLs)
    pdfborder={0 0 2}         % Specifies the border thickness: {horizontal vertical thickness}
}


% Only underline external links (URLs)
\let\oldhref\href
\renewcommand{\href}[2]{\ifx#1\urlprefix\oldhref{#1}{#2}\else\uline{\oldhref{#1}{#2}}\fi}


\renewcommand{\[}{\begin{equation}}
\renewcommand{\]}{\end{equation}}


\setcounter{secnumdepth}{5}
\usepackage{iftex}
\ifPDFTeX
  \usepackage[T1]{fontenc}
  \usepackage[utf8]{inputenc}
  \usepackage{textcomp} % provide euro and other symbols
\else % if luatex or xetex
  \usepackage{unicode-math} % this also loads fontspec
  \defaultfontfeatures{Scale=MatchLowercase}
  \defaultfontfeatures[\rmfamily]{Ligatures=TeX,Scale=1}
\fi
\usepackage{lmodern}
\ifPDFTeX\else
  % xetex/luatex font selection
\fi
% Use upquote if available, for straight quotes in verbatim environments
\IfFileExists{upquote.sty}{\usepackage{upquote}}{}
\IfFileExists{microtype.sty}{% use microtype if available
  \usepackage[]{microtype}
  \UseMicrotypeSet[protrusion]{basicmath} % disable protrusion for tt fonts
}{}
\makeatletter
\@ifundefined{KOMAClassName}{% if non-KOMA class
  \IfFileExists{parskip.sty}{%
    \usepackage{parskip}
  }{% else
    \setlength{\parindent}{0pt}
    \setlength{\parskip}{6pt plus 2pt minus 1pt}}
}{% if KOMA class
  \KOMAoptions{parskip=half}}
\makeatother
\setlength{\emergencystretch}{3em} % prevent overfull lines
\providecommand{\tightlist}{%
  \setlength{\itemsep}{0pt}\setlength{\parskip}{0pt}}
\usepackage[]{biblatex}
\addbibresource{refs.bib}
\usepackage{bookmark}
\IfFileExists{xurl.sty}{\usepackage{xurl}}{} % add URL line breaks if available
\urlstyle{same}

\title{Coupling constants in nuclear physics}
  \author{Matt Lilley}
  \date{\today}  % Default to today if no date is provided

\begin{document}
\maketitle

\section{Introduction}\label{introduction}

Understanding the interaction between nucleons and external fields is
essential in nuclear physics. We'll explore two coupling mechanisms that
arise in quantum nuclear interactions:

\begin{enumerate}
\def\labelenumi{\arabic{enumi}.}
\tightlist
\item
  \textbf{Relativistic phonon nuclear coupling (\(a \cdot cp\))} --
  where phonons couple to nucleons through momentum exchange (see
  \href{https://iopscience.iop.org/article/10.1088/1361-6455/acf3be}{Hagelstein
  2023} for more detail).
\item
  \textbf{Electric dipole coupling (\(d \cdot E\))} -- where an electric
  field couples to nucleons through electric dipole moments.
\item
  \textbf{Magnetic dipole coupling (\(\mu \cdot B\))} - where a magnetic
  field couples to nucleonics through magnetic dipole moments.
\end{enumerate}

This document explores these couplings, derives their respective
coupling constants, and compares their strength.

\section{\texorpdfstring{Relativistic phonon nuclear coupling
(\(a \cdot cp\))}{Relativistic phonon nuclear coupling (a \textbackslash cdot cp)}}\label{relativistic-phonon-nuclear-coupling-a-cdot-cp}

\subsection{\texorpdfstring{\(p\) in
\(a \cdot cp\)}{p in a \textbackslash cdot cp}}\label{p-in-a-cdot-cp}

For a nucleus of mass \(M\) moving within a solid, the momentum-based
coupling energy scales as:

\[
E \sim p c 
\]

From kinetic energy considerations:

\[
\frac{p^2}{2M} = E \quad \Rightarrow \quad p = \sqrt{2ME}
\]

If \(E\) represents phonon oscillations:

\[
E = n \hbar \omega_A 
\]

where \(n\) is the phonon occupation number. Distributing this energy
over \(N\) atoms:

\[
p = \sqrt{\frac{2M}{N}} \sqrt{\hbar \omega_A} \sqrt{n} \label{eq:p}
\]

Note that later we'll drop the \(\sqrt{n}\) because it should be picked
up in the Hamiltonian operator \((b^{\dagger} + b)\).

\subsection{\texorpdfstring{\(a\) in
\(a \cdot cp\)}{a in a \textbackslash cdot cp}}\label{a-in-a-cdot-cp}

From Eq. 470 in section 6.11 of
\href{https://arxiv.org/pdf/2501.08338}{Models for nuclear fusion in the
solid state}, the \(a\) component of \(a \cdot cp\) can be approximated
as:

\[
a \sim \frac{1}{2} \frac{\Delta E}{M c^2} \frac{\pi}{m c}
\]

where:

\begin{itemize}
\tightlist
\item
  \(\Delta E\) is the nuclear transition energy,
\item
  \(m\) is the mass of a single nucleon within the nucleus,
\item
  \(\pi\) represents the relative momentum of that nucleon.
\end{itemize}

Since angular momentum is approximately \(\hbar\), and using the Fermi
scale \(l_F = 10^{-15}\) m, we estimate: \[
\pi \sim \frac{\hbar}{l_F} 
\]

Substituting this into the expression for \(a\) gives:

\[
a \sim \frac{1}{2} \frac{\Delta E}{M c^2} \frac{\bar\lambda_c}{l_F}
\]

where \(\bar\lambda_c = \hbar / m c\) is the reduced Compton wavelength,
approximately:

\[
\bar\lambda_c \approx 2 \times 10^{-16} \text{ m}.
\]

To account for hindrance effects in nuclear transitions, we introduce a
suppression factor \(O\), where \(O \sim 0.01\). This modifies the
expression to:

\[
a \sim \frac{1}{2} \frac{\Delta E}{M c^2} \frac{\bar\lambda_c}{l_F} O 
\]

which simplifies to:

\[
a \sim \frac{\Delta E}{M c^2} \times 10^{-3} 
\label{eq:a}
\]

The final value of \(a\) depends on both the nuclear transition type and
the specific nucleus under consideration.

\subsection{Overall coupling constant}\label{overall-coupling-constant}

Let's consider a single TLS interacts with a single phonon mode. The
Hamiltonian can be written as:

\[
H = \frac{\Delta E}{2} \sigma_z + \hbar\omega_A\left(b^{\dagger}b +\frac{1}{2}\right) + U\left( b^{\dagger} + b \right)\sigma_x
\]

where \(\Delta E\) is the transition energy between the 2 levels of the
TLS, \(\hbar\omega_A\) is the energy of each quantum of the field, and
\(U\) is the coupling constant between the TLS and the field. The
\(\sigma\) operators are the Pauli matrices and \(b^{\dagger}\), \(b\)
are the field creation and annihilation operators respectively. Note
that usually \(a\) is used for the field operators, but in these notes
we use \(b\) to avoid confusion with \(a \cdot cp\) .

The \(a \cdot cp\) coupling constant \(U\) can be defined by combining
Eq. \(\ref{eq:p}\) (without the \(\sqrt{n}\)) with Eq. \(\ref{eq:a}\):
\[
U = c \sqrt{\frac{2M}{N}} \sqrt{\hbar \omega_A} \times \frac{\Delta E}{M c^2} \times 10^{-3}
\]

Normalising \(U\) to \(\hbar \omega_A\) gives:

\[
\frac{U}{\hbar \omega_A} = \sqrt{\frac{2M c^2}{N}} \frac{1}{\sqrt{\hbar \omega_A}} \times \frac{\Delta E}{M c^2} \times 10^{-3}
\]

Rearranging and simplifying leads to:

\[
\frac{U}{\hbar \omega_A} = \sqrt{\frac{2}{N}} \sqrt{\frac{\Delta E}{M c^2}} \sqrt{\frac{\Delta E}{\hbar \omega_A}} \times 10^{-3}
\]

which can also be written as:

\[
\frac{U}{\hbar \omega_A} = \sqrt{\frac{2}{N}} \sqrt{\frac{\hbar \omega_A}{M c^2}} \frac{\Delta E}{\hbar \omega_A} \times 10^{-3}
\]

\subsection{Example for palladium with acoustic
phonons}\label{example-for-palladium-with-acoustic-phonons}

\begin{itemize}
\tightlist
\item
  \(\Delta E \approx 24 \times 10^{6}\) eV\\
\item
  \(M c^2 \approx 10^{11}\) eV\\
\item
  \(\hbar \omega_A \approx 10^{-8}\) eV\\
\item
  \(N \approx 10^{18}\)
\end{itemize}

\[
\frac{U}{\hbar \omega_A} \approx \sqrt{\frac{2}{10^{18}}} \sqrt{\frac{24 \times 10^6}{10^{11}}} \sqrt{\frac{24 \times 10^6}{10^{-8}}} \times 10^{-3}
\]

\[
\approx 10^{-6}
\]

\subsection{Dicke enhancement}\label{dicke-enhancement}

For an ensemble of \(N\) nuclei interacting collectively with a phonon
field, coupling is enhanced by \(\sqrt{N}\), leading to:

\[
\frac{U}{\hbar \omega_A} \sim 10^{3}
\]

Based on this, we can be far into the ``deep strong coupling'' regime
where \(U/\hbar \omega_A > 1\).

\section{Electric dipole coupling (E1
transitions)}\label{electric-dipole-coupling-e1-transitions}

\subsection{\texorpdfstring{\(E\) in
\(d \cdot E\)}{E in d \textbackslash cdot E}}\label{e-in-d-cdot-e}

Electric field strength due to phonons follows from force relations:

\[
F = \frac{dp}{dt} = ZeE
\]

For oscillatory motion:

\[
\frac{dp}{dt} \sim \omega_A p \Rightarrow E = \frac{\omega_A p}{Ze}
\]

Substituting our previous result for \(p\):

\[
E = \frac{\omega_A \sqrt{2M \hbar \omega_A n}}{Ze \sqrt{N}} \label{eq:E}
\]

\subsection{\texorpdfstring{\(d\) in
\(d \cdot E\)}{d in d \textbackslash cdot E}}\label{d-in-d-cdot-e}

We can connect two key expressions related to electric dipole
interactions:

\begin{enumerate}
\def\labelenumi{\arabic{enumi}.}
\tightlist
\item
  Radiation from an electric dipole -- describes how an oscillating
  electric dipole emits radiation.\\
\item
  Radiative decay rates from Weisskopf -- provides an estimate for
  transition rates.
\end{enumerate}

\subsubsection{Radiation from an electric
dipole}\label{radiation-from-an-electric-dipole}

The radiative decay rate due to dipole radiation is given by: \[
\gamma_{\text{rad}} = \frac{4}{3} \frac{1}{4 \pi \epsilon_0 \hbar} \frac{\omega^3}{c^3} d^2
\]

Rewriting in terms of the fine-structure constant \(\alpha\):

\[
\gamma_{\text{rad}} = \frac{4}{3} \frac{1}{e^2} \alpha \frac{\omega^3}{c^2} d^2
\]

where the fine-structure constant is defined as:

\[
\alpha = \frac{e^2}{4\pi \epsilon_0 \hbar c}
\]

\subsubsection{Weisskopf estimate for E1
transition}\label{weisskopf-estimate-for-e1-transition}

Weisskopf's formula for radiative decay is given by
(\href{http://www.dommelen.net/quantum2/style_a/nt_weis.html\#SECTION091258000000000000000}{Eq.
A.190 of Dommelen's book}):

\[
\gamma_{\text{rad}} = \frac{2 (L+1)}{L \left[(2L+1)!!\right]^2} \alpha (kR)^{2L} \omega \left( \frac{3}{L+3} \right)^2
\]

where:

\begin{itemize}
\tightlist
\item
  \(L\) is the multipolarity (\(L=1\) for dipole, \(L=2\) for
  quadrupole).
\item
  \(k\) is the wavenumber of the emitted radiation.
\item
  \(R\) is the nuclear radius, given by:
\end{itemize}

\[
R = R_0 A^{1/3}
\]

where \(R_0\) is the radius of a single nucleon and \(A\) is the number
of nucleons. Note that there exist other forms of Weisskopf's formula
that are
\href{http://www.dommelen.net/quantum2/style_a/ntgd.html\#SECTION086204000000000000000}{more
convenient for numerical evaluation} but they obscure the physical
constants.

For a dipole transition (\(L=1\)), this simplifies to:

\[
\gamma_{\text{rad}} = \frac{2 \times 2}{1 \times [3!!]^2} \alpha \frac{\omega^2}{c^2} R_0^2 A^{2/3} \omega \left(\frac{3}{4} \right)^2
\]

Rewriting more compactly:

\[
\gamma_{\text{rad}} = \frac{9}{4\times(3!!)^2} \alpha \frac{\omega^3}{c^2} R_0^2 A^{2/3}
\]

\subsubsection{Equating the two
expressions}\label{equating-the-two-expressions}

From the previous derivations, we equate:

\[
\frac{9}{4\times(3!!)^2} \alpha \frac{\omega^3}{c^2} R_0^2 A^{2/3} = \frac{4}{3} \frac{1}{e^2} \alpha \frac{\omega^3}{c^2} d^2
\]

Rearranging:

\[
\frac{27}{16 \times (3!!)^2} A^{2/3} e^2 R_0^2 = d^2
\]

Taking the square root:

\[
d = \frac{\sqrt{27}}{4 \times (3!!)} A^{1/3} e R_0
\]

which simplifies to:

\[
d = \frac{\sqrt{27}}{2880} A^{1/3} e R_0
\]

Approximating numerically:

\[
d \approx 2 \times 10^{-3} A^{1/3} e R_0 \label{eq:d}
\]

\subsection{Overall coupling
constant}\label{overall-coupling-constant-1}

If we again use this simple Hamiltonian in which a single TLS interacts
with a single phonon mode:

\[
H = \frac{\Delta E}{2} \sigma_z + \hbar\omega_A\left(a^{\dagger}a +\frac{1}{2}\right) + U\left( b^{\dagger} + b \right)\sigma_x
\]

then a \(d \cdot E\) coupling constant \(U\) can be defined by combining
Eq. \(\ref{eq:E}\) (without the \(\sqrt{n}\)) with Eq. \(\ref{eq:d}\):

\[
\frac{U}{\hbar \omega_A} = \frac{1}{\hbar \omega_A} \frac{\omega_A \sqrt{2M \hbar \omega_A}}{Ze \sqrt{N}} \times 2 \times 10^{-3} A^{1/3} e R_0
\]

Rearranging,

\[
\frac{U}{\hbar \omega_A} = \frac{\sqrt{2}}{Z \sqrt{N}} \sqrt{\frac{M c^2}{\hbar \omega_A}} \frac{\hbar \omega_A R_0}{\hbar c} A^{1/3} \times 2 \times 10^{-3}
\]

We recognize \(\hbar c / R_0\) as the localization energy of a nucleon,
which we call \(E_L\). Thus, we obtain:

\[
\frac{U}{\hbar \omega_A} = \frac{\sqrt{2}}{Z \sqrt{N}} \sqrt{\frac{M c^2}{\hbar \omega_A}} \frac{\hbar \omega_A}{E_L} A^{1/3} \times 2 \times 10^{-3}
\]

which can also be written as:

\[
\frac{U}{\hbar \omega_A} = \frac{\sqrt{2}}{Z \sqrt{N}} \sqrt{\frac{M c^2}{E_L}} \sqrt{\frac{\hbar \omega_A}{E_L}} A^{1/3} \times 2 \times 10^{-3}
\]

Note how the expressions for \(a \cdot cp\) and \(d \cdot E\) have an
interesting reciprocal relationship if we see that \(E_L\) plays the
role of \(\Delta E\).

\subsection{Example of Pd with Acoustic
Phonons}\label{example-of-pd-with-acoustic-phonons}

Given:

\begin{itemize}
\tightlist
\item
  \(A \approx 106\)
\item
  \(N \approx 10^{18}\)
\item
  \(Z \approx 46\)
\item
  \(M c^2 \approx 10^{11}\) eV
\item
  \(\hbar \omega_A \approx 10^{-8}\) eV
\end{itemize}

First, let's calculate the localization energy:

\[
E_L = \frac{\hbar c}{R_0} = \frac{6.6 \times 10^{-34} \times 3 \times 10^8}{10^{-15}}
\]

\[
= 2 \times 10^{-10} \text{ J} = 1.2 \times 10^9 \text{ eV} \approx 10^9 \text{ eV}
\]

Now, substituting these numbers gives:

\[
\frac{U}{\hbar \omega_A} \approx \frac{\sqrt{2} }{106 \times 10^9} \times \sqrt{\frac{10^{11}}{10^{-8}}} \times \frac{10^{-8}}{10^9} \times 2 \times 10^{-3} \times 106^{1/3}
\]

Approximating:

\[
\approx \frac{\sqrt{2} \sqrt{10}}{46 \times 10^9} \times 10^9 \times 10^{-17} \times 2 \times 10^{-3} \times 106^{1/3}
\]

\[
\approx 0.1 \times 2 \times 10^{-20} \times 106^{1/3}
\]

\[
\approx 10^{-20}
\]

\subsection{Dicke enhancement}\label{dicke-enhancement-1}

For an ensemble of \(N\) nuclei interacting collectively with a phonon
field, coupling is enhanced by \(\sqrt{N}\), leading to:

\[
\frac{U}{\hbar \omega_A} \sim 10^{-12}
\]

and so even with Dicke enhancement, dipole coupling remains in the weak
coupling regime.

\section{Magnetic dipole coupling (M1
transitions)}\label{magnetic-dipole-coupling-m1-transitions}

\subsection{\texorpdfstring{\(B\) in
\(\mu\cdot B\)}{B in \textbackslash mu\textbackslash cdot B}}\label{b-in-mucdot-b}

We assume there is an externally driven oscillatory magnetic field \(B\)
with frequency \(\omega\) in some volume V. Since field energy density
\(\sim \mu_0B^2\) then: \[
\frac{1}{\mu_0} B^2 V = n\hbar\omega
\] where \(n\) is the field occupation number.

We can therefore write: \[
B = \sqrt{\frac{\mu_0n\hbar\omega}{V}}
\label{eq:B}
\]

\subsection{\texorpdfstring{\(\mu\) in
\(\mu \cdot B\)}{\textbackslash mu in \textbackslash mu \textbackslash cdot B}}\label{mu-in-mu-cdot-b}

In order to calculate the dipole moment \(\mu\) associated with the
\(\mu\cdot B\) coupling, we'll pursue a similar analysis as we did for
E1 transitions, namely:

We can connect two key expressions related to magnetic dipole
interactions:

\begin{enumerate}
\def\labelenumi{\arabic{enumi}.}
\tightlist
\item
  Radiation from a magnetic dipole -- describes how an oscillating
  magnetic dipole emits radiation.
\item
  Radiative decay rates from Weisskopf -- provides an estimate for
  transition rates.
\end{enumerate}

\subsubsection{Radiation from a magnetic
dipole}\label{radiation-from-a-magnetic-dipole}

The radiative decay rate due to dipole radiation is given by:

\[
\gamma_{\text{rad}} = \frac{\mu_0}{12 \pi \hbar} \frac{\omega^3}{c^3} \mu^2
\]

Rewriting in terms of the fine-structure constant \(\alpha\):

\[
\gamma_{\text{rad}} = \frac{\alpha \omega^3}{3 e^2 c^4} \mu^2
\]

where the fine-structure constant is defined as:

\[
\alpha = \frac{e^2}{4\pi \epsilon_0 \hbar c}
\]

\subsubsection{Weisskopf estimate for M1
transition}\label{weisskopf-estimate-for-m1-transition}

Weisskopf's formula for radiative decay is given by
(\href{http://www.dommelen.net/quantum2/style_a/nt_weis.html\#SECTION091258000000000000000}{Eq.
A.192 of Dommelen's book}):

\[
\gamma_{\text{rad}}  =  10\frac{2(L+1)}{L [(2L+1)!!]^2} \alpha (kR)^{2L} \omega \left( \frac{3}{l+3} \right)^2 \left( \frac{\hbar}{m_p c R} \right)^2
\]

where:

\begin{itemize}
\tightlist
\item
  \(L\) is the multipolarity (\(L=1\) for dipole, \(L=2\) for
  quadrupole).
\item
  \(k\) is the wavenumber of the emitted radiation.
\item
  \(m_p\) is the proton mass
\item
  \(R\) is the nuclear radius, given by:
\end{itemize}

\[
R = R_0 A^{1/3}
\]

where \(R_0\) is the radius of a single nucleon and \(A\) is the number
of nucleons. Note that there exist other forms of Weisskopf's formula
that are
\href{http://www.dommelen.net/quantum2/style_a/ntgd.html\#SECTION086204000000000000000}{more
convenient for numerical evaluation} but they obscure the physical
constants.

The last term can be related to the reduced Compton wavelength
\(\bar\lambda_c = \hbar / m c\):

\[
\gamma_{\text{rad}}  =  10\frac{2(L+1)}{L [(2L+1)!!]^2} \alpha (kR)^{2L} \omega \left( \frac{3}{l+3} \right)^2 \left( \frac{\bar\lambda_c}{R} \right)^2
\]

It's instructive to compare the radiation rate for a magnetic dipole vs
electric dipole:

\[
\gamma_{\rm rad, B} = \gamma_{\rm rad, E} \times 10\left( \frac{\bar\lambda_c}{R} \right)^2
\]

Given that \(R_0 \sim 10^{-15} \ \rm m\) and
\(\bar\lambda_c \approx 2 \times 10^{-16} \text{ m}\) then:

\[
\gamma_{\rm rad, B} = \gamma_{\rm rad, E} \times 2.5\left( \frac{1}{A} \right)^{2/3}
\]

For \(A\approx100\), \(\gamma_{\rm rad, B} = 0.1\gamma_{\rm rad, E}\).

For a dipole transition (\(L=1\)), Weisskopf's formula simplifies to:

\[
\gamma_{\text{rad}} = \frac{20 \times 2}{1 \times [3!!]^2} \alpha \frac{\omega^2}{c^2} R^2 \omega \left(\frac{3}{4} \right)^2 \left( \frac{\bar\lambda_c}{R} \right)^2
\]

Rewriting more compactly:

\[
\gamma_{\text{rad}} = \frac{20}{(3!!)^2} \alpha \frac{\omega^3}{c^2} \bar\lambda_c^2
\]

\subsubsection{Equating the two
expressions}\label{equating-the-two-expressions-1}

\[
\frac{20}{(3!!)^2} \alpha \frac{\omega^3}{c^2} \bar\lambda_c^2 = \frac{\alpha \omega^3}{3 e^2 c^4} \mu^2
\]

\[
\frac{20}{(720)^2} \left( \frac{\hbar}{m_p c} \right)^2 3e^2c^2 = \mu^2
\]

\[
\frac{60}{(720)^2} \left( \frac{e\hbar}{m_p} \right)^2= \mu^2
\]

\[
\frac{\sqrt{60}}{720} \frac{e\hbar}{m_p} = \mu
\]

And so \[
\mu \approx 0.02 \mu_N
\label{eq:mu}
\] Where \(\mu_N = e\hbar/m_p \approx 5\times 10^{-27} \ \rm J/T\) is
the nuclear magneton.

\subsection{Overall coupling
constant}\label{overall-coupling-constant-2}

If we again use this simple Hamiltonian in which a single TLS interacts
with a single mode but this time it's not a phonon mode but a magnon
mode:

\[
H = \frac{\Delta E}{2} \sigma_z + \hbar\omega_A\left(a^{\dagger}a +\frac{1}{2}\right) + U\left( b^{\dagger} + b \right)\sigma_x
\]

then a \(\mu \cdot B\) coupling constant \(U\) can be defined by simply
multiplying Eq. \(\ref{eq:mu}\) by Eq. \(\ref{eq:B}\) (without the
\(\sqrt{n}\)) :

\[
U \approx 0.02 \frac{{\mu_N}B}{\sqrt{n}}
\]

\[
U \approx 0.02 {\mu_N}\sqrt{\frac{\mu_0\hbar\omega}{V}}
\]

\subsection{Example with low frequency
solenoid}\label{example-with-low-frequency-solenoid}

For \(1 \ \rm MHz\) field oscillations (\(\sim 4 \ \rm neV\)), and using
a volume \(V = 0.001 \ \rm m^{-3}\) \[
\begin{aligned}
U &\approx 0.02 \times 5\times 10^{-27}\times \sqrt{\frac{4\pi\times 10^{-7}\times 4\times 10^{-9}\times 1.6\times 10^{-19}}{0.001}} \\
\frac{U}{\hbar\omega} &\approx 0.02 \times 5\times 10^{-27}\times\sqrt{\frac{4\pi\times 10^{-7} }{0.001 \times 4\times 10^{-9}\times 1.6\times 10^{-19}}} \\
\frac{U}{\hbar\omega} &\approx 2.8\times10^{-16}
\end{aligned}
\]

\printbibliography


\end{document}
