% Options for packages loaded elsewhere
\PassOptionsToPackage{unicode}{hyperref}
\PassOptionsToPackage{hyphens}{url}
\documentclass[
]{article}
\usepackage{xcolor}
\usepackage{amsmath,amssymb}

\usepackage{hyperref}
\usepackage{ulem}  % For underlining links

\hypersetup{
    colorlinks=false,         % Disable colored links (boxes will appear)
    citebordercolor=red,      % Border color for citation links
    linkbordercolor=red,      % Border color for internal links
    urlbordercolor=blue,      % Border color for external links (URLs)
    pdfborder={0 0 2}         % Specifies the border thickness: {horizontal vertical thickness}
}


% Only underline external links (URLs)
\let\oldhref\href
\renewcommand{\href}[2]{\ifx#1\urlprefix\oldhref{#1}{#2}\else\uline{\oldhref{#1}{#2}}\fi}


\renewcommand{\[}{\begin{equation}}
\renewcommand{\]}{\end{equation}}


\setcounter{secnumdepth}{5}
\usepackage{iftex}
\ifPDFTeX
  \usepackage[T1]{fontenc}
  \usepackage[utf8]{inputenc}
  \usepackage{textcomp} % provide euro and other symbols
\else % if luatex or xetex
  \usepackage{unicode-math} % this also loads fontspec
  \defaultfontfeatures{Scale=MatchLowercase}
  \defaultfontfeatures[\rmfamily]{Ligatures=TeX,Scale=1}
\fi
\usepackage{lmodern}
\ifPDFTeX\else
  % xetex/luatex font selection
\fi
% Use upquote if available, for straight quotes in verbatim environments
\IfFileExists{upquote.sty}{\usepackage{upquote}}{}
\IfFileExists{microtype.sty}{% use microtype if available
  \usepackage[]{microtype}
  \UseMicrotypeSet[protrusion]{basicmath} % disable protrusion for tt fonts
}{}
\makeatletter
\@ifundefined{KOMAClassName}{% if non-KOMA class
  \IfFileExists{parskip.sty}{%
    \usepackage{parskip}
  }{% else
    \setlength{\parindent}{0pt}
    \setlength{\parskip}{6pt plus 2pt minus 1pt}}
}{% if KOMA class
  \KOMAoptions{parskip=half}}
\makeatother
\setlength{\emergencystretch}{3em} % prevent overfull lines
\providecommand{\tightlist}{%
  \setlength{\itemsep}{0pt}\setlength{\parskip}{0pt}}
\usepackage[]{biblatex}
\addbibresource{refs.bib}
\usepackage{bookmark}
\IfFileExists{xurl.sty}{\usepackage{xurl}}{} % add URL line breaks if available
\urlstyle{same}

\title{README}
  \author{Ida}
  \date{\today}  % Default to today if no date is provided

\begin{document}
\maketitle

\section{Notes}\label{notes}

A collection of unfinished notes on various topics related to
nucleonics.

You write notes in
\href{https://www.markdownguide.org/basic-syntax/}{markdown format},
including things like equations, references etc, and then behind the
scenes there is a Github Action that will convert these markdown files
into both PDF and Latex (see folders with those names). The advantage is
that:

\begin{itemize}
\tightlist
\item
  Markdown is web friendly
\item
  Latex is journal friendly
\item
  PDF is email friendly
\end{itemize}

We'll always have formats for every use case.

Below is a little demo of some of the essential markdown features. Check
out the source code to see how human readable it all is.

\section{Section}\label{section}

\subsection{Subsection}\label{subsection}

\subsubsection{Subsubsection}\label{subsubsection}

This is some \textbf{bold text}.

This is a list:

\begin{itemize}
\tightlist
\item
  Markdown
\item
  is
\item
  awesome
\end{itemize}

A numbered list:

\begin{enumerate}
\def\labelenumi{\arabic{enumi}.}
\tightlist
\item
  It's
\item
  so
\item
  Easy
\end{enumerate}

This is an inline equation \(E=mc^2\).

This is a display equation:

\[
E = mc^2
\label{eq:emc2}
\]

This equation has a label that can be referenced like this Eq.
\(\ref{eq:emc2}\). In Github, the equation numbers don't show when
rendering markdown and so the Eq. reference will look broken. However,
the Latex and PDF versions do have numbers and equation references work
as you'd expect.

We can write a bibliographic reference like this
\autocite{einstein1905}. It doesn't render here in markdown but will in
the PDF and Latex versions. The linking is done via \texttt{refs.bib}
that's founds in the \texttt{pandoc} folder.

Hyperlinks are a piece of cake, e.g.~here are more
\href{https://www.markdownguide.org/basic-syntax/}{examples of what you
can do in markdown}.

\printbibliography


\end{document}
