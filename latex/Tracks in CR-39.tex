% Options for packages loaded elsewhere
\PassOptionsToPackage{unicode}{hyperref}
\PassOptionsToPackage{hyphens}{url}
\documentclass[
]{article}
\usepackage{xcolor}
\usepackage{amsmath,amssymb}

\usepackage{hyperref}
\usepackage{ulem}  % For underlining links

\hypersetup{
    colorlinks=false,         % Disable colored links (boxes will appear)
    citebordercolor=red,      % Border color for citation links
    linkbordercolor=red,      % Border color for internal links
    urlbordercolor=blue,      % Border color for external links (URLs)
    pdfborder={0 0 2}         % Specifies the border thickness: {horizontal vertical thickness}
}


% Only underline external links (URLs)
\let\oldhref\href
\renewcommand{\href}[2]{\ifx#1\urlprefix\oldhref{#1}{#2}\else\uline{\oldhref{#1}{#2}}\fi}


\renewcommand{\[}{\begin{equation}}
\renewcommand{\]}{\end{equation}}


\setcounter{secnumdepth}{5}
\usepackage{iftex}
\ifPDFTeX
  \usepackage[T1]{fontenc}
  \usepackage[utf8]{inputenc}
  \usepackage{textcomp} % provide euro and other symbols
\else % if luatex or xetex
  \usepackage{unicode-math} % this also loads fontspec
  \defaultfontfeatures{Scale=MatchLowercase}
  \defaultfontfeatures[\rmfamily]{Ligatures=TeX,Scale=1}
\fi
\usepackage{lmodern}
\ifPDFTeX\else
  % xetex/luatex font selection
\fi
% Use upquote if available, for straight quotes in verbatim environments
\IfFileExists{upquote.sty}{\usepackage{upquote}}{}
\IfFileExists{microtype.sty}{% use microtype if available
  \usepackage[]{microtype}
  \UseMicrotypeSet[protrusion]{basicmath} % disable protrusion for tt fonts
}{}
\makeatletter
\@ifundefined{KOMAClassName}{% if non-KOMA class
  \IfFileExists{parskip.sty}{%
    \usepackage{parskip}
  }{% else
    \setlength{\parindent}{0pt}
    \setlength{\parskip}{6pt plus 2pt minus 1pt}}
}{% if KOMA class
  \KOMAoptions{parskip=half}}
\makeatother
\setlength{\emergencystretch}{3em} % prevent overfull lines
\providecommand{\tightlist}{%
  \setlength{\itemsep}{0pt}\setlength{\parskip}{0pt}}
\usepackage[]{biblatex}
\addbibresource{refs.bib}
\usepackage{bookmark}
\IfFileExists{xurl.sty}{\usepackage{xurl}}{} % add URL line breaks if available
\urlstyle{same}

\title{Tracks in CR-39}
  \author{Matt Lilley}
  \date{\today}  % Default to today if no date is provided

\begin{document}
\maketitle

\section{What is CR-39}\label{what-is-cr-39}

CR-39 (Columbia Resin \#39) is a lightweight, impact-resistant plastic
polymer. It is a thermosetting plastic derived from allyl diglycol
carbonate (ADC) monomers.

When the ADC monomer (\(\rm C_{12} H_{18} O_7\)) is polymerised, the
CR-39 structure contains allyl groups (\(\rm-CH_2-CH=CH_2\)) and
carbonate (\(\rm -COO-\)) functional groups.

\section{Radiation detection}\label{radiation-detection}

\subsection{Charged particles}\label{charged-particles}

When high-energy charged particles (e.g., alpha particles, protons, or
heavy ions) pass through CR-39, they break chemical bonds along their
path, creating latent damage tracks in the polymer. These tracks are
invisible initially but weaken the polymer structure at those points.
The tracks can be made visible through an etching process.

When the exposed CR-39 is submerged in a chemical etchant (usually 6M
NaOH or KOH at 60--70°C), the etching solution dissolves the damaged
regions faster than the undamaged areas, making the tracks visible under
a microscope. The resulting track diameter and shape provide information
about the energy and type of radiation.

\subsection{Neutrons}\label{neutrons}

Although not charged, neutrons can be detected by CR-39 through an
indirect mechanism.

Proton Recoil (Elastic Scattering) - most common

Secondary Reactions with Carbon or Oxygen - less common

Fission fragments with contaminants

\paragraph{}\label{section}

\printbibliography


\end{document}
