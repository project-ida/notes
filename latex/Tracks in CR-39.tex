% Options for packages loaded elsewhere
\PassOptionsToPackage{unicode}{hyperref}
\PassOptionsToPackage{hyphens}{url}
\documentclass[
]{article}
\usepackage{xcolor}
\usepackage{amsmath,amssymb}

\usepackage{hyperref}
\usepackage{ulem}  % For underlining links
\normalem

\hypersetup{
    colorlinks=false,         % Disable colored links (boxes will appear)
    citebordercolor=red,      % Border color for citation links
    linkbordercolor=red,      % Border color for internal links
    urlbordercolor=blue,      % Border color for external links (URLs)
    pdfborder={0 0 2}         % Specifies the border thickness: {horizontal vertical thickness}
}


% Only underline external links (URLs)
\let\oldhref\href
\renewcommand{\href}[2]{\ifx#1\urlprefix\oldhref{#1}{#2}\else\uline{\oldhref{#1}{#2}}\fi}


\renewcommand{\[}{\begin{equation}}
\renewcommand{\]}{\end{equation}}


\setcounter{secnumdepth}{5}
\usepackage{iftex}
\ifPDFTeX
  \usepackage[T1]{fontenc}
  \usepackage[utf8]{inputenc}
  \usepackage{textcomp} % provide euro and other symbols
\else % if luatex or xetex
  \usepackage{unicode-math} % this also loads fontspec
  \defaultfontfeatures{Scale=MatchLowercase}
  \defaultfontfeatures[\rmfamily]{Ligatures=TeX,Scale=1}
\fi
\usepackage{lmodern}
\ifPDFTeX\else
  % xetex/luatex font selection
\fi
% Use upquote if available, for straight quotes in verbatim environments
\IfFileExists{upquote.sty}{\usepackage{upquote}}{}
\IfFileExists{microtype.sty}{% use microtype if available
  \usepackage[]{microtype}
  \UseMicrotypeSet[protrusion]{basicmath} % disable protrusion for tt fonts
}{}
\makeatletter
\@ifundefined{KOMAClassName}{% if non-KOMA class
  \IfFileExists{parskip.sty}{%
    \usepackage{parskip}
  }{% else
    \setlength{\parindent}{0pt}
    \setlength{\parskip}{6pt plus 2pt minus 1pt}}
}{% if KOMA class
  \KOMAoptions{parskip=half}}
\makeatother
\setlength{\emergencystretch}{3em} % prevent overfull lines
\providecommand{\tightlist}{%
  \setlength{\itemsep}{0pt}\setlength{\parskip}{0pt}}
\usepackage[]{biblatex}
\addbibresource{refs.bib}
\usepackage{bookmark}
\IfFileExists{xurl.sty}{\usepackage{xurl}}{} % add URL line breaks if available
\urlstyle{same}

\title{Tracks in CR-39}
  \author{Matt Lilley}
  \date{\today}  % Default to today if no date is provided

\begin{document}
\maketitle

\section{What is CR-39}\label{what-is-cr-39}

CR-39 (Columbia Resin \#39) is a lightweight, impact-resistant plastic
polymer. It is a thermosetting plastic derived from allyl diglycol
carbonate (ADC) monomers.

When the ADC monomer (\(\rm C_{12} H_{18} O_7\)) is polymerised, the
CR-39 structure contains allyl groups (\(\rm-CH_2-CH=CH_2\)) and
carbonate (\(\rm -COO-\)) functional groups.

\section{Radiation detection}\label{radiation-detection}

\subsection{Charged particles}\label{charged-particles}

When high-energy charged particles (e.g., alpha particles, protons, or
heavy ions) pass through CR-39, they break chemical bonds along their
path, creating latent damage tracks in the polymer. These tracks are
invisible initially but weaken the polymer structure at those points.
The tracks can be made visible through an etching process.

When the exposed CR-39 is submerged in a chemical etchant (usually 6M
NaOH or KOH at 60--70°C), the etching solution dissolves the damaged
regions faster than the undamaged areas, making the tracks visible under
a microscope. The resulting track diameter and shape provide information
about the energy and type of radiation.

\subsection{Neutrons}\label{neutrons}

Unlike charged particles such as alpha particles or protons, neutrons
are electrically neutral and do not directly ionize or break chemical
bonds as they pass through CR-39. However, they can still be detected
indirectly through interactions with the atomic nuclei in the CR-39
polymer or any added materials. Two primary mechanisms enable neutron
detection in CR-39: \textbf{proton recoil via elastic scattering} (the
most common method) and \textbf{secondary reactions with carbon or
oxygen} (less common). Additionally, in some cases, fission fragments
from neutron-induced reactions with contaminants or external converters
can contribute to detection.

\subsubsection{Proton Recoil (Elastic
Scattering)}\label{proton-recoil-elastic-scattering}

The proton recoil method relies on elastic scattering between incident
neutrons and hydrogen nuclei (protons) within the CR-39 polymer. CR-39
contains a significant amount of hydrogen due to its chemical
composition (\(\rm C_{12} H_{18} O_7\)), making this mechanism
effective, particularly for fast neutrons.

\begin{itemize}
\tightlist
\item
  \textbf{Process}: When a fast neutron collides with a hydrogen
  nucleus, it transfers some of its kinetic energy to the proton. The
  recoiling proton, being a charged particle, then travels through the
  CR-39, breaking chemical bonds and creating a latent damage track
  along its path---similar to tracks produced by other charged
  particles.
\item
  \textbf{Detection}: These latent tracks are revealed through the same
  chemical etching process (e.g., submersion in 6M NaOH or KOH at
  60--70°C) used for charged particles. The size and shape of the etched
  tracks depend on the energy of the recoiling proton, which correlates
  with the energy of the incident neutron.
\item
  \textbf{Advantages}: This method is highly effective for detecting
  fast neutrons (typically above \textasciitilde100 keV), as the energy
  transfer efficiency in elastic scattering is maximized when the target
  nucleus (hydrogen) has a mass similar to the neutron.
\item
  \textbf{Limitations}: Thermal neutrons (low-energy neutrons,
  \textasciitilde0.025 eV) are less likely to produce detectable recoil
  protons due to their low kinetic energy, making this method less
  sensitive to them without additional techniques.
\end{itemize}

\subsubsection{Secondary Reactions with Carbon or
Oxygen}\label{secondary-reactions-with-carbon-or-oxygen}

Neutrons can also interact with the carbon (\(\rm C\)) or oxygen
(\(\rm O\)) nuclei in the CR-39 polymer, though these reactions are less
frequent than proton recoil due to lower interaction cross-sections and
higher energy thresholds.

\begin{itemize}
\tightlist
\item
  \textbf{Process}: When a high-energy neutron (typically in the MeV
  range) collides with a carbon or oxygen nucleus, it may induce nuclear
  reactions such as (\(\rm n, p\)) or (\(\rm n, \alpha\)). For example:

  \begin{itemize}
  \tightlist
  \item
    \(\rm ^{12}C(n, p)^{12}B\): A neutron knocks out a proton, leaving a
    boron isotope.
  \item
    \(\rm ^{16}O(n, \alpha)^{13}C\): A neutron triggers the emission of
    an alpha particle, producing a carbon isotope.
  \end{itemize}
\item
  \textbf{Detection}: The emitted charged particles (protons or alpha
  particles) create latent tracks in the CR-39, which are then etched
  and analyzed microscopically. The track characteristics can provide
  clues about the reaction type and neutron energy.
\item
  \textbf{Significance}: These reactions are more relevant for very
  high-energy neutrons but contribute less to overall detection compared
  to proton recoil, since the abundance of hydrogen in CR-39 and the
  higher cross-section for neutron scattering with hydrogen make proton
  recoil more likely, while nuclear reactions with carbon or oxygen are
  rarer and require more energy. \#\#\# Fission Fragments with
  Contaminants
\end{itemize}

In some specialized applications, CR-39 can detect neutrons by
incorporating external materials or contaminants that undergo
neutron-induced fission.

\begin{itemize}
\tightlist
\item
  \textbf{Process}: Materials such as uranium-235, thorium-232, or
  boron-10 can be placed in contact with or embedded in the CR-39. When
  neutrons (especially thermal neutrons) interact with these nuclei,
  they may trigger fission (in uranium or thorium) or reactions like
  \(\rm ^{10}B(n, \alpha)^{7}Li\). The resulting fission fragments or
  alpha particles, being highly energetic charged particles, produce
  distinct damage tracks in the CR-39.
\item
  \textbf{Detection}: After etching, these tracks are typically larger
  and more pronounced than those from proton recoil, allowing
  differentiation of neutron-induced fission events.
\item
  \textbf{Applications}: This method is particularly useful for
  detecting thermal neutrons, which are inefficiently detected via
  proton recoil. External converter layers (e.g., boron or uranium
  coatings) are often used to enhance sensitivity in neutron dosimetry
  or environmental monitoring.
\end{itemize}

\subsubsection{Practical Considerations}\label{practical-considerations}

\begin{itemize}
\tightlist
\item
  \textbf{Energy Sensitivity}: The effectiveness of neutron detection in
  CR-39 depends on the neutron energy spectrum. Fast neutrons are best
  detected via proton recoil, while thermal neutrons require converters
  or fissionable materials.
\item
  \textbf{Etching Conditions}: The etching process must be optimized to
  distinguish neutron-induced tracks from background noise or tracks
  caused by other radiation types.
\item
  \textbf{Analysis}: Microscopic analysis of track density, size, and
  shape, often combined with calibration against known neutron sources,
  allows researchers to quantify neutron flux and energy.
\end{itemize}

\printbibliography


\end{document}
