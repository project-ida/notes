% Options for packages loaded elsewhere
\PassOptionsToPackage{unicode}{hyperref}
\PassOptionsToPackage{hyphens}{url}
\documentclass[
]{article}
\usepackage{xcolor}
\usepackage{amsmath,amssymb}

\usepackage{hyperref}
\usepackage{ulem}  % For underlining links

\hypersetup{
    colorlinks=false,         % Disable colored links (boxes will appear)
    citebordercolor=red,      % Border color for citation links
    linkbordercolor=red,      % Border color for internal links
    urlbordercolor=blue,      % Border color for external links (URLs)
    pdfborder={0 0 2}         % Specifies the border thickness: {horizontal vertical thickness}
}


% Only underline external links (URLs)
\let\oldhref\href
\renewcommand{\href}[2]{\ifx#1\urlprefix\oldhref{#1}{#2}\else\uline{\oldhref{#1}{#2}}\fi}


\renewcommand{\[}{\begin{equation}}
\renewcommand{\]}{\end{equation}}


\setcounter{secnumdepth}{5}
\usepackage{iftex}
\ifPDFTeX
  \usepackage[T1]{fontenc}
  \usepackage[utf8]{inputenc}
  \usepackage{textcomp} % provide euro and other symbols
\else % if luatex or xetex
  \usepackage{unicode-math} % this also loads fontspec
  \defaultfontfeatures{Scale=MatchLowercase}
  \defaultfontfeatures[\rmfamily]{Ligatures=TeX,Scale=1}
\fi
\usepackage{lmodern}
\ifPDFTeX\else
  % xetex/luatex font selection
\fi
% Use upquote if available, for straight quotes in verbatim environments
\IfFileExists{upquote.sty}{\usepackage{upquote}}{}
\IfFileExists{microtype.sty}{% use microtype if available
  \usepackage[]{microtype}
  \UseMicrotypeSet[protrusion]{basicmath} % disable protrusion for tt fonts
}{}
\makeatletter
\@ifundefined{KOMAClassName}{% if non-KOMA class
  \IfFileExists{parskip.sty}{%
    \usepackage{parskip}
  }{% else
    \setlength{\parindent}{0pt}
    \setlength{\parskip}{6pt plus 2pt minus 1pt}}
}{% if KOMA class
  \KOMAoptions{parskip=half}}
\makeatother
\setlength{\emergencystretch}{3em} % prevent overfull lines
\providecommand{\tightlist}{%
  \setlength{\itemsep}{0pt}\setlength{\parskip}{0pt}}
\usepackage[]{biblatex}
\addbibresource{refs.bib}
\usepackage{bookmark}
\IfFileExists{xurl.sty}{\usepackage{xurl}}{} % add URL line breaks if available
\urlstyle{same}

\title{Dicke model}
  \author{Matt Lilley}
  \date{\today}  % Default to today if no date is provided

\begin{document}
\maketitle

\section{Introduction}\label{introduction}

The purpose of this document is to describe the Dicke model.

\section{Rabi model}\label{rabi-model}

Before we can talk about the Dicke model, we must first familiarise
ourselves with a single two level system (TLS) interacting with a single
mode (i.e.~single frequency/wavelength) of a quantised field. This is
often called the Rabi model and its Hamiltonian can be written as:

\[
H_{\rm Rabi} = \frac{\Delta E}{2} \sigma_z + \hbar\omega\left(a^{\dagger}a +\frac{1}{2}\right) + U\left( a^{\dagger} + a \right)(\sigma_+ + \sigma_-)
\label{eq:rabiH}
\]

where \(\Delta E\) is the transition energy between the 2 levels of the
TLS, \(\hbar\omega\) is the energy of each quantum of the field, and
\(U\) is the coupling constant between the TLS and the field. The
\(\sigma\) operators are the Pauli spin matrices and \(a^{\dagger}\),
\(a\) are the field creation and annihilation operators respectively.

It's worth noting that we're using the Pauli spin matrices as a
mathematical tool to describe two levels. Just keep in mind that we're
not really talking about spin angular momentum here.

The TLS has just 2 states denoted by \(|\pm\rangle\) but the quantised
field has infinitely many states denoted by the number of quanta
\(|n\rangle\). The combined state of the system can therefore be
represented by \(|n, \pm\rangle\). Each quantum state can be
conceptually thought of as a pendulum whose frequency is related to the
energy of the state (see
e.g.~\href{https://journals.aps.org/pra/abstract/10.1103/PhysRevA.85.052111}{Briggs
et.al}).

The strength of the interaction between the TLS and the field is not
only determined by the constants \(U\) but also how many field quanta
\(n\) we have. This is because of how the field operators work: \[
a^{\dagger} |n,\pm\rangle = \sqrt{n+1}|n+1,\pm\rangle
\label{eq:fieldcreate}
\]

\[
a |n,\pm\rangle = \sqrt{n}|n-1,\pm\rangle \\
\label{eq:fielddestroy}
\]

\[
a^{\dagger}a |n,\pm\rangle = n|n,\pm\rangle
\]

The more field quanta we have (the stronger the field), the larger the
coupling terms will be compared to the TLS term which remains unchanged.

\section{Dicke model}\label{dicke-model}

The Dicke model describes a system where we have \(N\) identical TLS
coupled to a single mode (i.e.~single frequency/wavelength) of a
quantised field. The Dicke Hamiltonians is a simple extension of the
Rabi Hamiltonian in Eq. \(\ref{eq:rabiH}\) in the sense that we just add
\(N\) copies of the TLS terms as seen below:

\[
H_{\text{Dicke}} = \frac{\Delta E}{2} \sum_{i=1}^N \sigma_z^{(i)} +  \hbar\omega\left(a^{\dagger}a +\frac{1}{2}\right) + U \sum_{i=1}^N (a^\dagger + a) (\sigma_+^{(i)} + \sigma_-^{(i)})
\label{eq:dickeH}
\]

The states of this system are described by
\(|n, \pm, \pm, \pm, \pm, ... \rangle\).

It's worth emphasising that there is no spatial dependence of the field
in Eq. \(\ref{eq:dickeH}\). One way to understand this physically is
that all the TLS are very close together in the sense that they are
located in a region of space that is much smaller than the wavelength of
the mode. In that situation, all the TLS will experience the same
strength of field at any given moment - in other words the field appears
constant in space. This is the how Dicke originally presented his ideas
in \href{https://journals.aps.org/pr/abstract/10.1103/PhysRev.93.99}{his
1954 paper}. It's also possible to use this Hamiltonian to describe many
TLS arranged in a very special way so that they are placed at integer
multiples of the mode wavelength.

Much like how the strength of the field changes the size of the coupling
term in the Rabi model (and also in the Dicke model), the number of TLS
also has an effect - this is the origin of what's called Dicke
superradiance. It is however difficult to see this right now because Eq.
\(\ref{eq:dickeH}\) is not written in a convenient form. We're going to
need to rewrite it by appealing to the physics of spin.

\subsection{Pseudo-spin}\label{pseudo-spin}

We noted earlier that the use of the Pauli spin matrices is just a
mathematical tool to describe two levels. Although we're not describing
spin here, we are working with what is often called ``pseudo spin''. The
\href{https://en.wikipedia.org/wiki/Angular_momentum_operator}{rules of
angular momentum} work just as well for pseudo angular momentum. In
particular, the rules of
\href{https://www.tcm.phy.cam.ac.uk/~bds10/aqp/lec7_compressed.pdf}{angular
momentum addition} and conservation.

This means that we can treat all the TLS together as if they are a
single object with many levels whose energies are determined by the
addition rules of pseudo angular momentum. The Hamilton then looks like:

\[
H =  \Delta E J_{z} + \hbar\omega\left(a^{\dagger}a +\frac{1}{2}\right) + 2U\left( a^{\dagger} + a \right)(J_{+} + J_{-})
\label{eq:dickeHpseudo}
\]

where the total pseudo total angular momentum operators (\(J\)) for
\(N\) TLS are:

\[
J_{+} + J_{-} = J_{x} = \frac{1}{2}\overset{N}{\underset{i=1}{\Sigma}} \sigma_{i x} \,\,\,\,\,\, J_{z} = \frac{1}{2}\overset{N}{\underset{i=1}{\Sigma}} \sigma_{i z}
\]

and noting that \(i\) in \(\sigma_i\) means that this operator only acts
on TLS number \(i\) .

When written in this way, each state can now be described in terms of 3
numbers \(|n, j, m\rangle\) where \(j\) describes the total pseudo
angular momentum number (which is conserved) and \(m\) describes the z
component of the total pseudo angular momentum (which can change). This
notation allows us to conveniently describe situations where excitations
are ``delocalised'' among the TLS. By far the most significant kind of
delocalised states are called ``Dicke states'' which have the largest
\(j=j_{\max} = N/2\). Dicke states are symmetric in the sense that if
you swap any of the TLS around, the state remains unchanged. For
example, consider a single excitation in 4 TLS - the Dicke state looks
like: \[
\Psi_0 = \frac{1}{\sqrt{4}}\left(| 0, +, -, -, - \rangle + | 0, -, +, -, - \rangle + | 0, -, -, +, - \rangle + | 0, -, -, -, + \rangle \right)
\]

Notice that if you swap any two TLS, the state looks the same.

The above state can instead be described by \(j_{\max}= 4/2  = 2\) and
\(m = 1\times 1/2 + 3\times -1/2 =-1\)

\[
\Psi_0 = |0,2,-1>
\]

In this way of describing the system, the
\href{https://en.wikipedia.org/wiki/Ladder_operator\#Angular_momentum}{ladder
operators} \(J_{+}\) and \(J_{-}\) create and destroy excitations of the
TLS. This causes a raising and lowering of the \(m\) value like this: \[
J_+ |n, j, m\rangle  =  \sqrt{j(j + 1) - m(m + 1)} |n, j, m + 1\rangle
\]

\[
J_- |n, j, m\rangle =  \sqrt{j(j + 1) - m(m - 1)} |n, j, m - 1\rangle
\]

These ladder operators are conceptually similar to the creation and
annihilation operators of the field (see Eqs. \(\ref{eq:fieldcreate}\)
and \(\ref{eq:fielddestroy}\)). The details are however more complicated
due to the addition rules of angular momentum. Despite the complexity,
we know that the maximum total angular momentum \(j_{\max} = N/2\) (from
\(N\) TLS with pseudo angular momentum \(1/2\)). We can therefore see
that the number of TLS is going to have an effect on the coupling
between TLS and field. Exactly what effect depends on the details. Let's
explore some of those details next.

\subsection{Superradiance}\label{superradiance}

Dicke states with \(j=j_{\max}\) are significant because of the
acceleration properties that they offer; something people often describe
as superradiance.

For the case with \textbf{all TLS excited} , \(m=N/2\): \[
J_- |n, N/2, N/2\rangle  = \sqrt{N} |n, N/2, N/2 - 1\rangle
\]

For the first de-excitation, the coupling terms gets enhanced by
\(\sqrt{N}\) . This might not seem surprising at first glance because we
have \(N\) TLS excited and so we should expect the rate of emission
(which go as the square of the coupling) to be enhanced by \(N\).

For the case of \textbf{a single excitation} , \(m=-N/2 + 1\):

\[
J_- |n, N/2, -N/2 +1\rangle  = \sqrt{N}|n, N/2, -N/2\rangle
\]

For this singe de-excitation the coupling term also gets enhanced by
\(\sqrt{N}\) . This is more surprising because the rate of emission
(which go as the square of the coupling) to be enhanced by \(N\) even
though there is only a single excitation.

For the case of \textbf{50\% excitation}, \(m=0\):

\[
J_- |n, N/2, 0 \rangle  = \sqrt{N^2 + N}|n, N/2, -1\rangle
\]

For the first de-excitation, the coupling terms gets enhanced by
\(\sim\sqrt{N^2}\) for large \(N\). In other words, the rate of emission
(which go as the square of the coupling) to be enhanced by \(N^2\) -
this is where the super in superradiance comes from.

\printbibliography


\end{document}
