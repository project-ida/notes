% Options for packages loaded elsewhere
\PassOptionsToPackage{unicode}{hyperref}
\PassOptionsToPackage{hyphens}{url}
\documentclass[
]{article}
\usepackage{xcolor}
\usepackage{amsmath,amssymb}

\usepackage{hyperref}
\usepackage{ulem}  % For underlining links
\normalem

\hypersetup{
    colorlinks=false,         % Disable colored links (boxes will appear)
    citebordercolor=red,      % Border color for citation links
    linkbordercolor=red,      % Border color for internal links
    urlbordercolor=blue,      % Border color for external links (URLs)
    pdfborder={0 0 2}         % Specifies the border thickness: {horizontal vertical thickness}
}


% Only underline external links (URLs)
\let\oldhref\href
\renewcommand{\href}[2]{\ifx#1\urlprefix\oldhref{#1}{#2}\else\uline{\oldhref{#1}{#2}}\fi}


\renewcommand{\[}{\begin{equation}}
\renewcommand{\]}{\end{equation}}


\setcounter{secnumdepth}{5}
\usepackage{iftex}
\ifPDFTeX
  \usepackage[T1]{fontenc}
  \usepackage[utf8]{inputenc}
  \usepackage{textcomp} % provide euro and other symbols
\else % if luatex or xetex
  \usepackage{unicode-math} % this also loads fontspec
  \defaultfontfeatures{Scale=MatchLowercase}
  \defaultfontfeatures[\rmfamily]{Ligatures=TeX,Scale=1}
\fi
\usepackage{lmodern}
\ifPDFTeX\else
  % xetex/luatex font selection
\fi
% Use upquote if available, for straight quotes in verbatim environments
\IfFileExists{upquote.sty}{\usepackage{upquote}}{}
\IfFileExists{microtype.sty}{% use microtype if available
  \usepackage[]{microtype}
  \UseMicrotypeSet[protrusion]{basicmath} % disable protrusion for tt fonts
}{}
\makeatletter
\@ifundefined{KOMAClassName}{% if non-KOMA class
  \IfFileExists{parskip.sty}{%
    \usepackage{parskip}
  }{% else
    \setlength{\parindent}{0pt}
    \setlength{\parskip}{6pt plus 2pt minus 1pt}}
}{% if KOMA class
  \KOMAoptions{parskip=half}}
\makeatother
\setlength{\emergencystretch}{3em} % prevent overfull lines
\providecommand{\tightlist}{%
  \setlength{\itemsep}{0pt}\setlength{\parskip}{0pt}}
\usepackage[]{biblatex}
\addbibresource{refs.bib}
\usepackage{bookmark}
\IfFileExists{xurl.sty}{\usepackage{xurl}}{} % add URL line breaks if available
\urlstyle{same}

\title{Dicke model}
  \author{Matt Lilley}
  \date{\today}  % Default to today if no date is provided

\begin{document}
\maketitle

\section{Introduction}\label{introduction}

The purpose of this document is to describe the Dicke model.

\section{Rabi model}\label{rabi-model}

Before we can talk about the Dicke model, we must first familiarise
ourselves with a single two level system (TLS) interacting with a single
mode (i.e.~single frequency/wavelength) of a quantised field. This is
often called the Rabi model and its Hamiltonian can be written as:

\[
H_{\rm Rabi} = \frac{\Delta E}{2} \sigma_z + \hbar\omega\left(a^{\dagger}a +\frac{1}{2}\right) + U\left( a^{\dagger} + a \right)(\sigma_+ + \sigma_-)
\label{eq:rabiH}
\]

where \(\Delta E\) is the transition energy between the 2 levels of the
TLS, \(\hbar\omega\) is the energy of each quantum of the field, and
\(U\) is the coupling constant between the TLS and the field. The
\(\sigma\) operators are the
\href{https://ocw.mit.edu/courses/5-61-physical-chemistry-fall-2007/3b1fb40c61e7f939861b190bedbc57a7_lecture24.pdf}{Pauli
spin matrices} that act on the TLS, where \(\sigma_+\) and \(\sigma_-\)
act as raising and lowering operators. The \(a^{\dagger}\), \(a\) are
the field creation and annihilation operators respectively.

It's worth noting that we're using the Pauli spin matrices as a
mathematical tool to describe two levels. Just keep in mind that we're
not really talking about spin angular momentum here.

The TLS has just 2 states denoted by \(|\pm\rangle\) but the quantised
field has infinitely many states denoted by the number of quanta
\(|n\rangle\). The combined state of the system can therefore be
represented by \(|n, \pm\rangle\). Each quantum state can be
conceptually thought of as a pendulum whose frequency is related to the
energy of the state (see
e.g.~\href{https://journals.aps.org/pra/abstract/10.1103/PhysRevA.85.052111}{Briggs
et.al}).

The strength of the interaction between the TLS and the field is not
only determined by the constant \(U\) but also how many field quanta
\(n\) we have. This is because of how the field operators work:

\[
a^{\dagger} |n,\pm\rangle = \sqrt{n+1}|n+1,\pm\rangle
\label{eq:fieldcreate}
\]

\[
a |n,\pm\rangle = \sqrt{n}|n-1,\pm\rangle \\
\label{eq:fielddestroy}
\]

\[
a^{\dagger}a |n,\pm\rangle = n|n,\pm\rangle
\]

The more field quanta we have (the stronger the field), the larger the
coupling terms will be.

\section{Dicke model}\label{dicke-model}

The Dicke model describes a system where we have \(N\) identical TLS
coupled to a single mode (i.e.~single frequency/wavelength) of a
quantised field. The Dicke Hamiltonians is a simple extension of the
Rabi Hamiltonian in Eq. \(\ref{eq:rabiH}\) in the sense that we just add
\(N\) copies of the TLS terms as seen below:

\[
H_{\text{Dicke}} = \frac{\Delta E}{2} \sum_{i=1}^N \sigma_z^{(i)} +  \hbar\omega\left(a^{\dagger}a +\frac{1}{2}\right) + U \sum_{i=1}^N (a^\dagger + a) (\sigma_+^{(i)} + \sigma_-^{(i)})
\label{eq:dickeH}
\]

The states of this system are described by
\(|n, \pm, \pm, \pm, \pm, ... \rangle\).

It's worth emphasising that there is no spatial dependence of the field
in Eq. \(\ref{eq:dickeH}\). One way to understand this physically is
that all the TLS are very close together in the sense that they are
located in a region of space that is much smaller than the wavelength of
the mode. In that situation, all the TLS will experience the same
strength of field at any given moment - in other words the field appears
constant in space. This is the how Dicke originally presented his ideas
in \href{https://journals.aps.org/pr/abstract/10.1103/PhysRev.93.99}{his
1954 paper}. It's also possible to use this Hamiltonian to describe many
TLS arranged in a very special way so that they are placed at integer
multiples of the mode wavelength.

Much like how the strength of the field changes the size of the coupling
term in the Rabi model (and also in the Dicke model), the number of TLS
also has an effect - this is the origin of what's called Dicke
superradiance. It is however difficult to see this right now because Eq.
\(\ref{eq:dickeH}\) is not written in a convenient form. We're going to
need to rewrite it by appealing to the physics of spin.

\subsection{Pseudo-spin}\label{pseudo-spin}

We noted earlier that the use of the Pauli spin matrices is just a
mathematical tool to describe two levels. Although we're not describing
spin here, we are working with what is often called ``pseudo spin''. The
\href{https://en.wikipedia.org/wiki/Angular_momentum_operator}{rules of
angular momentum} work just as well for pseudo angular momentum. In
particular, the rules of
\href{https://www.tcm.phy.cam.ac.uk/~bds10/aqp/lec7_compressed.pdf}{angular
momentum addition} and conservation.

This means that we can treat all the TLS together as if they are a
single object with many levels whose energies are determined by the
addition rules of pseudo angular momentum. The Hamilton then looks like:

\[
H =  \Delta E J_{z} + \hbar\omega\left(a^{\dagger}a +\frac{1}{2}\right) + 2U\left( a^{\dagger} + a \right)(J_{+} + J_{-})
\label{eq:dickeHpseudo}
\]

where the total pseudo total angular momentum operators (\(J\)) for
\(N\) TLS are:

\[
J_{+} + J_{-} = J_{x} = \frac{1}{2}\overset{N}{\underset{i=1}{\Sigma}} \sigma_{i x} \,\,\,\,\,\, J_{z} = \frac{1}{2}\overset{N}{\underset{i=1}{\Sigma}} \sigma_{i z}
\]

and noting that \(i\) in \(\sigma_i\) means that this operator only acts
on TLS number \(i\) .

When written in this way, states can now be described in terms of 3
numbers \(|n, j, m\rangle\) where \(j\) describes the total pseudo
angular momentum number (which is conserved) and \(m\) describes the z
component of the total pseudo angular momentum (which can change). This
notation allows us to conveniently describe situations where excitations
are ``delocalised'' among the TLS. A delocalised excitation means that
the excitation is shared among many TLS in such a way that you don't
know which TLS holds the excitation at any moment.

By far the most significant kind of delocalised states are called
``Dicke states'' which have the largest \(j=j_{\max} = N/2\). Dicke
states are symmetric in the sense that if you swap any of the TLS
around, the state remains unchanged. For example, consider a single
excitation among 4 TLS - the Dicke state looks like:

\[
\Psi = \frac{1}{\sqrt{4}}\left(| n, +, -, -, - \rangle + | n, -, +, -, - \rangle + | n, -, -, +, - \rangle + | n, -, -, -, + \rangle \right)
\]

Notice that if you swap any two TLS, the state looks the same.

The above state can instead be described by \(j_{\max}= 4/2  = 2\) and
\(m = 1\times 1/2 + 3\times -1/2 =-1\)

\[
\Psi = |n,2,-1>
\]

In this way of describing the system, the
\href{https://en.wikipedia.org/wiki/Ladder_operator\#Angular_momentum}{ladder
operators} \(J_{+}\) and \(J_{-}\) create and destroy delocalised
excitations of the combined TLS. This causes a raising and lowering of
the \(m\) value like this:

\[
J_+ |n, j, m\rangle  =  \sqrt{j(j + 1) - m(m + 1)} |n, j, m + 1\rangle
\]

\[
J_- |n, j, m\rangle =  \sqrt{j(j + 1) - m(m - 1)} |n, j, m - 1\rangle
\]

These ladder operators are conceptually similar to the creation and
annihilation operators of the field (see Eqs. \(\ref{eq:fieldcreate}\)
and \(\ref{eq:fielddestroy}\)). The details are however more complicated
due to the addition rules of angular momentum. Despite the complexity,
we know that the maximum total angular momentum \(j_{\max} = N/2\) (from
\(N\) TLS with pseudo angular momentum \(1/2\)). We can therefore see
that the number of TLS is going to have an effect on the coupling
between TLS and field. Exactly what effect depends on the details. Let's
explore some of those details next.

\subsection{Superradiance}\label{superradiance}

Dicke states with \(j=j_{\max}\) are significant because of the
acceleration properties that they offer; something people often describe
as superradiance.

Let's operate using \(J_-\) on some states. This represents a single
de-excitation of the combined TLS.

For the case with \textbf{all TLS excited} , \(m=N/2\):

\[
J_- |n, N/2, N/2\rangle  = \sqrt{N} |n, N/2, N/2 - 1\rangle
\]

This means that for the first de-excitation, the coupling terms gets
enhanced by \(\sqrt{N}\) . This might not seem surprising at first
glance because we have \(N\) TLS excited and so we should expect the
rate of emission (which go as the square of the coupling) to be enhanced
by \(N\).

For the case of \textbf{a single excitation} , \(m=-N/2 + 1\):

\[
J_- |n, N/2, -N/2 +1\rangle  = \sqrt{N}|n, N/2, -N/2\rangle
\]

For this singe de-excitation the coupling term also gets enhanced by
\(\sqrt{N}\) . This is more surprising because the rate of emission
(which go as the square of the coupling) to be enhanced by \(N\) even
though there is only a single excitation.

For the case of \textbf{50\% excitation}, \(m=0\):

\[
J_- |n, N/2, 0 \rangle  = \sqrt{N^2 + N}|n, N/2, -1\rangle
\]

For the first de-excitation, the coupling terms gets enhanced by
\(\sim\sqrt{N^2}\) for large \(N\). In other words, the rate of emission
(which go as the square of the coupling) to be enhanced by \(N^2\) -
this is where the super in superradiance comes from.

\subsubsection{Understanding
superradiance}\label{understanding-superradiance}

Superradiance might at first seem counterintuitive, but we can
understand it from one of the most fundamental principles of quantum
mechanics
\href{https://www.feynmanlectures.caltech.edu/III_01.html\#Ch1-S7}{according
to Richard Feynman} which reads:

\begin{quote}
``When an event can occur in several alternative ways, the probability
amplitude for the event is the sum of the probability amplitudes for
each way considered separately. There is interference''
\end{quote}

Let's take the example of 2 excitations amongst 4 TLS. The initial
delocalised Dicke state looks like:

\[
\Psi_i = \frac{1}{\sqrt{6}}\left(| 0, +, +, -, - \rangle + | 0, +, -, +, - \rangle + | 0, +, -, -, + \rangle + | 0, -, +, +, - \rangle + | 0, -, +, -, + \rangle + | 0, -, -, +, + \rangle \right)
\]

We can see there are 6 different configurations for the TLS. Each of the
2 excitations in those 6 configurations could transition from \(+\) to a
\(-\) with a release of a single field quanta. That means each of the 6
configurations has 2 emission paths that it could go in order to reach
one of 4 configurations in final the state:

\[
\Psi_f = \frac{1}{\sqrt{4}}\left(| 1, +, -, -, - \rangle + | 1, -, +, -, - \rangle + | 1, -, -, +, - \rangle + | 1, -, -, -, + \rangle \right)
\]

The total number of emission paths is therefore \(6\times 2 = 12\). Each
of these paths contributes the same to the overall emission amplitude
because the Dicke state is constructed with \(+\) between each of the
configurations that make up the state. This creates what's called
``constructive interference'' where the effects of each path add up to a
larger effect.

To get the numbers right, we must remember that our states are
normalised. The 6 configurations in our starting state means dividing
the amplitude by \(\sqrt{6}\). The 4 configurations in the final state
means dividing the amplitude by \(\sqrt{4}\). So the overall amplitude
enhancement factor is:

\[
\frac{6\times 2}{\sqrt{6}\sqrt{4}} = \sqrt{6}
\]

and so the probability enhancement factor (which is proportional to the
emission rate) is the square of this, i.e.~6. To derive a general
formula, we just have to do this counting and normalising for the
general case:

\[
\left(\frac{^N C_{N_+} N_+}{\sqrt{^N C_{N_+}}\sqrt{^N C_{{N_+}-1}}}\right)^2 = N_+\left(N-N_++1\right)
\]

where \(N_+\) is the number of excitations amongst the \(N\) TLS.

Now we can really understand just how important those ``\(+\)'s'' are
that make the Dicke state symmetric. As soon as you allow any ``\(-\)''
you reduce the emission rates. Take for example the case of 2 TLS with a
single delocalised excitation. If instead of a symmetric Dicke state \[
\Psi = \frac{1}{\sqrt{2}}\left(| 0, +, -\rangle + | 0, -, + \rangle \right)
\]

we instead use

\[
\Psi = \frac{1}{\sqrt{2}}\left(| 0, +, -\rangle - | 0, -, + \rangle \right)
\]

then we get no emission at all because we get complete destructive
interference of the two paths. Such states are often referred to as
``dark states''.

\subsubsection{Limitations of
superradiance}\label{limitations-of-superradiance}

The biggest limitation associated with superradiance is the requirement
for many TLS to be in a space much smaller than a wavelength so that all
TLS see the same field at any moment in time. This is another way of
saying that we have to make sure we can indeed use the Dicke Hamiltonian
in Eq. \(\ref{eq:dickeH}\) to describe the system.

Consider for example the acceleration of the decay of the
\(\rm 14\ keV\) transition of excited \(\rm ^{57}Fe\) as described in
\href{https://www.nature.com/articles/s41567-017-0001-z}{Chumakov 2017}.
There, a \(\rm 14\ keV\) x-ray laser is responsible for exciting the
\(\rm ^{57}Fe\) initially. In such an experiment, it's possible to
create a Dicke state with an extremely large \(N\) proportional to the
spot size of the laser. In the case of iron with a surface density on
the order of \(10^{15} \ \rm m^{-2}\) with an x-ray spot size of
\(1 \ \rm \mu m\) there are \(N=10^7\) atoms.

While we might delight in the prospect of a \(10^{7}\) enhancement of
the decay rate, we must also consider the wavelength of the mode that
the atoms are eventually going to collectively emit into. The wavelength
associated with \(\rm 14\ keV\) is only about \(\sim 1\AA\). Since the
number density of iron is about \(8\times 10^{28} \rm m^{-3}\) then we'd
have less than 1 atom per cubic wavelength and so we'd not be able to
apply the Dicke Hamiltonian. Although all \(10^{7}\) atoms see the same
initial \(\rm 14\ keV\) exciting x-ray beam, because the collective
emission happens over distances spanning many wavelengths then there is
potential for interference which can destroy the superradiance effects.
Indeed in the Chumakov paper, they only observed a factor of 10 increase
in the decay rate.

One way to overcome the limitation described above is to use a mode with
a lower frequency and so a longer wavelength. There is however a price
to pay for this approach. If the field and TLS are not matched,
i.e.~\(\hbar\omega \neq \Delta E\) , then virtual transitions must take
place in which quanta after quanta are ``emitted'' until there are
enough field quanta \(n\) so that energy balance is reached
i.e.~\(n\hbar\omega = \Delta E\). This is known as downconversion. The
rate of this process scales at most as fast as
\((nN^2U^2)^{\Delta E / \hbar\omega}\) and so the only way this process
can work is if there are enough TLS (\(N\)) and the field is initially
strong enough (\(n\)) to make the overall coupling large.

Alternatively if we have two groups of TLS, system A and system B and
they are energy matched in the sense that \(\Delta E_A = \Delta E_B\)
then a mismatched field can be used to increase the rate of excitation
transfer between A and B without the constraints described above because
the field does not have to be ``responsible'' for the energy balance in
the final state. We'll explore this more in the next section.

\subsection{Supertransfer}\label{supertransfer}

\begin{quote}
TODO
\end{quote}

\printbibliography


\end{document}
