% Options for packages loaded elsewhere
\PassOptionsToPackage{unicode}{hyperref}
\PassOptionsToPackage{hyphens}{url}
\documentclass[
]{article}
\usepackage{xcolor}
\usepackage{amsmath,amssymb}

\usepackage{hyperref}
\usepackage{ulem}  % For underlining links
\normalem

\hypersetup{
    colorlinks=false,         % Disable colored links (boxes will appear)
    citebordercolor=red,      % Border color for citation links
    linkbordercolor=red,      % Border color for internal links
    urlbordercolor=blue,      % Border color for external links (URLs)
    pdfborder={0 0 2}         % Specifies the border thickness: {horizontal vertical thickness}
}


% Only underline external links (URLs)
\let\oldhref\href
\renewcommand{\href}[2]{\ifx#1\urlprefix\oldhref{#1}{#2}\else\uline{\oldhref{#1}{#2}}\fi}


\renewcommand{\[}{\begin{equation}}
\renewcommand{\]}{\end{equation}}


\setcounter{secnumdepth}{5}
\usepackage{iftex}
\ifPDFTeX
  \usepackage[T1]{fontenc}
  \usepackage[utf8]{inputenc}
  \usepackage{textcomp} % provide euro and other symbols
\else % if luatex or xetex
  \usepackage{unicode-math} % this also loads fontspec
  \defaultfontfeatures{Scale=MatchLowercase}
  \defaultfontfeatures[\rmfamily]{Ligatures=TeX,Scale=1}
\fi
\usepackage{lmodern}
\ifPDFTeX\else
  % xetex/luatex font selection
\fi
% Use upquote if available, for straight quotes in verbatim environments
\IfFileExists{upquote.sty}{\usepackage{upquote}}{}
\IfFileExists{microtype.sty}{% use microtype if available
  \usepackage[]{microtype}
  \UseMicrotypeSet[protrusion]{basicmath} % disable protrusion for tt fonts
}{}
\makeatletter
\@ifundefined{KOMAClassName}{% if non-KOMA class
  \IfFileExists{parskip.sty}{%
    \usepackage{parskip}
  }{% else
    \setlength{\parindent}{0pt}
    \setlength{\parskip}{6pt plus 2pt minus 1pt}}
}{% if KOMA class
  \KOMAoptions{parskip=half}}
\makeatother
\setlength{\emergencystretch}{3em} % prevent overfull lines
\providecommand{\tightlist}{%
  \setlength{\itemsep}{0pt}\setlength{\parskip}{0pt}}
\usepackage[]{biblatex}
\addbibresource{refs.bib}
\usepackage{bookmark}
\IfFileExists{xurl.sty}{\usepackage{xurl}}{} % add URL line breaks if available
\urlstyle{same}

\title{Strong-field quantum phonodynamics}
  \author{Matt Lilley}
  \date{\today}  % Default to today if no date is provided

\begin{document}
\maketitle

\section{Introduction}\label{introduction}

I had a chat with my friend Chris Ridgers about his work on laser plasma
interactions in a regime described as ``Strong-field QED''. The laser
field is so intense that perturbation theory no longer works. Chris
remarked on how part of the work involved using the ``low'' energy laser
photons to cause an electron to emit a photon with many orders or
magnitude more energy. Superficially this sounded similar to the
phonon-nuclear system in which we use a low energy phonon to mediate a
nuclear energy transition. We also have a highly excited phonon field
that makes the system non-perturbative. Chris described various
mathematical tools used in his work and I wondered whether we might be
able to use some of them in our work.

In this document, I'll try and describe a regime that we might call
``Strong-field quantum phonodynamics'' or ``Strong-field QPD'' and
compare/contrast it with what people sometimes refer to as ``Deep strong
coupling''.

Unlike ``Strong-field QED'' that has as well defined parameter range
given in terms of the ``normalised vector potential''
\(a_0\equiv eE/m_e\omega c\gg 1\), for ``Strong field QPD'' I've not yet
come up with a single parameter. Let's see whether one comes out of the
analysis in this document.

\section{Dicke model}\label{dicke-model}

\emph{For more detail on the Dicke model, see the
\href{https://github.com/project-ida/notes/blob/main/pdf/Dicke\%20model.pdf}{notes}
I made on the subject.}

\subsection{Describing the states}\label{describing-the-states}

The Dicke model describes a system where we have \(N\) identical TLS
coupled to a single mode (i.e.~single frequency/wavelength) of a
quantised field. The Dicke Hamiltonians can be written as:

\[
H =  \Delta E J_{z} + \hbar\omega\left(a^{\dagger}a +\frac{1}{2}\right) + 2U\left( a^{\dagger} + a \right)(J_{+} + J_{-})
\label{eq:dickeHpseudo}
\]

where \(\Delta E\) is the transition energy between the 2 levels of the
TLS, \(\hbar\omega\) is the energy of each quantum of the field, and
\(U\) is the coupling constant between the TLS and the field. The
\(a^{\dagger}\), \(a\) are the field creation and annihilation operators
respectively and the \(J\) operators are total pseudo angular momentum
operators:

\[
J_{+} + J_{-} = J_{x} = \frac{1}{2}\overset{N}{\underset{i=1}{\Sigma}} \sigma_{i x} \,\,\,\,\,\, J_{z} = \frac{1}{2}\overset{N}{\underset{i=1}{\Sigma}} \sigma_{i z}
\]

Here \(\sigma\) are the
\href{https://ocw.mit.edu/courses/5-61-physical-chemistry-fall-2007/3b1fb40c61e7f939861b190bedbc57a7_lecture24.pdf}{Pauli
spin matrices} the \(i\) in \(\sigma_i\) means that this operator only
acts on TLS number \(i\) .

When written in this way, states are described in terms of 3 numbers
\(|n, j, m\rangle\) where \(n\) describes the number of field quanta,
\(j\) describes the total pseudo angular momentum number (which is
conserved) and \(m\) describes the z component of the total pseudo
angular momentum (which can change). This notation allows us to
conveniently describe situations where excitations are ``delocalised''
among the TLS. By far the most significant kind of delocalised states
are called ``Dicke states'' which have the largest \(j=j_{\max} = N/2\).
These states are capable of accelerated emission rates due to
superradiance. Dicke starts are symmetric in the sense that if you swap
any of the TLS around, the state remains unchanged. For example,
consider a single excitation in 4 TLS - the Dicke state written in
\(|n,\pm,\pm, \pm, \pm\rangle\) notation looks like:

\[
\Psi = \frac{1}{\sqrt{4}}\left(| n, +, -, -, - \rangle + | n, -, +, -, - \rangle + | n, -, -, +, - \rangle + | n, -, -, -, + \rangle \right)
\]

Notice that if you swap any two TLS, the state looks the same.

The above state can instead be described by \(j_{\max}= 4/2  = 2\) and
\(m = 1\times 1/2 + 3\times -1/2 =-1\)

\[
\Psi = |n,2,-1>
\]

\subsection{Coupling strength}\label{coupling-strength}

The strength of the interaction between the TLS and the field is not
only determined by the constant \(U\) but also how many field quanta
\(n\) we have. This is because of how the field operators work:

\[
a^{\dagger} |n,j,m\rangle = \sqrt{n+1}|n+1,j,m\rangle
\label{eq:fieldcreate}
\]

\[
a |n,j,m\rangle = \sqrt{n}|n-1,j,m\rangle \\
\label{eq:fielddestroy}
\]

\[
a^{\dagger}a |n,j,m\rangle = n|n,j,m\rangle
\]

The more field quanta we have (the stronger the field), the larger the
coupling terms will be compared to the TLS term which remains unchanged.

The coupling is also affected by the number of TLS we have. This is
because of how the the
\href{https://en.wikipedia.org/wiki/Ladder_operator\#Angular_momentum}{ladder
operators} \(J_{+}\) and \(J_{-}\) create and destroy excitations of the
TLS. This causes a raising and lowering of the \(m\) value like this:

\[
J_+ |n, j, m\rangle  =  \sqrt{j(j + 1) - m(m + 1)} |n, j, m + 1\rangle
\]

\[
J_- |n, j, m\rangle =  \sqrt{j(j + 1) - m(m - 1)} |n, j, m - 1\rangle
\]

These ladder operators are conceptually similar to the creation and
annihilation operators of the field (see Eqs. \(\ref{eq:fieldcreate}\)
and \(\ref{eq:fielddestroy}\)). The details are however more complicated
due to the addition rules of angular momentum. Despite the complexity,
we know that the maximum total angular momentum \(j_{\max} = N/2\) (from
\(N\) TLS with pseudo angular momentum \(1/2\)). We can therefore see
that the number of TLS is going to have an effect on the coupling
between TLS and field. The effect scales like at least \(\sqrt{N}\) and
at most \(N\) (see
\href{https://github.com/project-ida/notes/blob/main/pdf/Dicke\%20model.pdf}{Dicke
model notes} for more detail).

The coupling term in Eq. \(\ref{eq:dickeHpseudo}\) therefore
conservatively scales like \(U\sqrt{N}\sqrt{n}\) and optimistically as
\(UN\sqrt{n}\).

\section{Deep strong coupling}\label{deep-strong-coupling}

In the notes on
\href{https://github.com/project-ida/notes/blob/main/pdf/Deep\%20strong\%20coupling.pdf}{Deep
strong coupling}, we described a regime in which all the terms in the
Hamiltonian in Eq. \(\ref{eq:dickeHpseudo}\) were of the same order. In
other words:

\[
\Delta E \sim n\hbar\omega \sim U\sqrt{N}\sqrt{n}
\label{eq:deepstrongcoupling}
\]

where we have taken the conservative Dicke scaling of \(\sqrt{N}\). This
regime is characterised by a
\href{https://en.wikipedia.org/wiki/Dicke_model\#Superradiant_transition_and_Dicke_superradiance}{superradiant
phase transition}.

For a given \(\Delta E, \hbar\omega\), Eq.
\(\ref{eq:deepstrongcoupling}\) fixes the strength of the field \(n\)
and the number of TLS \(N\). For nuclear transitions mediated by phonons
via a relativistic phonon nuclear coupling, we found that we could
\textbf{not} satisfy Eq. \(\ref{eq:deepstrongcoupling}\).

Mathematically, the reason Eq. \(\ref{eq:deepstrongcoupling}\) is hard
to satisfy is because \(n\hbar\omega \sim U\sqrt{N}\sqrt{n}\) limits how
large \(n\) can be because \(n\) grows faster than \(\sqrt{n}\).

What happens if we look for a regime in which \(n\) is not constrained?

\section{Strong field}\label{strong-field}

Let's imagine that the field is so strong, i.e.~\(n\) is so large, that
the following is satisfied:

\[
\Delta E \lesssim U\sqrt{N}\sqrt{n} \ll n\hbar \omega
\label{eq:strongfield}
\]

In this regime, the field retains its ``identity'' but the TLS gets
significantly altered by the field and the coupling.

Eq. \(\ref{eq:strongfield}\) can in principle be satisfied for any type
of coupling by just increasing the field strength. In practice, there
will be physical limits on how strong a field can be. Let's try and work
through an example using relativistic phonon nuclear coupling to mediate
a nuclear transition with phonons.

\subsection{Relativistic phonon nuclear
coupling}\label{relativistic-phonon-nuclear-coupling}

From notes on
\href{https://github.com/project-ida/notes/blob/main/pdf/Coupling\%20constants\%20in\%20nuclear\%20physics.pdf}{Coupling
constants in nuclear physics}, we derived the relativistic phonon
nuclear coupling as:

\[
\frac{U}{\hbar \omega_A} = \sqrt{\frac{2}{N}} \sqrt{\frac{\Delta E}{M c^2}} \sqrt{\frac{\Delta E}{\hbar \omega_A}} \times 10^{-3}
\label{eq:phononcoupling}
\]

where \(N\) is the number of nuclei involved in the phonon motion, and
\(M\) is the mass of the nucleus and \(\omega_A\) is the acoustic phonon
mode frequency. The first condition in Eq. \(\ref{eq:strongfield}\)
gives: \[
\begin{aligned}
\Delta E &\lesssim U\sqrt{N}\sqrt{n} \\
\frac{\Delta E}{\hbar \omega_A} &\lesssim \sqrt{2} \sqrt{\frac{\Delta E}{M c^2}} \sqrt{\frac{\Delta E}{\hbar \omega_A}} \times 10^{-3}\sqrt{n} \\
1 &\lesssim \sqrt{2} \sqrt{\frac{\hbar\omega_A}{M c^2}} \times 10^{-3}\sqrt{n} \\
\frac{1}{2}\frac{Mc^2}{\hbar\omega_A}\times 10^6 &\lesssim n \\
\frac{1}{2}Mc^2\times 10^6  &\lesssim n\hbar\omega_A
\end{aligned}
\label{eq:strongfieldconditiononn}
\]

For palladium nuclear transitions mediated by acoustic phonons,
\(M c^2 \approx 10^{11}\) eV and \(\hbar \omega_A \approx 10^{-8}\) eV .
This tells us that the number of phonons must be at least:

\[
n \gtrsim 5\times 10^{24}
\]

And the total energy of the phonons is at least:

\[
\frac{1}{2}\times 10^{11} \times 10^{6} \times 1.6\times 10^{-19} = 8 \, \rm mJ
\label{eq:minfieldenergy}
\]

This does not seem like an outrageous amount of energy.

The second condition in Eq. \(\ref{eq:strongfield}\) gives:

\[
\begin{aligned}
U\sqrt{N}\sqrt{n} &\ll  n\hbar \omega_A \\
\sqrt{2} \sqrt{\frac{\Delta E}{M c^2}} \sqrt{\frac{\Delta E}{\hbar \omega_A}} \times 10^{-3}\sqrt{n} &\ll n \\
2\frac{\Delta E}{M c^2} \frac{\Delta E}{\hbar \omega_A} \times 10^{-6} \ll n
\end{aligned}
\]

For a \(\rm 24 \ MeV\) palladium transition mediated by acoustic phonons
we have:

\[
n \gg 10^{6}
\] So, we can describe some different regions of \(n\) for this
palladium example:

\begin{itemize}
\tightlist
\item
  \(n<10^{6}\) - weak field, the coupling term is greater than the total
  field energy but is a lot less than the TLS energy.
\item
  \(10^6 < n < 5 \times 10^{24}\) - intermediate field, the field and
  the TLS dominate over the coupling.
\item
  \(n > 5\times 10^{24}\) - strong field, the field dominates but now
  the coupling is bigger than the TLS energy
\end{itemize}

When the field is strong, and the field quanta have a low enough energy
such that \(U\sqrt{N}>\hbar\omega\) then there can be a free exchange of
energy between the field and the TLS because even though an individual
low energy energy quanta cannot ``hold'' all the energy of a TLS
transition, the coupling term can. Another way of thinking about this is
that incredibly unlikely transitions like the downconversion of nuclear
energy into phonon energy or the upconversion of phonon energy to
nuclear energy become possible.

\subsubsection{Ion traps}\label{ion-traps}

Although the phonon energy doesn't seem that large - Eq.
\(\ref{eq:minfieldenergy}\) gave \(8 \ \rm mJ\) - it can be a bit
deceiving. For example, one might imagine giving \(8 \ \rm mJ\) of
energy to a single nucleus via the electric field of an ion trap such as
a Paul trap like the one used in
\href{https://www.nature.com/articles/s41467-021-21425-8}{Cai 2021}.
Such a setup might allow us to explore relativistic phonon nuclear
coupling of a single nucleus. However, we must consider both the
physical limits on the size of the electric field that would drive the
oscillatory motion and the size of the experiment over which
acceleration of the nucleus occurs.

From notes on
\href{https://github.com/project-ida/notes/blob/main/pdf/Coupling\%20constants\%20in\%20nuclear\%20physics.pdf}{Coupling
constants in nuclear physics}, we derived the magnitude of an
oscillating electric field associated with phonon motion to be:

\[
E = \frac{\omega_A \sqrt{2M n\hbar \omega_A}}{Ze \sqrt{N}} \label{eq:E}
\]

The electric field depends on the frequency and so in principle we can
bring the field down to manageable levels. However, the smaller the
field, the greater than size of the experiment because smaller field has
to act over a greater distance in order to produce phonon energies
required from Eq. \(\ref{eq:strongfieldconditiononn}\).

A simple test of practicality is to substitute the largest practical
\(E\) field of \(\sim 10^{11}\). It should be noted that this field is
used for acceleration of charged particles and not for trapping of ions.
It will however give us a good bound for how large a hypothetical trap
might need to be.

For a single palladium nucleus with \(Z \sim 50\), the size of the trap
\(d\) can be estimated from force multiplied by distance:

\[
\begin{aligned}
ZeEd &= 8\times 10^{-3} \\
d  &= \frac{8\times 10^{-3}}{ZeE} \\
d  &\sim \frac{5\times 10^{16}}{50 \times 10^{11}} \\
d &\sim 10^4 \ \rm m
\end{aligned}
\]

A monstrously large ion trap.

A smaller electric field would only make the the situation worse and so
we can conclude that the strong field regime cannot be accessed for a
single nucleus in an ion trap.

In some ways this might not be so surprising because for a single
nucleus, we're requiring \(\frac{1}{2}Mc^2\times 10^6 \ \rm eV\) to be
transferred to it - one million times more energy than it's rest mass
energy!!!

One might hope to alleviate the problems described above by adding more
nuclei to the trap and thus bringing down the energy per nucleus.
However, in order to avoid collective non-neutral plasma effects coming
into play we'd need to limit \(N < 1000\). This would still make the
trap \(\sim 10 \ \rm m\) which is orders of magnitude larger that
typical trap sizes.

\subsection{Magnetic dipole coupling}\label{magnetic-dipole-coupling}

From notes on
\href{https://github.com/project-ida/notes/blob/main/pdf/Coupling\%20constants\%20in\%20nuclear\%20physics.pdf}{Coupling
constants in nuclear physics}, we derived the magnetic dipole coupling
as: \[
U \approx 0.02 \frac{{\mu_N}B}{\sqrt{n}}
\] where \(\mu_N = e\hbar/m_p \approx 5\times 10^{-27} \ \rm J/T\) is
the nuclear magneton

The first condition in Eq. \(\ref{eq:strongfield}\) gives: \[
\begin{aligned}
\Delta E &\lesssim U\sqrt{N}\sqrt{n} \\
{\Delta E} &\lesssim  0.02 \mu_N B \sqrt{N} \\
1 &\lesssim 0.02\sqrt{N}\frac{\mu_N B}{\Delta E} \\
\end{aligned}
\] We can turn this into a condition for the number of nuclei \(N\) by
using the largest reasonable magnetic field strength of
\(B\sim 0.1 \ \rm T\). Let's look at the \(\rm 14 \ keV\) nuclear
transition of \(\rm ^{57}Fe\) . \[
\begin{aligned}
N &\gtrsim \left(50\frac{\Delta E}{\mu_N B}\right)^2 \\
&\gtrsim 2500 \times \left(\frac{14 \times 10^3 \times 1.6\times 10^{-19}}{5\times 10^{-27}\times 0.1}\right)^2
\\
&\gtrsim 5 \times 10^{28}
\end{aligned}
\] This works out at about \(5000 \ \rm kg\) of iron - at the very
least!

The second condition in Eq. \(\ref{eq:strongfield}\) gives: \[
\begin{aligned}
U\sqrt{N}\sqrt{n} &\ll  n\hbar \omega \\
0.02 \mu_N B \sqrt{N} &\ll \frac{1}{\mu_0} B^2 V  \\
N \ll \left(50\frac{1}{\mu_0\mu_N}BV\right)^2
\end{aligned}
\] Again we'll use \(B=0.1 \ \rm T\) . For the volume let's use the
volume of \(5 \times 10^{28}\) iron atoms. The density of iron is about
\(8000 \ \rm kg/m^{-3}\) so \(V \approx 0.6 \ \rm m^3\) . Substituting
the numbers gives: \[
N \ll 2\times 10^{65}
\] This condition is well satisfied.

If we were to magnetic quanta with energy \(4 \ \rm neV\) which have
frequency \(f \approx 1 \ \rm MHz\) and wavelength
\(\lambda \approx 300 \ \rm  m\) (which would define a coherence domain
for the Dicke model) then we could in principle reach the strong field
regime using magnetic coupling. It would be a LOT of iron of course and
the solid would be very large too.

\printbibliography


\end{document}
